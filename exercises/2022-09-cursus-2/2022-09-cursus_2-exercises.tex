\documentclass[a4paper]{article}

\author{TeXniCie}
\date{\today}
\title{Manual for thesis}

%%% Overview of this file in order:
% Packages which don't need options
% Packages which have one or few options
% Geometry package
% Header/footer settings
% Theorem styles
% Enable/disable parindents
% Reference and bibliography settings
% Front/main/back-matter


%%%%%%%%%%%%%%%%%%%%%%%%%%%%%%%%%%%%%%%%%

\usepackage{
		%layout,		% Allow visualisation of all the margins
		subfiles,		% For separate main and sub documents
		graphicx,		% For image modifications and the figure environment
		amsmath,		% For the AMS math styles
		amssymb,		% The extended AMS math symbol list
		amsthm,			% For use of theorems (works together with thmtools)
		fancyhdr,		% For fancy headers and footers on pages
		%gensymb,		% For easy generic symbols (uniform use in math and text mode)
		%sidecap,		% For use of captions next to a float (figure, table, etc)
		subcaption,		% For easy subfigures in a plot (with nice captions)
		tikz,			% Difficult drawing of awesome vector plots
		%listings,		% For listing pieces of code in a nice and neat way
		multicol,		% For easy local multicolumn use
		color,			% For handy colour definitions (used cause of styling)
		%calc,			% To calculate stuff for the back-end
		%mdwlist,		% For customising lists
		thmtools,		% Lets you define your own theorem style (used for all the fancy theorems, definitions etc.)
		etoolbox,		% Allows adjustment of commands (used to reset the claim counter at the end of a proof).
		xspace,			% Makes latex not eat spaces after commands
        hyperref,		% Makes links, references, the Table of Contents, etc. clickable.
        mathtools,      % Extensible symbols, such as brackets, arrows, harpoons, etc.;
        inputenc,       % Allows to use things like ö instead of \"o in your text.
        wrapfig,        % wrap text around a smaller figure
        framed          % allows for boxes around your important equations or theorems
        }

%%%%%%%%%%%%%%%%%%%%%%%%%%%%%%%%%%%%%%%%%%

\usepackage[english]{babel} % Correct language setting, 'british', 'american'='english' or 'dutch'.
\usepackage[autostyle]{csquotes} % Fixes quotes to correspond to the babel language.
% Note the difference between ``quotes'' and ''quotes'' when using different languages.


%%%%%%%%%%%%%%%%%%%%%%%%%%%%%%%%%%%%%%%%%%

 \usepackage[margin=2.5cm]{geometry}
 % Change the shape of a page (custom margins etc.)
 % paper=a4paper slightly changes the style through the whole document.
    %%%% We set the margins for whole document here, except the titlepage. The titlepage uses special margins; see titlepage.tex.


%%%%%%%%%%%%%%%%%%%%%%%%%%%%%%%%%%%%%%%%%%

%%% This is about changing the headers and footers (i.e. Top and bottom of the page)

\pagestyle{fancy}% use fancyheaders with the bar on the top
\fancyhf{} % Clear the normal style
\fancyhead[L]{\bfseries\leftmark} %this places the section number and name in the top left
\fancyhead[R]{\bfseries\thepage}% this places the pagenumber in the top right
	

%%%%%%%%%%%%%%%%%%%%%%%%%%%%%%%%%%%%%%%%%%%%
%%%%		Theorem style

% The set-up is as follows, first you give the 'style' of your theorem. This determines whether it for instance is plain, or italic. Secondly you can give an option for the symbol on the end, normally it is nothing. But you could add some to increase the readability of your text. Finally you can use numberwithin to add the number of your section/theorem before your equations. This is useful if you want to keep the numbers of your equation in check (In this thesis there where over a 100) and keeps in order where the equations are.
%Finally you can use sibling to let different 'theorems' count together. Hence you will get Theorem 1 Definition 2 Claim 3, instead of Theorem 1 Definition 1 Claim 1. This is a matter of taste.

% Theorem definitions
\declaretheorem[style=definition,numberwithin=subsection]{definition} %If you want your theorems to be counted per section instead of subsection, then just remove the sub from the numberwithin
\declaretheorem[style=definition,qed=$\triangle$,sibling=definition]{example}% sibling says with what type of theorems you wan the numbering to count with.

\declaretheorem[style=plain,sibling=definition]{theorem}
\declaretheorem[style=plain,sibling=definition]{lemma}
\declaretheorem[style=plain,sibling=definition]{proposition}
\declaretheorem[style=plain,sibling=definition]{corollary}
\declaretheorem[style=definition]{claim}
\declaretheorem[style=definition,sibling=example]{remark}

\AtEndEnvironment{proof}{\setcounter{claim}{0}} % Sets the claim number to 0 after ending a proof

% You can make short-hands like these. 

\newcommand{\thm}[2]{\begin{theorem} #1 \begin{proof} #2 \end{proof} \end{theorem}}
\newcommand{\lm}[2]{\begin{lemma} #1 \begin{proof} #2 \end{proof} \end{lemma}}
\newcommand{\df}[1]{\begin{definition} #1 \end{definition}}
\newcommand{\clm}[1]{\begin{claim} #1 \end{claim}}


%%%%%%%%%%%%%%%%%%%%%%%%%%%%%%%%%%%%%%%%%%%%

%Let equation numbers be numbered by 1.1, 1.2, 2.1, etc where the first number is the section number. section can also be replaced by chapter when using book class.
\numberwithin{equation}{section}

%%%%%%%%%%%%%%%%%%%%%%%%%%%%%%%%%%%%%%%%%%%%

% Comment/uncomment the following to disable/enable parindents:
\setlength\parindent{0pt}

%%%%%%%%%%%%%%%%%%%%%%%%%%%%%%%%%%%%%%%%%%%%

%%%% Add the bibliography with some settings:
% package:
\usepackage[% Options
style = numeric-comp, % 
% Choose the style of your citations (and correspondingly your bibliography).
% Few examples:
% numeric = [15, 16, 17, 20], numeric-comp = [15-17, 20], numeric-verb = [15]; [16]; [17]; [20], alphabetic = [Jon99, Wil93, BT86, Zil13], authoryear = Jones 99, Wilfred 93, Bohr, Turing 86, Ziltener 13.
%
% List of all: (you probably want a version of numeric, alphabetic or authoryear)
% numeric, numeric-comp, numeric-verb,
% nature (like numeric, but with '23.' instead of '[23]' in the bibliography),
% apa (does not work well with out with only 'year'; really needs a full date)
% alphabetic, alphabetic-verb,
% authoryear, authoryear-comp, authoryear-ibid, authoryear-icomp,
% authortitle, authortitle-comp, authortitle-ibid, authortitle-icomp, authortitle-terse, authortitle-tcomp, authortitle-ticomp,
% verbose, verbose-ibid, verbose-note, verbose-inote, verbose-trad1, verbose-trad2, verbose-trad3,
% reading, (draft, debug)
sorting = none, % 
% Choose how the bibliography is sorted.
% Options: nty, nyt, nyvt, anyt, anyvt, ynt, ydnt, none, (debug)
% Here n = name, t = title, y = year, v = volume, a = alphabetic label, ...d = ... descending
% So nty = sort by name, then title, then year.
backend = biber%
% This is the default, and should almost alway be kept as such. 'backend = bibtex' is legacy.
]{biblatex}

% If you use APA, you will need:
% \DeclareLanguageMapping{english}{english-apa}

% There exist related packages for specific styles like biblatex-chicago (Chicago manual of style citations) or biblatex-jura (German legal citations). You most likely won't need them or use them.

% For more information and a nice matrix with typesupport, see:
% https://en.wikibooks.org/wiki/LaTeX/Bibliography_Management#biblatex

% The source file for you bibliography:
\addbibresource{bibfile.bib}
% It's possible to add multiple bib files and separate them based on label (so to have two different references lists e.g. to seperate main sources from minor sources or books and theses from misc sources); see in 3.7 of the BibLaTeX documentation if you want to do stuff like that.

%%%%%%%%%%%%%%%%%%%%%%%%%%%%%%%%%%%%%%%%%%%%

%%%%% frontmatter/mainmatter/backmatter:
\newcommand\frontmatter{%
    \cleardoublepage
    \pagenumbering{roman}} %small Roman numbers

\newcommand\mainmatter{%
    \cleardoublepage
    \pagenumbering{arabic}} %normal numbers

\newcommand\backmatter{%
    \cleardoublepage %% double page style
    %\clearpage %% single page style
    \pagenumbering{Roman}} %capital Roman numbers
   




\title{Oefeningen \LaTeX-cursus week 2}
\author{\TeX niCie\\(Vincent Kuhlmann)}
\date{3 oktober 2022}

\usepackage{minted}
\setminted[tex]{fontsize=\small, autogobble=true, linenos=false, frame=none}

\setcounter{secnumdepth}{0}

\begin{document}
    \maketitle

    % \CheckBox[]{aaa}

    \section{Deel 1: Document, referenties en `Theorem'}
    Vergeet de volgende packages niet toe te voegen aan je preamble:
    \begin{minted}{tex}
        \usepackage[a4paper,margin=2.54cm]{geometry}
        \usepackage{amsmath,amssymb,amsthm}
        \usepackage[bookmarksnumbered]{hyperref}
    \end{minted}
    \bigskip

    \begin{exercise}[Geometry]\label{ex:aaaa}
        Maak een A6-document in landscape met voorbeeldtekst van \nolinkurl{lipsum.com}. Zet de
        horizontale marges op $ 2\text{ cm} $ en vertical marges op $ 3\text{ cm} $.

        Hint: De volgende opties van geometry kunnen van pas komen: left, right, top, bottom,
        vmargin, hmargin, landscape, a6paper.\\
        Documentatie over gebruik van geometry package: \url{https://ctan.org/pkg/geometry}
    \end{exercise}

    \begin{exercise}[Titels]
        Voeg een paar \mintinline{tex}{\section}'s toe aan je bestand, en een
        table of contents op zijn eigen pagina.

        Het \mintinline{tex}{\section} commando laat een optioneel argument toe.
        Voeg een section ermee toe, bijvoorbeeld
        \mintinline{tex}{\section[Intro]{Introductie}} en kijk wat er gebeurt in je table of
        contents.
    \end{exercise}

    \begin{exercise}[PDF TOC]
        Voeg \mintinline{tex}{\usepackage[bookmarksnumbered]{hyperref}} toe aan
        je preamble. Download je document als PDF en open het. Kijk of je de
        table of contents van je PDF-lezer kan vinden.

        Wat gebeurt er als je \mintinline{tex}{bookmarksopen} toevoegt als option?
    \end{exercise}

    \begin{exercise}[URL's]
        Voeg de volgende link toe aan je bestand:\\
        \nolinkurl{https://en.wikipedia.org/wiki/Electromagnetic_tensor}
        \begin{enumerate}[label=\alph*)]
            \item Wat gebeurt er als je de link direct in je code plakt?
            Kan je die foutmelding fixen?
            \item Plak nu dezelfde link in het argument van het \mintinline{tex}{\url}-commando
            van hyperref: \mintinline{tex}{\url{...}}. Heb je dezelfde fix nog nodig?
            \item Wat gebeurt er als je de \texttt{https://} weglaat?
        \end{enumerate}
    \end{exercise}

    \begin{exercise}[\textbackslash eqref]
        De amsmath package definieert het commando \mintinline{tex}{\eqref{...}}.
        Voeg een genummerde vergelijking toe aan je document, met een label, en
        kijk wat het verschil is tussen \mintinline{tex}{\ref{...}}
        en \mintinline{tex}{\eqref{...}}.
    \end{exercise}

    \begin{exercise}[Labels]
        Wat gebeurt er als je aan een niet-bestaande label refereert?
    \end{exercise}

    \begin{exercise}[Stelling met bewijs]
        Voeg een theorem met proof toe in je bestand voor je favoriete stelling of bewijs.
    \end{exercise}

    \begin{exercise}[Definitie]
        Voeg een `Definitie' toe aan je bestand, en refereer eraan in je bestand.
    \end{exercise}

    \section{Deel 2: Figuren, matrices en tabellen}

    Vergeet de volgende packages niet toe te voegen aan je preamble:
    \begin{minted}{tex}
        \usepackage[a4paper,margin=2.54cm]{geometry}
        \usepackage{amsmath,amssymb,amsthm}
        \usepackage{graphicx}
        \usepackage{subcaption}
        \usepackage{booktabs}
        \usepackage[bookmarksnumbered]{hyperref}
    \end{minted}
    \bigskip

    \begin{exercise}[Figure]
        Is het mogelijk een figure environment te maken zonder \mintinline{tex}{\includegraphics}?
    \end{exercise}

    \begin{exercise}[Figuurplaatsing]
        Cre\"eer een scenario waarbij \LaTeX{} je figuurplaatsingsadvies niet opvolgt.
    \end{exercise}

    \begin{exercise}[Subfigure]
        Maak een figure met veel subfigures erin. Kijk wat de verschillende parameters
        doen. Wat doet de \mintinline{tex}{\textwidth}? Wat doet de \mintinline{tex}{[b]}?
    \end{exercise}

    \begin{exercise}[Matrix]
        Maak een matrix met een verticale streep langs beide kanten, zoals de
        notatie voor determinant van een matrix. Kan je vinden welke environment (eindigend op
        matrix) dit al standaard doet?
    \end{exercise}

    \begin{exercise}[Align]
        Hoe gedraagt het align-environment zich als je meer dan twee `kolommen' hebt?
    \end{exercise}

    \begin{exercise}[Wiskunde in tabellen]
        Maak een simpele tabel met wat woorden, nummers en wiskundige symbolen erin, b.v. $ \sqrt{2} $.
    \end{exercise}

    \begin{exercise}[Kolomscheidingen]
        Wat gebeurt er als een regel te veel kolommen heeft? En wat als het te
        weinig kolommen heeft?
    \end{exercise}

    \begin{exercise}[Alignering]
        Zoek op wat de mogelijke aligneringen zijn voor een kolom in een tabular
        en probeer ze uit.
    \end{exercise}

    \begin{exercise}[Booktabs]
        Maak een tabel waarbij je \mintinline{tex}{\toprule}, \mintinline{tex}{\midrule}
        en \mintinline{tex}{\bottom} van booktabs gebruikt (zie slides), om
        een goed uitziende tabel te krijgen. Je kan \mintinline{tex}{\cmidrule} gebruiken om een
        gedeeltelijke horizontale lijn te krijgen.
    \end{exercise}

    \begin{exercise}[Excellent]
        Gebruik \mintinline{tex}{\multicolumn} om op een rij twee kolommen samen te voegen
        (zoek op hoe het commando werkt, of deduceer het van de slides). Als je
        wil kan je ook het \mintinline{tex}{\multirow}-commando van het package multirow
        uitproberen.
    \end{exercise}

    \begin{exercise}[\textbackslash autoref]
        Waarin verschilt het commando \mintinline{tex}{\autoref} (gedefinieerd door
        \mintinline{tex}{hyperref}) van het simpele \mintinline{tex}{\ref}?
    \end{exercise}

    \begin{exercise}[Babel]
        Voeg een table of contents toe, een \mintinline{tex}{\autoref} referentie
        naar een vergelijking, en een figuur. Kijk welke automatische termen
        babel ervoor geeft in verschillende talen.
    \end{exercise}

    \section{Uitbreiding}

    \begin{exercise}[tabularx]
        
    \end{exercise}

    \begin{exercise}[Referentie naar subfigure]
        Kan je een \mintinline{tex}{\label{...}} toevoegen aan de caption van een subfigure?
        Hoe ziet een referentie daaraan eruit?
    \end{exercise}

    \begin{exercise}[Subfigure nesting]
        Plaats een subfigure binnen een andere subfigure. Hoe ziet dit eruit?
        Hoe vreselijk voelt dat?
    \end{exercise}

    \begin{exercise}[\textbackslash hfill]
        Maak een figure met kleine subfigures erin. Wat gebeurt er als je
        \mintinline{tex}{\hfill} toevoegt tussen de subfigures?
    \end{exercise}

    \begin{exercise}[\textbackslash texorpdfstring]
        Wat doet het command \mintinline{tex}{\texorpdfstring{}{}} van het hyperref package?
        Zoek het op in de documentatie van de package.
    \end{exercise}

    \begin{exercise}[pageref]
        
    \end{exercise}

    \begin{exercise}
        Voeg een aantal alinea's van \href{https://lipsum.com}{lipsum.com} toe aan je bestand, en spreidt
        ze uit over meerdere pagina's:
        \begin{itemize}
            \item Pagina 1: Landscape A5-papier met marges $ 2\text{cm} $
            \item Pagina 2: Papier van dimensies $ 100\text{mm}\times 100\text{mm} $
            met marges $ 1\text{cm} $ en $ 2\text{cm} $ langs de onderkant
            \item Pagina 3: Portrait A6-papier met marges $ 0\text{cm} $
        \end{itemize}

        Hint: \url{https://tex.stackexchange.com/a/528245/242407}

        Opmerking: de pagina grootte van A6-papier is 105mm:148mm
    \end{exercise}
\end{document}