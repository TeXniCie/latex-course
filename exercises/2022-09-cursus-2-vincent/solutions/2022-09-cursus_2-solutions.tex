\documentclass[a4paper]{article}

\newif\ifishandout
\ishandoutfalse
%\ishandouttrue

\ifishandout
\documentclass[handout,aspectratio=32]{beamer}
\else
\documentclass[aspectratio=32]{beamer}
\fi

\usepackage[tabsize=4]{highlightlatex}

\setbeamertemplate{caption}[numbered]

\usecolortheme{rose}
%\useinnertheme[shadow]{rounded}
\useinnertheme{rounded}

\usetheme{Dresden}
\usecolortheme{dolphin}
\useoutertheme{miniframes}

\usepackage{subfiles}
\usepackage{amsmath,amssymb,amsthm,commath,mathtools}
\usepackage{esint}
\usepackage{enumerate}
\usepackage{subcaption}
\usepackage{graphicx}
\usepackage{xcolor}
\usepackage{adjustbox}
\usepackage{soul}
\usepackage{booktabs}
\usepackage{tabularx}
\usepackage{environ}
\usepackage[dutch]{babel}
\usepackage[utf8]{inputenc}
\usepackage{fancyvrb}
\usepackage{marvosym}
\usepackage{csquotes}
\usepackage[style=numeric]{biblatex}
\usepackage{textcomp}
%\usepackage{enumitem}
\usepackage{hyperref}
\usepackage{xkeyval}

\addbibresource{\subfix{assets/fakebib.bib}}

\DeclareMathOperator{\Image}{Image}

% Source: https://tex.stackexchange.com/questions/41683/why-is-it-that-coloring-in-soul-in-beamer-is-not-visible
\let\UL\ul
\makeatletter
\renewcommand\ul{
	\let\set@color\beamerorig@set@color
	\let\reset@color\beamerorig@reset@color
	\UL
}

\let\ST\st
\makeatletter
\def\st#1{
	\begingroup
	\let\set@color\beamerorig@set@color
	\let\reset@color\beamerorig@reset@color
	\def\SOUL@uleverysyllable{%
		\rlap{%
			%\color{red}
			\the\SOUL@syllable
			\SOUL@setkern\SOUL@charkern}%
		\SOUL@ulunderline{%
			\phantom{\the\SOUL@syllable}}%
	}%
	\ST{#1}%
	\endgroup
}
\makeatother
% https://tex.stackexchange.com/questions/71051/strikeout-in-different-color-appears-behind-letters-not-on-top-of-them

\setulcolor{red}
\setstcolor{red}

% Override if you want. Else you can delete it.
%\colorlet{curlyBrackets}{red!50!blue}
%\colorlet{squareBrackets}{blue!50!white}
%\colorlet{codeBackground}{gray!10!white}
%\colorlet{comment}{green!40!black}

\updatehighlight{
	name = default,
	color = {blue!90!black},
	add = {
		\knowncommand, \figref, \textcolor, \maketitle, \subsubsection,
		\textasciigrave, \textasciiacute, \tag, \middle, \mathbb, \abs,
		\mathcal, \middle, \dfrac, \subfile, \autoref, \eqref, \cites,
		\tableofcontents, \printbibliography, \fullcite, \parencite,
		\addbibresource, \DeclareLanguageMapping, \textcite, \intertext,
		\sum, \dif, \norm, \text, \dod, \dpd, \int, \partial,
		\DeclareMathOperator
	},
	name = structure,
	add = {
	},
}

\updatehighlight{
	name = greenDollar,
	style = {\itshape\color{green!70!black}},
	add = {
		% The dollar sign is provided an extra time just to
		% calm down TeXstudio's code highlighting.
		$, $
	},
	name = accentA,
	color = green!60!black,
	add = {
		\inAccA
	},
	%
	name = accentB,
	color = red!60!black,
	add = {
		\inAccB, \includegraphics
	},
	%
	name = accentC,
	color = orange!100!black,
	add = {
		\inAccC
	}
}

\lstset{tabsize=4}
\def\defaultgobble{8}

%\hllconfigure{
%	gobbletabs=3,
%}

\def\Zphantomconceal#1#2{%
	\only<#2->{\rlap{#1}}\phantom{#1}%
	%\only<#2->{#3}\unless\ifishandout\only<-#1>{\phantom{#3}}\fi
}

\def\phantomconceal#1#2{%
	\Zphantomconceal{#1}{#2}%
}

\newcommand\hideformula[2][2]{%
	%\hll|$| \only<2->{\hll|\\sqrt\{2\}|}\only<-1>{??} \hll|$|
	\hll|$| \phantomconceal{\hll|#2|}{#1} \hll|$|
}

\newcommand\hidelatex[2][2]{%
	\phantomconceal{\hll|#2|}{#1}
}%

\newcount\showcount

%\newcommand\showformula[2]{%
%	#1 & %
%	\expandafter\hideformula\expandafter[\the\showcount]{#2}%
%}
%
%\newcommand\showformula[2]{%
%	\global\showcount=\numexpr\showcount + 1\relax
%	\showformula*{#1}{#2}%
%}

\makeatletter

\def\showformula@i#1#2{%
	#1 & %
	\expandafter\hideformula\expandafter[\the\showcount]{#2}%
}

%\def\showformula{%
%	\@ifstar{%
%		\global\showcount=\numexpr\showcount + 1\relax
%		\showformula@i
%	}{%
%		\showformula@i
%	}%
%}

\def\showformula#1#2{
	#1 & \global\showcount=\numexpr\showcount + 1\relax
	\expandafter\hideformula\expandafter[\the\showcount]{#2}%
}

\def\showformulaa#1#2{
	#1 & %
	\expandafter\hideformula\expandafter[\the\showcount]{#2}%
}

\def\showlatex#1#2{
	#1 & \global\showcount=\numexpr\showcount + 1\relax
	\expandafter\hidelatex\expandafter[\the\showcount]{#2}%
}

\def\showlatexx#1#2{
	#1 & %
	\expandafter\hidelatex\expandafter[\the\showcount]{#2}%
}

\makeatother

\newlength{\naturalwidth}
\newlength{\minimumwidth}
\newbox\naturalsizebox
\newcommand{\atleastwidth}[2][2cm]{%
	\savebox\naturalsizebox{#2}%
	\settowidth\naturalwidth{#2}%
	\naturalwidth=\wd\naturalsizebox
	\minimumwidth=\dimexpr #1\relax
	\leavevmode%(\the\naturalwidth, \the\minimumwidth)%
	\ifdim\naturalwidth<\minimumwidth\relax
	\makebox[\minimumwidth][l]{\usebox{\naturalsizebox}}%
	\else
	\usebox{\naturalsizebox}%
	\fi
}

\newcommand{\atleastwidthr}[2][2cm]{%
	\savebox\naturalsizebox{#2}%
	\settowidth\naturalwidth{#2}%
	\naturalwidth=\wd\naturalsizebox
	\minimumwidth=\dimexpr #1\relax
	\leavevmode%(\the\naturalwidth, \the\minimumwidth)%
	\ifdim\naturalwidth<\minimumwidth\relax
	\makebox[\minimumwidth][r]{\usebox{\naturalsizebox}}%
	\else
	\usebox{\naturalsizebox}%
	\fi
}

\lstset{framexleftmargin=0.25em,xleftmargin=0.25em}

\NewEnviron{bluebox}{
	\begingroup
		\adjustbox{cfbox=blue!40!white 2pt 10pt,valign=t,bgcolor=blue!5!white}{%
			\begin{minipage}[t]{\dimexpr\linewidth-24pt\relax}
				\BODY
			\end{minipage}%
		}%
	\endgroup
}

\newcounter{maxrecentdisplay}
\setcounter{maxrecentdisplay}{27}

\newcounter{recentcount}
\setcounter{recentcount}{0}

\newcounter{recentskipremaining}

\def\vertlistsep{\hspace{2em}\textcolor{white!100!black}{\vrule width 0.5pt height 0.7\baselineskip\relax}\hspace{2em}}

\def\recentlist{}

%\newcommand{\addtorecentlist}[1]{%
%	\let\do\relax
%	\xdef\recentlist{\recentlist\do{#1}}%
%}

\newcommand{\addtorecentlist}[1]{%
	\bgroup
		\let\do\relax
		\expandafter\gdef\expandafter\recentlist\expandafter{\recentlist\do{#1}}%
		\addtocounter{recentcount}{1}%
	\egroup
	%
	%\xdef\recentlist{\recentlist\do{#1}}%
}

\newcommand{\clearrecentlist}{%
	\gdef\recentlist{}%
	\setcounter{recentcount}{0}%
}

\newif\ifisfirstrecentitem
\newcommand{\printrecentlist}{%
	\setcounter{recentskipremaining}{0}%
	\ifnum\value{recentcount}>\value{maxrecentdisplay}
		\setcounter{recentskipremaining}{\value{recentcount}-\value{maxrecentdisplay}}
	\fi
	%(\therecentskipremaining)
	%(\meaning\recentlist)
	\isfirstrecentitemtrue
	\def\do##1{%
		\ifnum\value{recentskipremaining}>0\relax
			\addtocounter{recentskipremaining}{-1}%
		\else		
			\unless\ifisfirstrecentitem
			\vertlistsep
			\fi
			\isfirstrecentitemfalse
			\textbf{##1}%
		\fi
	}%
	\recentlist
}

\newcommand{\recentpopfront}[1][1]{%
	\typeout{recentpopfront, before: \meaning\recentlist}
	\setcounter{recentskipremaining}{#1}%
	\let\origrecentlist\recentlist
	\clearrecentlist
	\def\do##1{%
		\ifnum\value{recentskipremaining}>0\relax
			\addtocounter{recentskipremaining}{-1}%
		\else		
			\addtorecentlist{##1}%
		\fi
	}%
	\origrecentlist
	\typeout{recentpopfront, after: \meaning\recentlist}
}

\newsavebox\printrecentbox
\savebox\printrecentbox{}
\newsavebox\scratchbox

% \AtBeginDocument{
% \setbox\scratchbox\printrecentbox
% }

\newcommand{\saveprintrecentbox}{%
	\bgroup
		\savebox\printrecentbox{\printrecentlist}%
		\global\setbox\printrecentbox\box\printrecentbox
	\egroup
	% \setbox\scratchbox\printrecentbox
	% \global\setbox\printrecentbox\scratchbox
	% \ifdim\wd\printrecentbox>0.9\textwidth
	% 	\savebox\printrecentbox{\adjustbox{right=0.9\textwidth}{\printrecentlist}}%
	% \else
	% 	\savebox\printrecentbox{\adjustbox{left=0.9\textwidth}{\printrecentlist}}%
	% \fi
}

\newcommand{\shrinkrecentbox}[1]{%
	{\loop
		%\clearrecentlist
		%\saveprintrecentbox
		%(SavedEmptyBox)
		%\iffalse


		\ifdim\wd\printrecentbox>\dimexpr #1\relax
		%
		\recentpopfront[1]%
		\saveprintrecentbox
	\repeat}%
}

% Based on miniframes code
\setbeamertemplate{headline}
{%
	\begin{beamercolorbox}[colsep=1.5pt]{upper separation line head}
	\end{beamercolorbox}
	\begin{beamercolorbox}{section in head/foot}
		\vskip2pt\insertnavigation{\paperwidth}\vskip2pt
	\end{beamercolorbox}%
	%
	\begin{beamercolorbox}[colsep=1.5pt]{middle separation line head}
	\end{beamercolorbox}
	\begin{beamercolorbox}[
		ht=2.5ex,
		dp=1.125ex,
		leftskip=.3cm,rightskip=.3cm plus1fil
		]{subsection in head/foot}
		\usebeamerfont{subsection in head/foot}%\insertsubsectionhead
		% \savebox\printrecentbox{\printrecentlist}%
		% \ifdim\wd\printrecentbox>0.9\textwidth
		% 	\adjustbox{right=0.9\textwidth}{\printrecentlist}%
		% \else
		% 	\adjustbox{left=0.9\textwidth}{\printrecentlist}%
		% \fi
		\saveprintrecentbox
		\ifdim\wd\printrecentbox>0.9\textwidth
			%(Shrinking box)
			%\PackageError{debug}{Width is \the\wd\printrecentbox}{}%
			\shrinkrecentbox{0.6\textwidth}%
		\else
			%(Not shrinking box)
		\fi
		\usebox\printrecentbox
		%\textbullet\ Hey
	\end{beamercolorbox}%
	%
	\begin{beamercolorbox}[colsep=1.5pt]{lower separation line head}
	\end{beamercolorbox}
}

\makeatletter

\NewEnviron{colC}[2][]{%
	\def\setpadd{}%
	\if\relax #1\relax
	\else
		%\def\setpadd{padding={0pt {\dimexpr ((#1)-\height)\relax} {0pt} {0pt}}}%
		\def\setpadd{%
			set depth={\dimexpr (#1)-\height\relax}%
		}
	\fi
	% \def\setparboxargs{}%
	% \if\relax #1\relax
	% \else
	% 	\def\setparboxargs{[t][\dimexpr #1\relax][]}%
	% \fi
	%
	\expandafter\adjustbox\expandafter{\setpadd,
		%margin=0pt,padding=0pt,
	%padding={0pt {\dimexpr (0.4\textheight-\height)/2\relax} {0pt} {\dimexpr (0.4\textheight-\height)/2\relax}},
		fbox=1pt 0pt 0pt,
		valign=M
	}%
	{%
		\parbox{\dimexpr #2-2pt\relax}{%
			\BODY
		}%
	}%
}

\NewEnviron{colT}[2][]{%
	\def\setpadd{}%
	\if\relax #1\relax
	\else
		\def\setpadd{%
			set depth={\dimexpr (#1)-\height\relax}%
		}%
	\fi
	%
	\expandafter\adjustbox\expandafter{\setpadd,
		fbox=1pt 0pt 0pt,
		valign=T
	}%
	{%
		\parbox{\dimexpr #2-2pt\relax}{%
			\BODY
		}%
	}%
}

\makeatother

\newlength\atleastlength


\newenvironment{noindentlist}{
	\begin{list}{\textbullet}{
		\leftmargin=0pt\relax
		\itemindent=0pt\relax
		\setlength{\itemsep}{2pt}
	}
}{
	\end{list}
}




\title{\vspace{-65pt} Oplossingen oefeningen \LaTeX-cursus Week 2}
\author{\TeX niCie\\{\small (Vincent Kuhlmann)}}
\date{4 oktober 2022}

\usepackage{minted}
\setminted[tex]{fontsize=\small, autogobble=true, linenos=false, frame=none}

\usemintedstyle{pastie}

\usepackage{wrapfig}
% \usepackage{cutwin}
\usepackage{booktabs}

\usepackage[english,latin,esperanto,french,icelandic,norsk,irish,dutch]{babel}

\addto\extrasbritish{%
    \def\exercisename{Exercise}%
    \def\exerciseautorefname{Exercise}%
}


\addto\extrasenglish{%
    \def\exercisename{Exercise}%
    \def\exerciseautorefname{Exercise}%
}


\addto\extrasdutch{%
    \def\exercisename{Oefening}%
    \def\exerciseautorefname{Oefening}%
}

\usepackage[bookmarksnumbered,colorlinks]{hyperref}

\theoremstyle{plain}
\newtheorem{theorem}{Stelling}

\theoremstyle{definition}
\newtheorem{definition}{Definitie}

\theoremstyle{remark}
\newtheorem{remark}{Opmerking}

\setcounter{secnumdepth}{0}

\newenvironment{solution}{\par\medskip\textbf{Oplossing:}}{\medskip}

\newenvironment{demobox}{
    \begin{adjustbox}{frame=1pt 10pt}%
        \begin{minipage}{\linewidth-22pt}
}{
        \end{minipage}
    \end{adjustbox}
}

\begin{document}
\maketitle

% \CheckBox[]{aaa}

\section{Deel 1: Document, referenties en `Theorem'}
Zorg dat je steeds minstens deze packages hebt in je preamble:
\begin{minted}{tex}
        \usepackage[a4paper,margin=2.54cm]{geometry}
        \usepackage{amsmath,amssymb,amsthm}
        \usepackage[bookmarksnumbered]{hyperref}
    \end{minted}
\bigskip

\begin{exercise}[Geometry]
    Maak een A6-document in landscape met voorbeeldtekst van \nolinkurl{lipsum.com}. Zet de
    horizontale marges op $ 2\text{ cm} $ en vertical marges op $ 3\text{ cm} $.

    Hint: De volgende opties van geometry kunnen van pas komen: left, right, top, bottom,
    vmargin, hmargin, landscape, a6paper.\\
    Documentatie over gebruik van geometry package: \url{https://ctan.org/pkg/geometry}


    \begin{solution}
        \begin{minted}{tex}
            \documentclass{article}
            \usepackage[a6paper,hmargin=2cm,vmargin=3cm]{geometry}
            ...
        
            \begin{document}
                Lorem ipsum dolor ...
            \end{document}
        \end{minted}
    \end{solution}
\end{exercise}

\begin{exercise}[Titels]
    Voeg een paar \mintinline{tex}{\section}'s toe aan je bestand, en een
    table of contents op een aparte pagina.

    Het \mintinline{tex}{\section} commando laat een optioneel argument toe.
    Voeg een section ermee toe, bijvoorbeeld
    \mintinline{tex}{\section[Intro]{Introductie}} en kijk wat er gebeurt in je table of
    contents.

    \begin{solution}
        Het optionele argument geeft een alternatieve titel aan die gebruikt wordt in de
        inhoudsopgave. Kan handig zijn als je heel lange titels zou nodig hebben, maar een afkorting
        voldoende is voor de inhoudsopgave.

        \begin{minted}{tex}
            \documentclass{article}
            ...

            \begin{document}
                \tableofcontents
                \newpage

                \section[intro]{Introductie}
                ...
                \section{...}
            \end{document}
        \end{minted}
    \end{solution}
\end{exercise}

\begin{exercise}[PDF TOC]
    Voeg \mintinline{tex}{\usepackage[bookmarksnumbered]{hyperref}} toe aan
    je preamble. Download je document als PDF en open het. Kijk of je de
    table of contents van je PDF-lezer kan vinden.
    Wat gebeurt er als je \mintinline{tex}{bookmarksopen} toevoegt als option voor hyperref?

    \begin{solution}
        Bookmarksopen zorgt dat standaard alles opengeklapt is, dus subsections zie je al meteen
        in je PDF lezer zonder de sections te hoeven openklappen. Is een afweging van hoe fijn of
        storend die diepe niveau's altijd tonen is.
    \end{solution}
\end{exercise}

\begin{exercise}[URL's]
    Voeg de volgende link toe aan je bestand:\\
    \nolinkurl{https://en.wikipedia.org/wiki/Electromagnetic_tensor}
    \begin{enumerate}[label=\alph*)]
        \item Wat gebeurt er als je de link direct in je code plakt?
              Kan je die foutmelding fixen?
        \item Plak nu dezelfde link in het argument van het \mintinline{tex}{\url}-commando
              van hyperref: \mintinline{tex}{\url{...}}. Heb je dezelfde fix nog nodig?
        \item Wat gebeurt er als je de \texttt{https://} weglaat?
    \end{enumerate}

    \begin{solution}
        \begin{enumerate}[label=\alph*)]
            \item De underscore is een speciaal karakter omdat het subscript in wiskundemodus aangeeft.
            Helaas moet je dus zelfs in tekstmodus een underscore forceren met \mintinline{tex}{\_} (backslash ervoor).
            \item Het \mintinline{tex}{\url}-commando van hyperref maakt de link aanklikbaar, maar
            zorgt er ook voor dat het underscorekarakter tijdelijk niet meer ge\"interpreteerd wordt
            als subscript. Nadat het argument is ingelezen, wordt alles weer teruggezet.
            De fix is dus niet meer nodig, maar het kan ook geen kwaad.
            \item Als je de \texttt{https://} weglaat formatteert hij de link nog steeds met een
            teletype font, maar de link is niet meer aanklikbaar. Als je wil dat de link aanklikbaar
            blijft zonder dat je de \texttt{https://} nodig hebt kan je het volgende gebruiken:
            \begin{minted}{tex}
                \href{https://en.wikipedia.org/wiki/Electromagnetic_tensor}{%
                    \nolinkurl{en.wikipedia.org/wiki/Electromagnetic_tensor}
                }
            \end{minted}
        \end{enumerate}
    \end{solution}
\end{exercise}

\begin{exercise}[\textbackslash eqref]
    De amsmath package definieert het commando \mintinline{tex}{\eqref{...}}.
    Voeg een genummerde vergelijking toe aan je document, met een label, en
    kijk wat het verschil is tussen \mintinline{tex}{\ref{...}}
    en \mintinline{tex}{\eqref{...}}.
    \begin{solution}
        De \mintinline{tex}{\eqref{...}} print ook het nummer, maar met al haakjes errond.

        \begin{minted}{tex}
            \begin{align}\label{eq:geweldig}
                x = y + \sqrt{17}
            \end{align}
            Zie \ref{eq:geweldig} of \eqref{eq:geweldig}. Maar (\ref{eq:geweldig}) kan ook.
        \end{minted}
        \begin{demobox}
            \begin{align}\label{eq:geweldig}
                x = y + \sqrt{17}
            \end{align}
            Zie \ref{eq:geweldig} of \eqref{eq:geweldig}. Maar (\ref{eq:geweldig}) kan ook.
        \end{demobox}
    \end{solution}
\end{exercise}

\begin{exercise}[Labels]
    Wat gebeurt er als je aan een niet-bestaande label refereert?

    \begin{solution}
        In je output krijg je \textbf{??} te staan, en je krijgt een compiler warning.

        % \begin{minted}{tex}
        %     Zie: \ref{huh}
        % \end{minted}
        % \begin{demobox}
        %     Zie: \ref{huh}
        % \end{demobox}
    \end{solution}
\end{exercise}

\newpage
\begin{exercise}[Stelling met bewijs]
    Voeg een theorem met proof toe in je bestand voor je favoriete stelling of bewijs.

    \begin{solution}
        \begin{minted}{tex}
            \documentclass{article}

            \usepackage{amsmath,amssymb,amsthm}
            \usepackage{commath,mathtools}
            \usepackage[dutch]{babel}
            \newtheorem{theorem}{Stelling}

            \begin{document}
            \begin{theorem}
                De Gaussische integraal $ A := \int_0^{\infty}e^{-x^2}\dif x $ heeft waarde $ \sqrt{\pi}/2 $.
                \begin{proof}
                    Merk op $ A\geq 0 $, dus $ A=\sqrt{A}^2 $. We vinden dan
                    \begin{align*}
                        A^2 &= \left(\int_0^{\infty}e^{-r^2}\dif r\right)^2\\
                        &= \int_0^{\infty}\int_0^{\infty}e^{-x^2-y^2}\dif x\dif y\\
                        &= \frac{1}{4}\int_{-\infty}^{\infty}\int_{-\infty}^{\infty}e^{-x^2-y^2}\dif x\dif y.
                    \intertext{en door omzetting naar poolco\"ordinaten vinden we}
                        &= \frac{1}{4}\int_0^{2\pi}\int_0^{\infty}e^{-r^2}r\dif r\dif\theta\\
                        &= \frac{2\pi}{4}\int_0^{\infty}e^{-r^2}\frac{r}{2r}\dif\,(r^2)\\
                        &= \frac{\pi}{4}(e^0 - e^{-\infty}) = \frac{\pi}{4}.
                    \end{align*}
                    We vinden dus $ A = \sqrt{\pi/4} = \sqrt{\pi}/2 $.
                \end{proof}
            \end{theorem}
            \end{document}
        \end{minted}
        \setlength\uprightIndent{0pt}

        \begin{demobox}
            \begin{theorem}
                De Gaussische integraal $ A := \int_0^{\infty}e^{-x^2}\dif x $ heeft waarde $ \sqrt{\pi}/2 $.
                \begin{proof}
                    % \begin{align*}
                    %     \int_0^{\pi}e^{-r^2}\dif r &= \frac{1}{\pi}\int_0^{\pi}\int_0^{\pi}e^{-r^2\cos^2(\theta)-r^2\sin^2(\theta)}\dif r\dif\theta\\
                    %     &= \frac{1}{\pi}\iint e^{-x^2-y^2}\frac{1}{x^2+y^2}\dif x\dif y
                    %     % \int_0^{\infty}\int_0^{\infty}e^{-(x^2+y^2)}\dif x\dif y\\
                    %     % &= \int_0^{\infty}e^{}
                    % \end{align*}\
                    Merk op $ A>0 $, dus $ A=\sqrt{A^2} $. We kunnen dus berekenen
                    \begin{align*}
                        A^2 &= \left(\int_0^{\infty}e^{-r^2}\dif r\right)^2\\
                        &= \int_0^{\infty}\int_0^{\infty}e^{-x^2-y^2}\dif x\dif y\\
                        &= \frac{1}{4}\int_{-\infty}^{\infty}\int_{-\infty}^{\infty}e^{-x^2-y^2}\dif x\dif y,
                    \intertext{en door omzetting naar poolco\"ordinaten vinden we}
                        &= \frac{1}{4}\int_0^{2\pi}\int_0^{\infty}e^{-r^2}r\dif r\dif\theta\\
                        &= \frac{2\pi}{4}\int_0^{\infty}e^{-r^2}\frac{r}{2r}\dif\,(r^2)\\
                        &= \frac{\pi}{4}(e^0 - e^{-\infty}) = \frac{\pi}{4}.
                    \end{align*}
                    We vinden hiermee $ A = \sqrt{\pi/4} = \sqrt{\pi}/2 $.
                \end{proof}
            \end{theorem}
        \end{demobox}        
    \end{solution}
\end{exercise}

\begin{exercise}[Definitie]
    Voeg een `Definitie' toe aan je bestand, en refereer eraan in je bestand.

    \begin{solution}
        \begin{minted}{tex}
            \documentclass{article}
            \usepackage{amsthm}
            \usepackage{hyperref}
            
            \theoremstyle{definition}
            \newtheorem{definition}{Definitie}
            \begin{document}
                Als je echt wil weten wat een \LaTeX{} commando is, kijk dan naar \autoref{def:latexCommand}.
    
                \begin{definition}\label{def:latexCommand}
                    Een commando in \LaTeX{} is een karaktersequentie beginnende met een escape character
                    (catcode 0, standaard de backslash) gevolgd door oftewel een enkel karakter
                    oftewel door meerdere letters met catcode 11 (letter).\footnote{Deze
                    definitie heeft vast meerdere problemen.}
                \end{definition}
            \end{document}
        \end{minted}
    
        \begin{demobox}
            \setlength\uprightIndent{0pt}
            Als je echt wil weten wat een \LaTeX{} commando is, kijk dan naar \autoref{def:latexCommand}.
    
            \medskip
            \begin{definition}\label{def:latexCommand}
                Een commando in \LaTeX{} is een karaktersequentie beginnende met een escape character
                (catcode 0, standaard de backslash) gevolgd door oftewel een enkel karakter
                oftewel door meerdere letters met catcode 11 (letter).\footnote{Deze definitie heeft vast meerdere problemen.}
            \end{definition}
        \end{demobox}
    \end{solution}
\end{exercise}

\begin{exercise}[\textbackslash theoremstyle]\label{ex:theoremStyle}
Cre\"eer een nieuw bestand met de template van
\href{https://vkuhlmann.com/latex/example}{\nolinkurl{vkuhlmann.com/latex/example}}
(zet de 'Include Theorem, Lemma etc.' aan). Wat is het verschil in stijl tussen
\mintinline{tex}{\begin{theorem}},
\mintinline{tex}{\begin{definition}} en \mintinline{tex}{\begin{remark}}?
Probeer deze stijlen nu te veranderen door \mintinline{tex}{\theoremstyle{...}} commando's
toe te voegen, te verplaatsen en/of te verwijderen.

\begin{solution}
    De stijl van \mintinline{tex}{\theoremstyle{plain}} en \mintinline{tex}{\theoremstyle{definition}}
    hebben allebei de naam en nummer in het vet, maar bij de plain is de inhoud cursief en bij de
    definition rechtop. Bij de remark-style is de naam cursief (en niet vetgedrukt), en de
    inhoud rechtop.

    \begin{minted}{tex}
        \documentclass{article}
        \usepackage{amsthm}
        \theoremstyle{plain}
        \newtheorem{theorem}{Stelling}

        \theoremstyle{definition}
        \newtheorem{definition}{Definitie}

        \theoremstyle{remark}
        \newtheorem{remark}{Opmerking}

        \begin{document}
            \begin{definition}Definitie-inhoud
            \end{definition}
            \begin{theorem}Stelling-inhoud
                \begin{proof}
                    Met bewijs
                \end{proof}
            \end{theorem}
            \begin{remark}
                Opmerking-inhoud
            \end{remark}
        \end{document}
    \end{minted}
    \begin{demobox}
        \setlength\uprightIndent{0pt}
        \begin{definition}Definitie-inhoud
        \end{definition}
        \begin{theorem}Stelling-inhoud
            \begin{proof}
                Met bewijs
            \end{proof}
        \end{theorem}
        \begin{remark}
            Opmerking-inhoud
        \end{remark}
    \end{demobox}
\end{solution}
\end{exercise}

\begin{exercise}[Theorem nummering]%[Theorem numberwithin]
    %Probeer uit te vinden welk effect elk van de volgende codewijzigingen hebben:
    Welk effect heeft elk van de volgende codewijzigingen? (zelfde basisbestand als bij \autoref{ex:theoremStyle})
    \begin{enumerate}[label=\alph*)]
        \item \begin{minted}{tex}
            \newtheorem{theorem}{Theorem}[section] --> \newtheorem{theorem}{Theorem}
        \end{minted}
        \item \begin{minted}{tex}
            \newtheorem{lemma}[theorem]{Lemma} --> \newtheorem{lemma}{Lemma}
        \end{minted}
    \end{enumerate}

    \begin{solution}
        \begin{enumerate}[label=\alph*)]
            \item De nummeringen veranderen, bv. Theorem 1.3 wordt nu Theorem 3 en Theorem 2.1
            wordt Theorem 4, de Theorems worden niet meer per section genummerd.
            \item Lemma deelt zijn teller niet met Theorem, waardoor je bijvoorbeeld zowel
            Theorem 1 als Lemma 1 kan hebben.
        \end{enumerate}
    \end{solution}
\end{exercise}

\section{Deel 2: Figuren, matrices en tabellen}

Zorg dat je steeds minstens deze packages hebt in je preamble:
\begin{minted}{tex}
        \usepackage[a4paper,margin=2.54cm]{geometry}
        \usepackage{amsmath,amssymb,amsthm}
        \usepackage{graphicx}
        \usepackage{subcaption}
        \usepackage{booktabs}
        \usepackage[bookmarksnumbered]{hyperref}
    \end{minted}
\bigskip

\begin{exercise}[Figure]
    Is het mogelijk een figure environment te maken zonder \mintinline{tex}{\includegraphics}?
    Kan je in plaats ervan tekst, een inline formule of een tabel hebben?

    \begin{solution}
        Ja, maar uiteraard niet de bedoeling.
    \end{solution}
\end{exercise}

\begin{exercise}[Figuurplaatsing]
    Cre\"eer een scenario waarbij \LaTeX{} je figuurplaatsingsadvies niet opvolgt.

    \begin{solution}
        \begin{minted}{tex}
            \documentclass{article}

            \usepackage[a5paper,vmargin=4cm]{geometry}
            \usepackage{graphicx}
            \usepackage{lipsum}

            \begin{document}
                \lipsum[1]    

                \begin{figure}[h]
                    \includegraphics[width=\textwidth]{example-image-a}
                    \caption{Hoi}
                \end{figure}
                
                \lipsum[2]
            \end{document}
        \end{minted}
    \end{solution}
\end{exercise}

\begin{exercise}[Subfigure]
    Maak een figure met veel subfigures erin, gebaseerd op de code in de slides. Kijk wat de verschillende parameters
    doen. Wat doet de \mintinline{tex}{0.45\textwidth}? Wat doet de \mintinline{tex}{[b]}?%
    \footnote{Hint: Vervang de [b] door [t] of [c] en geef de afbeeldingen in de subfigures
        ongelijke hoogtes.}

    \begin{solution}
        De \mintinline{tex}{[b]} specifieert dat de subfiles verticaal gealigneerd worden op de
        onderkant van elke subfigure. Als je het op c zet wordt alles verticaal gecentreerd. Je
        kan de waarden ook laten verschillen per subfigure, maar het gedrag daarvan vereist
        meer begrip van \LaTeX{} dan het idee was achter deze oefening. De \mintinline{tex}{0.45\textwidth}
        geeft de breedte aan van elke subfigure.

        \begin{minted}{tex}
            \documentclass{article}
            \usepackage{subcaption}
            \usepackage{float}
            \usepackage{graphicx}

            \begin{document}
                \begin{figure}[H]
                    \centering
                    \begin{subfigure}[c]{0.45\textwidth}
                        \centering
                        \includegraphics[height=2cm,width=\linewidth]{example-image-a}
                        \caption{...}
                    \end{subfigure}
                    \begin{subfigure}[c]{0.2\textwidth}
                        \centering
                        \includegraphics[height=1cm,width=\linewidth]{example-image-b}
                        \caption{...}
                    \end{subfigure}
                    \begin{subfigure}[b]{0.2\textwidth}
                        \centering
                        \includegraphics[height=1cm,width=\linewidth]{example-image-c}
                        \caption{...}
                    \end{subfigure}
                    \begin{subfigure}[b]{0.2\textwidth}
                        \centering
                        \includegraphics[height=1cm,width=\linewidth]{example-image-c}
                        \caption{...}
                    \end{subfigure}
                    \caption{...}
                \end{figure}
            \end{document}
        \end{minted}

        \begin{demobox}
            \begin{figure}[H]
                \centering
                \begin{subfigure}[c]{0.45\textwidth}
                    \centering
                    \includegraphics[height=2cm,width=\linewidth]{example-image-a}
                    \caption{...}
                \end{subfigure}
                \begin{subfigure}[c]{0.2\textwidth}
                    \centering
                    \includegraphics[height=1cm,width=\linewidth]{example-image-b}
                    \caption{...}
                \end{subfigure}
                \begin{subfigure}[b]{0.2\textwidth}
                    \centering
                    \includegraphics[height=1cm,width=\linewidth]{example-image-c}
                    \caption{...}
                \end{subfigure}
                \begin{subfigure}[b]{0.2\textwidth}
                    \centering
                    \includegraphics[height=1cm,width=\linewidth]{example-image-c}
                    \caption{...}
                \end{subfigure}
                \caption{...}
            \end{figure}
        \end{demobox}
    \end{solution}
\end{exercise}

\begin{exercise}[Matrix]
    Maak een matrix met een verticale streep langs beide kanten i.p.v. haakjes, zoals de
    notatie voor determinant van een matrix. Kan je vinden welke environment (eindigend op
    matrix) dit al standaard doet?

    \begin{solution}
        \begin{minted}{tex}
            \begin{align*}
                \left|\begin{matrix}
                    0 & 1\\2&3
                \end{matrix}\right|
                = \begin{vmatrix}
                    0 & 1\\2&3
                \end{vmatrix}
            \end{align*}
        \end{minted}
        
        \begin{demobox}
            \begin{align*}
                \left|\begin{matrix}
                    0 & 1\\2&3
                \end{matrix}\right|
                = \begin{vmatrix}
                    0 & 1\\2&3
                \end{vmatrix}
            \end{align*}
        \end{demobox}
    \end{solution}
\end{exercise}

% \begin{tabular}{p{0.4\textwidth}l}
%     \begin{minipage}{\linewidth}
\begin{exercise}[Stelsel in matrix]
    % \begin{cutout}{2}{20pt}{\dimexpr\linewidth-2.5cm\relax}{6}
    %     AAA
    %   \end{cutout}
Stelsels lineaire vergelijkingen kunnen opgelost worden door ze te schrijven als een matrix en Gauss
eliminatie toe te passen.
% \begin{cutout}{3}{50pt}{\dimexpr\linewidth-50pt\relax}{6}
%     AAA
% \end{cutout}
%Een vergelijking $ 2x+y-z = 8 $ wordt dan de rij $ (2, 1, -1, 8) $.
%Dat het laatste getal geen eigenlijke variabele is, maken we in de wiskunde soms duidelijk door
%de extra kolom af te bakenen met een verticale streep.
Repliceer dit typische stelselmatrix:

\begin{tabularx}{\textwidth}{Xp{0.7\textwidth}}
\adjustbox{valign=t}{\small$\displaystyle
    \left(\begin{array}{rrr|r}
        2 & 1 & -1 & 8\\
        -3 & -1 & 2 & -11\\
        -2 & 1 & 2 & -3
    \end{array}\right)
$}&
\parbox[t]{0.65\textwidth}{\small De eerste rij komt overeen met de vergelijking $ 2x+y-z=8 $.\\
Getallenvoorbeeld van:\\\url{https://en.wikipedia.org/wiki/Gaussian_elimination}}
\end{tabularx}
% \end{exercise}
% \end{minipage}
% &
% {\small$\displaystyle
%     \left(\begin{array}{rrr|r}
%         2 & 1 & -1 & 8\\
%         -3 & -1 & 2 & 11\\
%         -2 & 1 & 2 & -3
%     \end{array}\right)
% $}

\begin{solution}
    \begin{minted}{tex}
        \begin{align*}
            \left(\begin{array}{rrr|r}
                2 & 1 & -1 & 8\\
                -3 & -1 & 2 & -11\\
                -2 & 1 & 2 & -3
            \end{array}\right)
        \end{align*}
    \end{minted}

    \begin{demobox}
        \begin{align*}
            \left(\begin{array}{rrr|r}
                2 & 1 & -1 & 8\\
                -3 & -1 & 2 & -11\\
                -2 & 1 & 2 & -3
            \end{array}\right)
        \end{align*}
    \end{demobox}
\end{solution}

\end{exercise}

\begin{exercise}[Align]
    Hoe gedraagt het align-environment zich als je meer dan twee `kolommen' hebt?

    \begin{solution}
        Het aligneert elk paar kolommen, en maakt een horizontale sprong tussen twee paren.

        \begin{minted}{tex}
            \begin{align*}
                0123&abcd&789&efgh\\
                1&2&3&4
            \end{align*}
        \end{minted}

        \begin{demobox}
            \begin{align*}
                0123&abcd&789&efgh\\
                1&2&3^4
            \end{align*}
        \end{demobox}
    \end{solution}
\end{exercise}

\begin{exercise}[Wiskunde in tabellen]
    Maak een simpele tabel met wat woorden, nummers en wiskundige symbolen erin (bijvoorbeeld $ \sqrt{2} $).

    \begin{solution}
        \begin{minted}{tex}
            \begin{tabular}{ll|c}
                Hoi & doei & $ \sqrt{2} $\\
                16 & 8 & $ \infty $\\
                \hline
            \end{tabular}
        \end{minted}

        \begin{demobox}
            \begin{tabular}{ll|c}
                Hoi&doei&$ \sqrt{2} $\\
                16&8&$ \infty $\\
                \hline
            \end{tabular}
        \end{demobox}
    \end{solution}
\end{exercise}

\begin{exercise}[Kolomscheidingen]
    Wat gebeurt er als een regel te veel kolommen heeft? En wat als het te
    weinig kolommen heeft?

    \begin{solution}
        Teveel: error `Extra alignment tab ...'. Te weinig: rest van de rij leeg.

        \begin{minted}{tex}
            \begin{tabular}{ll|c}
                Hoi&doei&$ \sqrt{2} $\\
                16&8\\%& $ \infty $&9&4\\
                \hline
            \end{tabular}
        \end{minted}

        \begin{demobox}
            \begin{tabular}{ll|c}
                Hoi&doei&$ \sqrt{2} $\\
                16&8\\%& $ \infty $&9&4\\
                \hline
            \end{tabular}
        \end{demobox}
    \end{solution}
\end{exercise}

\begin{exercise}[Alignering]
    Zoek op wat de mogelijke aligneringen zijn voor een kolom in een tabular
    en probeer ze uit.

    \begin{solution}
        \url{https://en.wikibooks.org/wiki/LaTeX/Tables#The_tabular_environment}

        \begin{minted}{tex}
            \begin{tabular}{p{2cm}||r|b{2.5cm}}
                Hoi hoe gaat het?&1&Goed, en met jou?\\
                Ook goed.&1234&Doei.
            \end{tabular}
        \end{minted}

        \begin{demobox}
            \begin{tabular}{p{2cm}||r|b{2.5cm}}
                Hoi hoe gaat het?&1&Goed, en met jou?\\
                Ook goed.&1234&Doei.
            \end{tabular}
        \end{demobox}
    \end{solution}
\end{exercise}

\begin{exercise}[Booktabs]
    Maak een tabel waarbij je \mintinline{tex}{\toprule}, \mintinline{tex}{\midrule}
    en \mintinline{tex}{\bottom} van booktabs gebruikt (zie slides), om
    een goed uitziende tabel te krijgen. Je kan \mintinline{tex}{\cmidrule} gebruiken om een
    gedeeltelijke horizontale lijn te krijgen.

    \begin{solution}
        Ik heb de slides gemaakt, dus ik heb dit al gedaan.
    \end{solution}
\end{exercise}

\begin{exercise}[Excellent]
    Gebruik \mintinline{tex}{\multicolumn} om op een rij twee kolommen samen te voegen
    (zoek op hoe het commando werkt, of deduceer het van de slides). Als je
    wil kan je ook het \mintinline{tex}{\multirow}-commando van het package multirow
    uitproberen.
    
    \begin{solution}
        Zie voorbeeld in de slides.
    \end{solution}
\end{exercise}

\begin{exercise}[\textbackslash autoref]
    Waarin verschilt het commando \mintinline{tex}{\autoref} (gedefinieerd door
    \mintinline{tex}{hyperref}) van het simpele \mintinline{tex}{\ref}?

    \begin{solution}
        Autoref zet ook de naam van het type referentie erbij. Voor vergelijkigen
        geeft dit dus een goed alternatief voor \mintinline{tex}{\eqref}, maar
        ik gebruik autoref vooral voor afbeeldingen, en eqref voor vergelijkingen.

        \begin{minted}{tex}
            Kijk eens naar \autoref{iets}.
            \begin{align}
                x &= 4\label{iets}
            \end{align}
        \end{minted}
        \begin{demobox}
            Kijk eens naar \autoref{iets}.
            \begin{align}
                x &= 4\label{iets}
            \end{align}
        \end{demobox}
    \end{solution}
\end{exercise}

\begin{exercise}[Babel]
    Voeg een table of contents toe, een \mintinline{tex}{\autoref} referentie
    naar een vergelijking, en een figuur. Kijk welke automatische termen
    babel ervoor geeft in verschillende talen.

    \begin{solution}
        \begin{minted}{tex}
            \documentclass{article}
    
            \usepackage{amsmath,amssymb}
            \usepackage{graphicx}
            \usepackage[dutch]{babel}
            %\usepackage[latin]{babel}
            %\usepackage[norwegian]{babel}
            %\usepackage[esperanto]{babel}
            %\usepackage[french]{babel}
            %\usepackage[icelandic]{babel}
            \usepackage{hyperref}

            \begin{document}
                \tableofcontents
                
                \section{AA}
                De trigonometrische identeit zie je in \autoref{eq:trigEq}.
                \begin{align}\label{eq:trigEq}
                    \cos^2(\theta) + \sin^2(\theta) = 1
                \end{align}
                \begin{figure}
                    \centering
                    \includegraphics[width=1cm]{example-image-a}
                    \caption{BB}
                \end{figure}
            \end{document}
        \end{minted}
    
        Als je een overzicht wil maken:
    
        \begin{minted}{tex}
            \documentclass{article}
            \usepackage[english,latin,esperanto,french,icelandic,norsk,irish,dutch]{babel}
            \usepackage{etoolbox}
            \usepackage{booktabs}
            \usepackage{hyperref}
    
            \begin{document}
                \def\nextitem{&}
                \def\do#1{
                    #1\nextitem
                    \begin{otherlanguage}{#1}
                        \contentsname
                    \end{otherlanguage}\nextitem
                    \begin{otherlanguage}{#1}
                        \figurename
                    \end{otherlanguage}\nextitem
                    \begin{otherlanguage}{#1}
                        \equationautorefname
                    \end{otherlanguage}\\
                }
                \begin{tabular}{l|lll}
                    \toprule
                    Taal&Inhoudsopgave&Figuur&Vergelijking\\
                    \midrule
                    \docsvlist{dutch,english,latin,esperanto,french,icelandic,norsk,irish}
                    \bottomrule
                \end{tabular}
            \end{document}
        \end{minted}
        \begin{demobox}
            \def\nextitem{&}% http://tex.stackexchange.com/a/89187/5764
            \def\do#1{
                #1\nextitem
                \begin{otherlanguage}{#1}
                    \contentsname
                \end{otherlanguage}\nextitem
                \begin{otherlanguage}{#1}
                    \figurename
                \end{otherlanguage}\nextitem
                \begin{otherlanguage}{#1}
                    \equationautorefname
                \end{otherlanguage}\\
            }
            \begin{tabular}{l|lll}
                \toprule
                Taal&Inhoudsopgave&Figuur&Vergelijking\\
                \midrule
                \docsvlist{dutch,english,latin,esperanto,french,icelandic,norsk,irish}
                \bottomrule
            \end{tabular}
        \end{demobox}
    \end{solution}
\end{exercise}

\section{Extra oefeningen} %\url{https://vkuhlmann.com/latex/exercises/2022-09-cursus/Week2_Vincent/uitbreiding}

Ga naar \href{https://vkuhlmann.com/go/d98d48}{\nolinkurl{vkuhlmann.com/go/d98d48}}
voor een extra set uitdagende oefeningen :)

% \section{Uitbreiding}

% \begin{exercise}[tabularx]
%     In de slides van tabellen is er een code voorbeeld die tabularx gebruikt. Wat zijn de verschillen
%     met de slide ervoor? Wat kan tabularx dus doen?
% \end{exercise}

% \begin{exercise}[Referentie naar subfigure]
%     Kan je een \mintinline{tex}{\label{...}} toevoegen aan de caption van een subfigure?
%     Hoe ziet een referentie daaraan eruit?
% \end{exercise}

% % \begin{exercise}[Subfigure nesting]
% %     Plaats een subfigure binnen een andere subfigure. Hoe ziet dit eruit?
% %     Hoe vreselijk voelt dat?
% % \end{exercise}

% \begin{exercise}[\textbackslash hfill]
%     Maak een figure met kleine subfigures erin. Wat gebeurt er als je
%     \mintinline{tex}{\hfill} toevoegt tussen de subfigures?
% \end{exercise}

% \begin{exercise}[\textbackslash texorpdfstring]
%     Wat doet het command \mintinline{tex}{\texorpdfstring{}{}} van het hyperref package?
%     Zoek het op in de documentatie van de package.
% \end{exercise}

% \begin{exercise}[pageref]

% \end{exercise}

% \begin{exercise}
%     Voeg een aantal alinea's van \href{https://lipsum.com}{lipsum.com} toe aan je bestand, en spreidt
%     ze uit over meerdere pagina's:
%     \begin{itemize}
%         \item Pagina 1: Landscape A5-papier met marges $ 2\text{cm} $
%         \item Pagina 2: Papier van dimensies $ 100\text{mm}\times 100\text{mm} $
%               met marges $ 1\text{cm} $ en $ 2\text{cm} $ langs de onderkant
%         \item Pagina 3: Portrait A6-papier met marges $ 0\text{cm} $
%     \end{itemize}

%     Hint: \url{https://tex.stackexchange.com/a/528245/242407}

%     Opmerking: de pagina grootte van A6-papier is 105mm:148mm
% \end{exercise}
\end{document}