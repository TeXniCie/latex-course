\documentclass{article}
\usepackage[a4paper,margin=2.54cm]{geometry}

\usepackage{amsmath,amssymb,amsthm}
\usepackage{commath,mathtools}
\usepackage{parskip}
\usepackage{graphicx}
\usepackage{xcolor}
\usepackage{subcaption}
\usepackage{fancyhdr}
\usepackage{enumerate}
\usepackage{enumitem}
\usepackage[dutch]{babel}
\usepackage{adjustbox}

\usepackage{minted}
\setminted[tex]{fontsize=\small, autogobble=true, linenos=false, frame=none}

\usemintedstyle{pastie}

\newenvironment{solution}{\par\medskip\textbf{Oplossing:}}{\medskip}

\newenvironment{demobox}{
    \begin{adjustbox}{frame=1pt 10pt}%
        \begin{minipage}{\linewidth-22pt}
}{
        \end{minipage}
    \end{adjustbox}
}

\usepackage[bookmarksnumbered]{hyperref}

\def\nowOutput#1{
    \qquad (#1)
}

% \makeatletter

% \def\codeAndOutput{%
%     \aftergroup\nowOutput
%     %\mintinline{tex}\expandafter\bgroup\@gobble CC\egroup
%     \expandafter\let\expandafter\openbrace\iftrue{\else}\fi
%     \expandafter\let\expandafter\closebrace\iffalse{\else}\fi
%     \expandafter\edef\expandafter\openbrace\expandafter{\openbrace}%
    
%     \PackageWarning{debug}{\meaning\openbrace}
%     \PackageWarning{debug}{\meaning\closebrace}
%     \def\pastemintinline{%
%         \mintinline{tex}%
%     }
%     \def\continue{
%         \PackageWarning{debug}{Opening char: \meaning\openingchar}
%         \expandafter\pastemintinline\aftergroup\nowOutput
%     }
%     %\expandafter\expandafter\expandafter\pastemintinline\expandafter\openbrace\@gobble
%     \afterassignment\continue
%     \let\openingchar=
%     %\expandafter\pastemintinline{aaa \sqrt{2}}
%     %\expandafter\pastemintinline\begingroup Hi\endgroup
%     %\mintinline{tex}\openbrace CC\closebrace
%     %\mintinline{tex}{CC}
% }
% \makeatother

\title{Lijst van geavanceerde wiskundige constructies in \LaTeX}
\date{4 oktober 2022}
\author{Vincent Kuhlmann}

\begin{document}
    \maketitle

    Deze lijst is niet heel georganiseerd zoals je ziet. De meeste constructies leer je gaandeweg
    in \LaTeX{} (en het is niet realistisch om een overzichtelijke lijst te maken van alles wat mogelijk is), maar voor degenen die bang zijn dat ze iets missen, neem deze lijst door,
    en stuur ons een mailtje bij vragen.

    \begin{enumerate}
        \item \mintinline{tex}{$ A\rightarrow B $}\qquad $ A\rightarrow B $
        \item \mintinline{tex}{$ A\to B $}\qquad $ A\to B $
        \item \mintinline{tex}{$ f: \mathbb{R}\to \mathbb{R}, x\mapsto x^2 $}\qquad $ f: \mathbb{R}\to \mathbb{R}, x\mapsto x^2 $
        \item \mintinline{tex}{$ \pi_1: \mathbb{R}^2\to \mathbb{R}, (x_1, x_2)\mapsto x_1 $}\qquad $ \pi_1: \mathbb{R}^2\to \mathbb{R}, (x_1, x_2)\mapsto x_1 $
        \item \mintinline{tex}{$ A\xrightarrow[\alpha\to\beta]{\text{erboven}}B $}\qquad $ A\xrightarrow[\alpha\to\beta]{\text{erboven}}B $
        \item \mintinline{tex}{$ B\xleftarrow{\text{erboven}} A $}\qquad $ B\xleftarrow{\text{erboven}} A $
        \item \begin{minted}{tex}
            De sommatie over $ x_i $ schrijven we als
            \begin{align*}
                \sum_{i=0}^{\infty} x_i
            \end{align*}
        \end{minted}
        \begin{demobox}
            De sommatie over $ x_i $ schrijven we als
            \begin{align*}
                \sum_{i=0}^{\infty} x_i.
            \end{align*}
        \end{demobox}
        \item \begin{minted}{tex}
            Het product van $ x_i $ schrijven we als
            \begin{align*}
                \prod_{i=0}^{\infty} x_i
            \end{align*}
        \end{minted}
        \begin{demobox}
            Het product van $ x_i $ schrijven we als
            \begin{align*}
                \prod_{i=0}^{\infty} x_i.
            \end{align*}
        \end{demobox}
        \item \begin{minted}{tex}
            De vereniging van een verzameling sets $ \mathcal{V} $ schrijven we als
            \begin{align*}
                \bigcup_{V\in\mathcal{V}} V
            \end{align*}
        \end{minted}
        \begin{demobox}
            De vereniging van een verzameling sets $ \mathcal{V} $ schrijven we als
            \begin{align*}
                \bigcup_{V\in\mathcal{V}} V
            \end{align*}
        \end{demobox}
        \item \mintinline{tex}{$ \frac{\sqrt{\mu}}{1-\rho(i)} $}\qquad $ \frac{\sqrt{\mu}}{1-\rho(i)} $
        \item \mintinline{tex}{$ \binom{6}{2} = \frac{6!}{2!4!} = 15 $}\qquad $ \binom{6}{2} = \frac{6!}{2!4!} = 15 $
        \item \mintinline{tex}{$ \mathbb{R}^{+} $}\qquad $ \mathbb{R}^{+} $
        \item \mintinline{tex}{$ \mathbb{R}_{>0} $}\qquad $ \mathbb{R}_{>0} $
        \item \mintinline{tex}{$ \mathbb{R}_{>0} = \left\{x\in\mathbb{R}\mid x > 0\right\} $}\qquad $ \mathbb{R}_{>0} = \left\{x\in\mathbb{R}\mid x > 0\right\} $
        \item \mintinline{tex}{$ \mathbb{R}_{>0} = \{x\in\mathbb{R} | x > 0\} $}\qquad $ \mathbb{R}_{>0} = \{x\in\mathbb{R} | x > 0\} $
        \item \mintinline{tex}{$ \mathbb{Z}_{\geq 0} = \left\{x\in\mathbb{Z}\mid x > 0\right\} $}\qquad $ \mathbb{Z}_{\geq 0} = \left\{x\in\mathbb{Z}\mid x > 0\right\} $
        \item \mintinline{tex}{$ a\qquad b $}\qquad {$ a\qquad b $}
        \item \mintinline{tex}{$ a^b/c\cdot d^{\sqrt[3]{2}} $}\qquad $ a^b/c\cdot d^{\sqrt[3]{2}} $
        \item \mintinline{tex}{$ 5\in\mathbb{Z} $}\qquad $ 5\in\mathbb{Z} $
        \item \mintinline{tex}{$ 5.3\not\in\mathbb{Z} $}\qquad $ 5.3\not\in\mathbb{Z} $
        \item \mintinline{tex}{$ \mathbb{R}\not\subseteq \mathbb{Z} $}\qquad $ \mathbb{R}\not\subseteq \mathbb{Z} $
        \item \mintinline{tex}{$ \mathbb{Z}\subseteq \mathbb{R} $}\qquad $ \mathbb{Z}\subseteq \mathbb{R} $
        \item \mintinline{tex}{$ 2\mathbb{Z} = \left\{x\in\mathbb{Z}\mid\text{$ x $ is even}\right\} $}\qquad $ 2\mathbb{Z} = \left\{x\in\mathbb{Z}\mid\text{$ x $ is even}\right\} $
        \item \mintinline{tex}{$ \langle a,b\rangle $}\qquad $ \langle a,b\rangle $
        \item \mintinline{tex}{$ f'(x) := \frac{\dif f(x)}{\dif x} $}\qquad $ f'(x) := \frac{\dif f(x)}{\dif x} $ (commath en mathtools nodig)
        \item \mintinline{tex}{$ f'(x) := \od{f(x)}{x} $}\qquad $ f'(x) := \od{f(x)}{x} $ (commath en mathtools nodig)
        \item \mintinline{tex}{$ (x=4\wedge y^2=x)\implies (y=2\vee y=-2) $}\qquad $ (x=4\wedge y^2=x)\implies (y=2\vee y=-2) $
        \item \mintinline{tex}{$ (x=4\text{ en } y^2=x)\implies (y=2\text{ of } y=-2 )$}\qquad $ (x=4\text{ en } y^2=x)\implies (y=2\text{ of } y=-2 )$
        \item \begin{minted}{tex}
            Voor $ x, y\in\real $ met $ y\geq 0 $ geldt
            \begin{align*}
                &x^2 = y\\
                &\iff x = \sqrt{y}\text{ of } x=-\sqrt{y}\\
                &\iff \abs{x} = \sqrt{y}.
            \end{align*}
        \end{minted}
        \begin{demobox}
            Voor $ x, y\in\real $ met $ y\geq 0 $ geldt
            \begin{align*}
                &x^2 = y\\
                &\iff x = \sqrt{y}\text{ of } x=-\sqrt{y}\\
                &\iff \abs{x} = \sqrt{y}.
            \end{align*}
        \end{demobox}
        \item \begin{minted}{tex}
            We evalueren de integraal,
            \begin{align*}
                \int_{a}^{b}x^2\dif x &= \left.\left[\frac{x^3}{3}\right]\right|_{x=a}^{x=b}\\
                &= \frac{b^3-a^3}{3}
            \end{align*}
        \end{minted}
        \begin{demobox}
            We evalueren de integraal,
            \begin{align*}
                \int_{a}^{b}x^2\dif x &= \left.\left[\frac{x^3}{3}\right]\right|_{x=a}^{x=b}\\
                &= \frac{b^3-a^3}{3}
            \end{align*}
        \end{demobox}
        \item Zonder de \mintinline{tex}{\left} en \mintinline{tex}{\right}:        
        \begin{minted}{tex}
            We evalueren de integraal,
            \begin{align*}
                \int_{a}^{b}x^2\dif x &= [\frac{x^3}{3}]|_{x=a}^{x=b}\\
                &= \frac{b^3-a^3}{3}
            \end{align*}
        \end{minted}
        \begin{demobox}
            We evalueren de integraal,
            \begin{align*}
                \int_{a}^{b}x^2\dif x &= [\frac{x^3}{3}]|_{x=a}^{x=b}\\
                &= \frac{b^3-a^3}{3}
            \end{align*}
        \end{demobox}
        \item \mintinline{tex}{$ \tilde{a} $}\qquad $ \tilde{a} $
        \item \mintinline{tex}{$ \hat{a} $}\qquad $ \hat{a} $
        \item \mintinline{tex}{$ \widehat{a} $}\qquad $ \widehat{a} $
        \item \mintinline{tex}{$ \widetilde{a} $}\qquad $ \widetilde{a} $
        \item \mintinline{tex}{$ a\sim b $}\qquad $ a\sim b $
        \item \mintinline{tex}{$ x\overset{\text{def}}{=} y+3 $}\qquad $ x\overset{\text{def}}{=} y+3 $
        \item Zonder de \mintinline{tex}{\left} en \mintinline{tex}{\right}:        
        \begin{minted}{tex}
            We evalueren de integraal,
            \begin{align*}
                \int_{a}^{b}x^2\dif x
                &= \underbrace{[\frac{x^3}{3}]|_{x=a}^{x=b}}_{\text{Dit ziet er lelijk uit}}\\
                &= \frac{b^3-a^3}{3}
            \end{align*}
        \end{minted}
        \begin{demobox}
            We evalueren de integraal,
            \begin{align*}
                \int_{a}^{b}x^2\dif x &= \underbrace{[\frac{x^3}{3}]|_{x=a}^{x=b}}_{\text{Dit ziet er lelijk uit}}\\
                &= \frac{b^3-a^3}{3}
            \end{align*}
        \end{demobox}
        \item \mintinline{tex}{Voor vectoren wordt zowel de notatie $ \vec{x} $ als $ \mathbf{x} $ gebruikt.}\qquad Voor vectoren wordt zowel de notatie $ \vec{x} $ als $ \mathbf{x} $ gebruikt.
        \item
         \begin{minted}{tex}
            Drie lelijke manieren om een identiteitsmatrix te noteren zijn
            \begin{align*}
                \begin{pmatrix}
                    1 & 0 & \cdots & 0\\
                    0 & \ddots & \ddots & \vdots\\
                    \vdots & \ddots & \ddots & 0\\
                    0 & \cdots & 0 & 1
                \end{pmatrix}
                = \begin{pmatrix}
                    1 & 0 & 0\\
                    0 &\ddots & 0\\
                    0 & 0 & 1
                \end{pmatrix}
                = \begin{pmatrix}
                    1 &  & \\
                    &\ddots & \\
                     &  & 1
                \end{pmatrix}
            \end{align*}
        \end{minted}
        \begin{demobox}
            Drie lelijke manieren om een identiteitsmatrix te noteren zijn
            \begin{align*}
                \begin{pmatrix}
                    1 & 0 & \cdots & 0\\
                    0 & \ddots & \ddots & \vdots\\
                    \vdots & \ddots & \ddots & 0\\
                    0 & \cdots & 0 & 1
                \end{pmatrix}
                = \begin{pmatrix}
                    1 & 0 & 0\\
                    0 &\ddots & 0\\
                    0 & 0 & 1
                \end{pmatrix}
                = \begin{pmatrix}
                    1 &  & \\
                    &\ddots & \\
                     &  & 1
                \end{pmatrix}
            \end{align*}
        \end{demobox}
    \end{enumerate}
\end{document}
