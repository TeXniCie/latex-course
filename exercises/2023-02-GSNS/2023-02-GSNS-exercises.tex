\documentclass[a4paper]{article}

\newif\ifishandout
\ishandoutfalse
%\ishandouttrue

\ifishandout
\documentclass[handout,aspectratio=32]{beamer}
\else
\documentclass[aspectratio=32]{beamer}
\fi

\usepackage[tabsize=4]{highlightlatex}

\setbeamertemplate{caption}[numbered]

\usecolortheme{rose}
%\useinnertheme[shadow]{rounded}
\useinnertheme{rounded}

\usetheme{Dresden}
\usecolortheme{dolphin}
\useoutertheme{miniframes}

\usepackage{subfiles}
\usepackage{amsmath,amssymb,amsthm,commath,mathtools}
\usepackage{esint}
\usepackage{enumerate}
\usepackage{subcaption}
\usepackage{graphicx}
\usepackage{xcolor}
\usepackage{adjustbox}
\usepackage{soul}
\usepackage{booktabs}
\usepackage{tabularx}
\usepackage{environ}
\usepackage[dutch]{babel}
\usepackage[utf8]{inputenc}
\usepackage{fancyvrb}
\usepackage{marvosym}
\usepackage{csquotes}
\usepackage[style=numeric]{biblatex}
\usepackage{textcomp}
%\usepackage{enumitem}
\usepackage{hyperref}
\usepackage{xkeyval}

\addbibresource{\subfix{assets/fakebib.bib}}

\DeclareMathOperator{\Image}{Image}

% Source: https://tex.stackexchange.com/questions/41683/why-is-it-that-coloring-in-soul-in-beamer-is-not-visible
\let\UL\ul
\makeatletter
\renewcommand\ul{
	\let\set@color\beamerorig@set@color
	\let\reset@color\beamerorig@reset@color
	\UL
}

\let\ST\st
\makeatletter
\def\st#1{
	\begingroup
	\let\set@color\beamerorig@set@color
	\let\reset@color\beamerorig@reset@color
	\def\SOUL@uleverysyllable{%
		\rlap{%
			%\color{red}
			\the\SOUL@syllable
			\SOUL@setkern\SOUL@charkern}%
		\SOUL@ulunderline{%
			\phantom{\the\SOUL@syllable}}%
	}%
	\ST{#1}%
	\endgroup
}
\makeatother
% https://tex.stackexchange.com/questions/71051/strikeout-in-different-color-appears-behind-letters-not-on-top-of-them

\setulcolor{red}
\setstcolor{red}

% Override if you want. Else you can delete it.
%\colorlet{curlyBrackets}{red!50!blue}
%\colorlet{squareBrackets}{blue!50!white}
%\colorlet{codeBackground}{gray!10!white}
%\colorlet{comment}{green!40!black}

\updatehighlight{
	name = default,
	color = {blue!90!black},
	add = {
		\knowncommand, \figref, \textcolor, \maketitle, \subsubsection,
		\textasciigrave, \textasciiacute, \tag, \middle, \mathbb, \abs,
		\mathcal, \middle, \dfrac, \subfile, \autoref, \eqref, \cites,
		\tableofcontents, \printbibliography, \fullcite, \parencite,
		\addbibresource, \DeclareLanguageMapping, \textcite, \intertext,
		\sum, \dif, \norm, \text, \dod, \dpd, \int, \partial,
		\DeclareMathOperator
	},
	name = structure,
	add = {
	},
}

\updatehighlight{
	name = greenDollar,
	style = {\itshape\color{green!70!black}},
	add = {
		% The dollar sign is provided an extra time just to
		% calm down TeXstudio's code highlighting.
		$, $
	},
	name = accentA,
	color = green!60!black,
	add = {
		\inAccA
	},
	%
	name = accentB,
	color = red!60!black,
	add = {
		\inAccB, \includegraphics
	},
	%
	name = accentC,
	color = orange!100!black,
	add = {
		\inAccC
	}
}

\lstset{tabsize=4}
\def\defaultgobble{8}

%\hllconfigure{
%	gobbletabs=3,
%}

\def\Zphantomconceal#1#2{%
	\only<#2->{\rlap{#1}}\phantom{#1}%
	%\only<#2->{#3}\unless\ifishandout\only<-#1>{\phantom{#3}}\fi
}

\def\phantomconceal#1#2{%
	\Zphantomconceal{#1}{#2}%
}

\newcommand\hideformula[2][2]{%
	%\hll|$| \only<2->{\hll|\\sqrt\{2\}|}\only<-1>{??} \hll|$|
	\hll|$| \phantomconceal{\hll|#2|}{#1} \hll|$|
}

\newcommand\hidelatex[2][2]{%
	\phantomconceal{\hll|#2|}{#1}
}%

\newcount\showcount

%\newcommand\showformula[2]{%
%	#1 & %
%	\expandafter\hideformula\expandafter[\the\showcount]{#2}%
%}
%
%\newcommand\showformula[2]{%
%	\global\showcount=\numexpr\showcount + 1\relax
%	\showformula*{#1}{#2}%
%}

\makeatletter

\def\showformula@i#1#2{%
	#1 & %
	\expandafter\hideformula\expandafter[\the\showcount]{#2}%
}

%\def\showformula{%
%	\@ifstar{%
%		\global\showcount=\numexpr\showcount + 1\relax
%		\showformula@i
%	}{%
%		\showformula@i
%	}%
%}

\def\showformula#1#2{
	#1 & \global\showcount=\numexpr\showcount + 1\relax
	\expandafter\hideformula\expandafter[\the\showcount]{#2}%
}

\def\showformulaa#1#2{
	#1 & %
	\expandafter\hideformula\expandafter[\the\showcount]{#2}%
}

\def\showlatex#1#2{
	#1 & \global\showcount=\numexpr\showcount + 1\relax
	\expandafter\hidelatex\expandafter[\the\showcount]{#2}%
}

\def\showlatexx#1#2{
	#1 & %
	\expandafter\hidelatex\expandafter[\the\showcount]{#2}%
}

\makeatother

\newlength{\naturalwidth}
\newlength{\minimumwidth}
\newbox\naturalsizebox
\newcommand{\atleastwidth}[2][2cm]{%
	\savebox\naturalsizebox{#2}%
	\settowidth\naturalwidth{#2}%
	\naturalwidth=\wd\naturalsizebox
	\minimumwidth=\dimexpr #1\relax
	\leavevmode%(\the\naturalwidth, \the\minimumwidth)%
	\ifdim\naturalwidth<\minimumwidth\relax
	\makebox[\minimumwidth][l]{\usebox{\naturalsizebox}}%
	\else
	\usebox{\naturalsizebox}%
	\fi
}

\newcommand{\atleastwidthr}[2][2cm]{%
	\savebox\naturalsizebox{#2}%
	\settowidth\naturalwidth{#2}%
	\naturalwidth=\wd\naturalsizebox
	\minimumwidth=\dimexpr #1\relax
	\leavevmode%(\the\naturalwidth, \the\minimumwidth)%
	\ifdim\naturalwidth<\minimumwidth\relax
	\makebox[\minimumwidth][r]{\usebox{\naturalsizebox}}%
	\else
	\usebox{\naturalsizebox}%
	\fi
}

\lstset{framexleftmargin=0.25em,xleftmargin=0.25em}

\NewEnviron{bluebox}{
	\begingroup
		\adjustbox{cfbox=blue!40!white 2pt 10pt,valign=t,bgcolor=blue!5!white}{%
			\begin{minipage}[t]{\dimexpr\linewidth-24pt\relax}
				\BODY
			\end{minipage}%
		}%
	\endgroup
}

\newcounter{maxrecentdisplay}
\setcounter{maxrecentdisplay}{27}

\newcounter{recentcount}
\setcounter{recentcount}{0}

\newcounter{recentskipremaining}

\def\vertlistsep{\hspace{2em}\textcolor{white!100!black}{\vrule width 0.5pt height 0.7\baselineskip\relax}\hspace{2em}}

\def\recentlist{}

%\newcommand{\addtorecentlist}[1]{%
%	\let\do\relax
%	\xdef\recentlist{\recentlist\do{#1}}%
%}

\newcommand{\addtorecentlist}[1]{%
	\bgroup
		\let\do\relax
		\expandafter\gdef\expandafter\recentlist\expandafter{\recentlist\do{#1}}%
		\addtocounter{recentcount}{1}%
	\egroup
	%
	%\xdef\recentlist{\recentlist\do{#1}}%
}

\newcommand{\clearrecentlist}{%
	\gdef\recentlist{}%
	\setcounter{recentcount}{0}%
}

\newif\ifisfirstrecentitem
\newcommand{\printrecentlist}{%
	\setcounter{recentskipremaining}{0}%
	\ifnum\value{recentcount}>\value{maxrecentdisplay}
		\setcounter{recentskipremaining}{\value{recentcount}-\value{maxrecentdisplay}}
	\fi
	%(\therecentskipremaining)
	%(\meaning\recentlist)
	\isfirstrecentitemtrue
	\def\do##1{%
		\ifnum\value{recentskipremaining}>0\relax
			\addtocounter{recentskipremaining}{-1}%
		\else		
			\unless\ifisfirstrecentitem
			\vertlistsep
			\fi
			\isfirstrecentitemfalse
			\textbf{##1}%
		\fi
	}%
	\recentlist
}

\newcommand{\recentpopfront}[1][1]{%
	\typeout{recentpopfront, before: \meaning\recentlist}
	\setcounter{recentskipremaining}{#1}%
	\let\origrecentlist\recentlist
	\clearrecentlist
	\def\do##1{%
		\ifnum\value{recentskipremaining}>0\relax
			\addtocounter{recentskipremaining}{-1}%
		\else		
			\addtorecentlist{##1}%
		\fi
	}%
	\origrecentlist
	\typeout{recentpopfront, after: \meaning\recentlist}
}

\newsavebox\printrecentbox
\savebox\printrecentbox{}
\newsavebox\scratchbox

% \AtBeginDocument{
% \setbox\scratchbox\printrecentbox
% }

\newcommand{\saveprintrecentbox}{%
	\bgroup
		\savebox\printrecentbox{\printrecentlist}%
		\global\setbox\printrecentbox\box\printrecentbox
	\egroup
	% \setbox\scratchbox\printrecentbox
	% \global\setbox\printrecentbox\scratchbox
	% \ifdim\wd\printrecentbox>0.9\textwidth
	% 	\savebox\printrecentbox{\adjustbox{right=0.9\textwidth}{\printrecentlist}}%
	% \else
	% 	\savebox\printrecentbox{\adjustbox{left=0.9\textwidth}{\printrecentlist}}%
	% \fi
}

\newcommand{\shrinkrecentbox}[1]{%
	{\loop
		%\clearrecentlist
		%\saveprintrecentbox
		%(SavedEmptyBox)
		%\iffalse


		\ifdim\wd\printrecentbox>\dimexpr #1\relax
		%
		\recentpopfront[1]%
		\saveprintrecentbox
	\repeat}%
}

% Based on miniframes code
\setbeamertemplate{headline}
{%
	\begin{beamercolorbox}[colsep=1.5pt]{upper separation line head}
	\end{beamercolorbox}
	\begin{beamercolorbox}{section in head/foot}
		\vskip2pt\insertnavigation{\paperwidth}\vskip2pt
	\end{beamercolorbox}%
	%
	\begin{beamercolorbox}[colsep=1.5pt]{middle separation line head}
	\end{beamercolorbox}
	\begin{beamercolorbox}[
		ht=2.5ex,
		dp=1.125ex,
		leftskip=.3cm,rightskip=.3cm plus1fil
		]{subsection in head/foot}
		\usebeamerfont{subsection in head/foot}%\insertsubsectionhead
		% \savebox\printrecentbox{\printrecentlist}%
		% \ifdim\wd\printrecentbox>0.9\textwidth
		% 	\adjustbox{right=0.9\textwidth}{\printrecentlist}%
		% \else
		% 	\adjustbox{left=0.9\textwidth}{\printrecentlist}%
		% \fi
		\saveprintrecentbox
		\ifdim\wd\printrecentbox>0.9\textwidth
			%(Shrinking box)
			%\PackageError{debug}{Width is \the\wd\printrecentbox}{}%
			\shrinkrecentbox{0.6\textwidth}%
		\else
			%(Not shrinking box)
		\fi
		\usebox\printrecentbox
		%\textbullet\ Hey
	\end{beamercolorbox}%
	%
	\begin{beamercolorbox}[colsep=1.5pt]{lower separation line head}
	\end{beamercolorbox}
}

\makeatletter

\NewEnviron{colC}[2][]{%
	\def\setpadd{}%
	\if\relax #1\relax
	\else
		%\def\setpadd{padding={0pt {\dimexpr ((#1)-\height)\relax} {0pt} {0pt}}}%
		\def\setpadd{%
			set depth={\dimexpr (#1)-\height\relax}%
		}
	\fi
	% \def\setparboxargs{}%
	% \if\relax #1\relax
	% \else
	% 	\def\setparboxargs{[t][\dimexpr #1\relax][]}%
	% \fi
	%
	\expandafter\adjustbox\expandafter{\setpadd,
		%margin=0pt,padding=0pt,
	%padding={0pt {\dimexpr (0.4\textheight-\height)/2\relax} {0pt} {\dimexpr (0.4\textheight-\height)/2\relax}},
		fbox=1pt 0pt 0pt,
		valign=M
	}%
	{%
		\parbox{\dimexpr #2-2pt\relax}{%
			\BODY
		}%
	}%
}

\NewEnviron{colT}[2][]{%
	\def\setpadd{}%
	\if\relax #1\relax
	\else
		\def\setpadd{%
			set depth={\dimexpr (#1)-\height\relax}%
		}%
	\fi
	%
	\expandafter\adjustbox\expandafter{\setpadd,
		fbox=1pt 0pt 0pt,
		valign=T
	}%
	{%
		\parbox{\dimexpr #2-2pt\relax}{%
			\BODY
		}%
	}%
}

\makeatother

\newlength\atleastlength


\newenvironment{noindentlist}{
	\begin{list}{\textbullet}{
		\leftmargin=0pt\relax
		\itemindent=0pt\relax
		\setlength{\itemsep}{2pt}
	}
}{
	\end{list}
}




\title{\vspace{-65pt} Exercises LaTeX workshop}
\author{\TeX niCie\\
{\small (Thomas, Vincent \& Hanneke)}
%\\{\small (Vincent Kuhlmann)}
}
\date{September 6, 2023}

\usepackage{minted}
\setminted{fontsize=\small, autogobble=true, linenos=false, frame=none}
% \setminted[tex]{fontsize=\small, autogobble=true, linenos=false, frame=none}
% \setminted[json]{fontsize=\small, autogobble=true, linenos=false, frame=none}

%\usemintedstyle{pastie}

\usepackage{wrapfig}
%\usepackage{cutwin}

\setcounter{secnumdepth}{0}

\begin{document}
\maketitle

% \CheckBox[]{aaa}

Remember the slides are available on \url{https://texnicie.nl}\quad Also, make sure you have at least these lines in your preamble:
\begin{minted}{tex}
        \usepackage[a4paper,margin=2.54cm]{geometry}
        \usepackage{amsmath,amssymb,amsthm}
        \usepackage{graphicx}
    \end{minted}

\section{Part 1: Text document}
% Zorg dat je steeds minstens deze packages hebt in je preamble:
% \begin{minted}{tex}
%         \usepackage[a4paper,margin=2.54cm]{geometry}
%         \usepackage{amsmath,amssymb,amsthm}
%         \usepackage[bookmarksnumbered]{hyperref}
%     \end{minted}
% \bigskip

\begin{exercise}[first document]
    Create a document with a title and a first line of text. Set the author to
    be your name. Change the paper size to a5paper, and set the margins to 1cm.
\end{exercise}

\begin{exercise}[emphasize]
    Emphasize some text by using \mintinline{tex}{\emph{your text}}. Put some another word
    or phrase in bold.
    
    % Extra: Can you find the difference between \mintinline{tex}{\emph} and \mintinline{tex}{\textit}?
\end{exercise}

\begin{exercise}[flushright]
    Find out what the \mintinline{tex}{\flushright} command does.
\end{exercise}

\begin{exercise}[headings]
    Create headings (section, subsection etc.), and create
    a table of contents for it. The table of contents should be on its own page.
\end{exercise}

\begin{exercise}[spacing]
    Let's make your document very\; s\,p\,a\,c\,i\,o\,u\,s. First, add the following
    lines to your preamble:
    \begin{minted}{tex}
    \usepackage{parskip}

    \setlength{\parskip}{20pt}
    \renewcommand{\baselinestretch}{1.5} 
    \end{minted}

    % They will make sure we are using paragraph spacing, with 20pt between paragraphs
    % (you can also use a value in centimeters if you prefer that). Additionally, the
    % line spacing is
    % set to 1.5.\footnote{If you are interested in how line spacing can be changed mid-document, see \url{https://tex.stackexchange.com/a/241121}}
    
    Check if this increases paragraph spacing and line spacing.

    Next, change the vertical margins to be 4 cm. Refer to the manual of the geometry
    package, or try what the following package options for geometry do: \mintinline{tex}{top=},
    \mintinline{tex}{bottom=}, \mintinline{tex}{vmargin=}.
    
\end{exercise}

\begin{exercise}[hyphenation]
    LaTeX can hyphenate words automatically. For this it needs the \texttt{babel} package,
    with package option \texttt{english} (i.e. \mintinline{tex}{\usepackage[english]{babel}}).
    Try to produce such hyphenation in your document.

    Hint: if you are having difficulty, increase the horizontal margin size, and change the paper
    size to A5 if you haven't already.
\end{exercise}

\begin{exercise}[special characters]
    Reproduce the following text:

    {\itshape When I woke up this morning, the temperature in my room was 13°C with 75\%
    humidity. I wrote down this data on my ``C:\textbackslash{}'' drive, in a file
    called temp\_room.txt. That morning the dollar-to-euro exchange rate was
    \$1.00 is €0.84.%
    % \footnote{This exchange rate was real in February 2021. How
    % times have changed. You don't have to reproduce this footnote, however you could with
    % \mintinline{tex}{\footnote{}}. Or just insert a superscript 1, if you really want that.}
    }

    Hints:
    \begin{itemize}
        \item You can use \mintinline{tex}{\textdegree} instead of pasting in a degree
        symbol, if you use \mintinline{tex}{\usepackage{gensymb}}.
        \item Look at the slide of typing special characters literally.
        \item Use \mintinline{tex}{\usepackage{lmodern}} for a nicer euro symbol.
        (You can enter a euro symbol directly in the code)
        \item For special characters it is often advisable to use \mintinline{tex}{\usepackage[utf8]{inputenc}}
        (which Overleaf includes by default). Then more characters can be typed in directly in code.
    \end{itemize}
\end{exercise}

\begin{exercise}[parskip]
    Add two paragraphs to your document, and observe the difference with \mintinline{tex}{\usepackage{parskip}}
    and without it. Which style do you prefer?
\end{exercise}

\begin{exercise}[manual spacing]
    Find out what the following commands do: \mintinline{tex}{\quad}, \mintinline{tex}{\qquad}, \mintinline{tex}{\hspace{2cm}},
    \mintinline{tex}{\;}, \mintinline{tex}{\!}, \mintinline{tex}{\vspace{2cm}}, \mintinline{tex}{\bigskip}.
\end{exercise}

\begin{exercise}[colors]
    Add package \mintinline{tex}{\usepackage{xcolor}}, produce the following text
    in red and orange colors:
    \textbf{\textcolor{orange}{Hi, I like the color \textcolor{red}{red}.}}
\end{exercise}

\section{Part 2: Formulas and figures}

\begin{exercise}
    \textit{Recreate the following expression in inline mode:}

$$\left(\frac{x^3}{3(x+1)^2}\right)^{\frac{1}{n}}$$

\end{exercise}
\begin{exercise}
\textit{Recreate the following proof by using align:}

    \bgroup\small
	The solution of $ax^2+bx+c=0$ where $a\neq 0$ is

	\begin{align}
		\frac{-b\pm \sqrt{d}}{2a}\text{ where }d=b^2-4ac
	\end{align}
	\begin{proof}
		We see that the equation is equivalent to
		\begin{align}
			ax^2+bx&=-c
			\intertext{Or equivalently}
			-\frac{c}{a}=x^2+\frac{b}{a}x&=x^2+2\frac{b}{2a}x
			\intertext{By adding $\left(\frac{b}{2a})\right)^2$ to both sides we get} 
			\left(\frac{b}{2a}\right)^2-\frac{c}{a}&=x^2+2\frac{b}{2a}+\left(\frac{b}{2a}\right)^2\\
			&=\left(x+\frac{b}{2a}\right)^2
			\intertext{By multiplying $4a^2$ to bot sides we get}
			b^2-4ac&=(2ax+b)^2
			\intertext{So}
			\pm \sqrt{b^2-4ac}&=2ax+b
			\intertext{And therefore}
			\frac{-b \pm \sqrt{b^2-4ac}}{2a}&=x
		\end{align}
	\end{proof}
    \egroup
\end{exercise}

\begin{exercise}[basic image]
    Find an image of your favourite animal species, and upload the image into your
    Overleaf document. First, use a direct \mintinline{tex}{\includegraphics{...}}
    with \mintinline{tex}{...} the name of the image. If this works, wrap
    a proper figure environment around it as seen in the slides.
\end{exercise}

\begin{exercise}[reference]
    Add a reference to a numbered equation and a figure in your text. Use the proper
    \LaTeX{} way of doing this, i.e. with \mintinline{tex}{\label{fig:cuteanimal}}
    and \mintinline{tex}{\ref{fig:cuteanimal}}. This ensures the numbers will stay
    correct.
\end{exercise}

\begin{exercise}[image trimming]
    You can crop an image from within \LaTeX{} using this command:
    \begin{minted}{tex}
        \includegraphics[width=0.9\linewidth,trim=10pt 10pt 10pt 10pt,clip]{example-image-a}
    \end{minted}
    Observe how changing the 4 numbers in the trim option (corresponding to left, bottom,
    right, top respecitvely) affects the cropping. Make sure you have added \mintinline{tex}{\usepackage{graphicx}}
    to your preamble!
\end{exercise}

% \begin{exercise}[PDF-viewer]
%     Ga naar het TeX-tabje in de activity bar links in VS~Code. Probeer de verschillende opties
%     onder `View LaTeX PDF'. Wat vind je het fijnste werken?
%     Verander de \texttt{latex-workshop.view.pdf.viewer} optie in de settings als je
%     een andere default wil.
% \end{exercise}

% \begin{exercise}[Inline math shortcut]
%     Stel een shortcut in voor het invoegen van inline math. Bijvoorbeeld door
%     het volgende toe te voegen aan je \texttt{keybindings.json}:
%     \begin{minted}{json}
%         {
%             "key": "ctrl+shift+m",
%             "when": "editorTextFocus && editorLangId == latex",
%             "command": "editor.action.insertSnippet",
%             "args": {
%                 "snippet": "\\$ ${1:} \\$$0"
%             }
%         },
%     \end{minted}
%     Check dat dit werkt.
% \end{exercise}

% \begin{exercise}[Errors en warnings]
%     Maak een error door een align met een witregel erin, en daarna een warning door
%     \mintinline{tex}{\label} twee keer te gebruiken met hetzelfde argument.
%     Waar zie je de errors en warnings in Visual Studio Code?
% \end{exercise}

% \begin{exercise}[LaTeX Workshop snippets]
%     Ga naar de volgende URL:
    
%     \url{https://github.com/James-Yu/LaTeX-Workshop/wiki/Snippets}.
    
%     Stel de \texttt{editor.suggest.snippetsPreventQuickSuggestions} in zoals aangegeven op de
%     pagina. Probeer vervolgens een figure, een section en een \mintinline{tex}{\textbf} te maken
%     met de default snippets en shortcuts die erop vermeld staan.
% \end{exercise}

% \begin{exercise}[Environment snippet]
%     Stel een snippet in voor het toevoegen van een environment. Kies als default environment
%     naam wat je denkt het meest te zullen gebruiken (bijvoorbeeld align).
% \end{exercise}

% \begin{exercise}[VS Code algemene shortcuts]
%     Ga naar \url{https://code.visualstudio.com/docs}, klik op `Keyboard Shortcut Reference Sheet'
%     en download de PDF voor jouw besturingssysteem.
%     Probeer wat shortcuts uit. Welke zouden voor jou handig kunnen zijn?
% \end{exercise}

% \begin{exercise}[Basisdocument snippet]
%     Maak een snippet die een basisdocument voor LaTeX voorziet, met alle \mintinline{tex}{\usepackage}'s
%     die je meestal nodig hebt.
% \end{exercise}

% \begin{exercise}[Python]
%     Als je Python kent, maak een Python bestand in VS Code, en zoek hoe je het kan
%     uitvoeren. Probeer ook de interactive console.
% \end{exercise}

% \pagebreak
% \section{Deel 2: Effici\"entie in LaTeX code}

% % Zorg dat je steeds minstens deze packages hebt in je preamble:
% % \begin{minted}{tex}
% %         \usepackage[a4paper,margin=2.54cm]{geometry}
% %         \usepackage{amsmath,amssymb,amsthm}
% %         \usepackage{graphicx}
% %         \usepackage{subcaption}
% %         \usepackage{booktabs}
% %         \usepackage[bookmarksnumbered]{hyperref}
% %     \end{minted}
% % \bigskip

% \begin{exercise}[Stelsel in matrix revisited]
% Stelsels lineaire vergelijkingen kunnen opgelost worden door ze te schrijven als een matrix en Gauss
% eliminatie toe te passen.
% Repliceer dit typische stelselmatrix:

% \begin{tabularx}{\textwidth}{Xp{0.7\textwidth}}
% \adjustbox{valign=t}{\small$\displaystyle
%     \left(\begin{array}{rrr|r}
%         2 & 1 & -1 & 8\\
%         -3 & -1 & 2 & -11\\
%         -2 & 1 & 2 & -3
%     \end{array}\right)
% $}&
% \parbox[t]{0.65\textwidth}{\small De eerste rij komt overeen met de vergelijking $ 2x+y-z=8 $.\\
% Getallenvoorbeeld van:\\\url{https://en.wikipedia.org/wiki/Gaussian_elimination}}
% \end{tabularx}

% Maak een environment hiervoor. Zorg dat het environment een argument heeft voor hoeveel
% kolommen er voor de verticale streep staan.

% Kan je dit een optioneel argument maken?
% \end{exercise}

% \begin{exercise}[Vector]
%     Definieer een commando die drie argumenten neemt, en er een kolommatrix van maakt.
% \end{exercise}

% \begin{exercise}[Commando \textbackslash input]
%     Kopieer het \texttt{.tex}-bestand van je vorige inleveropgave, en plaats de preamble ervan
%     in een ander bestand, dat je bijvoorbeeld \texttt{preamble.tex} noemt. Gebruik
%     \mintinline{tex}{\newif\ifishandout
\ishandoutfalse
%\ishandouttrue

\ifishandout
\documentclass[handout,aspectratio=32]{beamer}
\else
\documentclass[aspectratio=32]{beamer}
\fi

\usepackage[tabsize=4]{highlightlatex}

\setbeamertemplate{caption}[numbered]

\usecolortheme{rose}
%\useinnertheme[shadow]{rounded}
\useinnertheme{rounded}

\usetheme{Dresden}
\usecolortheme{dolphin}
\useoutertheme{miniframes}

\usepackage{subfiles}
\usepackage{amsmath,amssymb,amsthm,commath,mathtools}
\usepackage{esint}
\usepackage{enumerate}
\usepackage{subcaption}
\usepackage{graphicx}
\usepackage{xcolor}
\usepackage{adjustbox}
\usepackage{soul}
\usepackage{booktabs}
\usepackage{tabularx}
\usepackage{environ}
\usepackage[dutch]{babel}
\usepackage[utf8]{inputenc}
\usepackage{fancyvrb}
\usepackage{marvosym}
\usepackage{csquotes}
\usepackage[style=numeric]{biblatex}
\usepackage{textcomp}
%\usepackage{enumitem}
\usepackage{hyperref}
\usepackage{xkeyval}

\addbibresource{\subfix{assets/fakebib.bib}}

\DeclareMathOperator{\Image}{Image}

% Source: https://tex.stackexchange.com/questions/41683/why-is-it-that-coloring-in-soul-in-beamer-is-not-visible
\let\UL\ul
\makeatletter
\renewcommand\ul{
	\let\set@color\beamerorig@set@color
	\let\reset@color\beamerorig@reset@color
	\UL
}

\let\ST\st
\makeatletter
\def\st#1{
	\begingroup
	\let\set@color\beamerorig@set@color
	\let\reset@color\beamerorig@reset@color
	\def\SOUL@uleverysyllable{%
		\rlap{%
			%\color{red}
			\the\SOUL@syllable
			\SOUL@setkern\SOUL@charkern}%
		\SOUL@ulunderline{%
			\phantom{\the\SOUL@syllable}}%
	}%
	\ST{#1}%
	\endgroup
}
\makeatother
% https://tex.stackexchange.com/questions/71051/strikeout-in-different-color-appears-behind-letters-not-on-top-of-them

\setulcolor{red}
\setstcolor{red}

% Override if you want. Else you can delete it.
%\colorlet{curlyBrackets}{red!50!blue}
%\colorlet{squareBrackets}{blue!50!white}
%\colorlet{codeBackground}{gray!10!white}
%\colorlet{comment}{green!40!black}

\updatehighlight{
	name = default,
	color = {blue!90!black},
	add = {
		\knowncommand, \figref, \textcolor, \maketitle, \subsubsection,
		\textasciigrave, \textasciiacute, \tag, \middle, \mathbb, \abs,
		\mathcal, \middle, \dfrac, \subfile, \autoref, \eqref, \cites,
		\tableofcontents, \printbibliography, \fullcite, \parencite,
		\addbibresource, \DeclareLanguageMapping, \textcite, \intertext,
		\sum, \dif, \norm, \text, \dod, \dpd, \int, \partial,
		\DeclareMathOperator
	},
	name = structure,
	add = {
	},
}

\updatehighlight{
	name = greenDollar,
	style = {\itshape\color{green!70!black}},
	add = {
		% The dollar sign is provided an extra time just to
		% calm down TeXstudio's code highlighting.
		$, $
	},
	name = accentA,
	color = green!60!black,
	add = {
		\inAccA
	},
	%
	name = accentB,
	color = red!60!black,
	add = {
		\inAccB, \includegraphics
	},
	%
	name = accentC,
	color = orange!100!black,
	add = {
		\inAccC
	}
}

\lstset{tabsize=4}
\def\defaultgobble{8}

%\hllconfigure{
%	gobbletabs=3,
%}

\def\Zphantomconceal#1#2{%
	\only<#2->{\rlap{#1}}\phantom{#1}%
	%\only<#2->{#3}\unless\ifishandout\only<-#1>{\phantom{#3}}\fi
}

\def\phantomconceal#1#2{%
	\Zphantomconceal{#1}{#2}%
}

\newcommand\hideformula[2][2]{%
	%\hll|$| \only<2->{\hll|\\sqrt\{2\}|}\only<-1>{??} \hll|$|
	\hll|$| \phantomconceal{\hll|#2|}{#1} \hll|$|
}

\newcommand\hidelatex[2][2]{%
	\phantomconceal{\hll|#2|}{#1}
}%

\newcount\showcount

%\newcommand\showformula[2]{%
%	#1 & %
%	\expandafter\hideformula\expandafter[\the\showcount]{#2}%
%}
%
%\newcommand\showformula[2]{%
%	\global\showcount=\numexpr\showcount + 1\relax
%	\showformula*{#1}{#2}%
%}

\makeatletter

\def\showformula@i#1#2{%
	#1 & %
	\expandafter\hideformula\expandafter[\the\showcount]{#2}%
}

%\def\showformula{%
%	\@ifstar{%
%		\global\showcount=\numexpr\showcount + 1\relax
%		\showformula@i
%	}{%
%		\showformula@i
%	}%
%}

\def\showformula#1#2{
	#1 & \global\showcount=\numexpr\showcount + 1\relax
	\expandafter\hideformula\expandafter[\the\showcount]{#2}%
}

\def\showformulaa#1#2{
	#1 & %
	\expandafter\hideformula\expandafter[\the\showcount]{#2}%
}

\def\showlatex#1#2{
	#1 & \global\showcount=\numexpr\showcount + 1\relax
	\expandafter\hidelatex\expandafter[\the\showcount]{#2}%
}

\def\showlatexx#1#2{
	#1 & %
	\expandafter\hidelatex\expandafter[\the\showcount]{#2}%
}

\makeatother

\newlength{\naturalwidth}
\newlength{\minimumwidth}
\newbox\naturalsizebox
\newcommand{\atleastwidth}[2][2cm]{%
	\savebox\naturalsizebox{#2}%
	\settowidth\naturalwidth{#2}%
	\naturalwidth=\wd\naturalsizebox
	\minimumwidth=\dimexpr #1\relax
	\leavevmode%(\the\naturalwidth, \the\minimumwidth)%
	\ifdim\naturalwidth<\minimumwidth\relax
	\makebox[\minimumwidth][l]{\usebox{\naturalsizebox}}%
	\else
	\usebox{\naturalsizebox}%
	\fi
}

\newcommand{\atleastwidthr}[2][2cm]{%
	\savebox\naturalsizebox{#2}%
	\settowidth\naturalwidth{#2}%
	\naturalwidth=\wd\naturalsizebox
	\minimumwidth=\dimexpr #1\relax
	\leavevmode%(\the\naturalwidth, \the\minimumwidth)%
	\ifdim\naturalwidth<\minimumwidth\relax
	\makebox[\minimumwidth][r]{\usebox{\naturalsizebox}}%
	\else
	\usebox{\naturalsizebox}%
	\fi
}

\lstset{framexleftmargin=0.25em,xleftmargin=0.25em}

\NewEnviron{bluebox}{
	\begingroup
		\adjustbox{cfbox=blue!40!white 2pt 10pt,valign=t,bgcolor=blue!5!white}{%
			\begin{minipage}[t]{\dimexpr\linewidth-24pt\relax}
				\BODY
			\end{minipage}%
		}%
	\endgroup
}

\newcounter{maxrecentdisplay}
\setcounter{maxrecentdisplay}{27}

\newcounter{recentcount}
\setcounter{recentcount}{0}

\newcounter{recentskipremaining}

\def\vertlistsep{\hspace{2em}\textcolor{white!100!black}{\vrule width 0.5pt height 0.7\baselineskip\relax}\hspace{2em}}

\def\recentlist{}

%\newcommand{\addtorecentlist}[1]{%
%	\let\do\relax
%	\xdef\recentlist{\recentlist\do{#1}}%
%}

\newcommand{\addtorecentlist}[1]{%
	\bgroup
		\let\do\relax
		\expandafter\gdef\expandafter\recentlist\expandafter{\recentlist\do{#1}}%
		\addtocounter{recentcount}{1}%
	\egroup
	%
	%\xdef\recentlist{\recentlist\do{#1}}%
}

\newcommand{\clearrecentlist}{%
	\gdef\recentlist{}%
	\setcounter{recentcount}{0}%
}

\newif\ifisfirstrecentitem
\newcommand{\printrecentlist}{%
	\setcounter{recentskipremaining}{0}%
	\ifnum\value{recentcount}>\value{maxrecentdisplay}
		\setcounter{recentskipremaining}{\value{recentcount}-\value{maxrecentdisplay}}
	\fi
	%(\therecentskipremaining)
	%(\meaning\recentlist)
	\isfirstrecentitemtrue
	\def\do##1{%
		\ifnum\value{recentskipremaining}>0\relax
			\addtocounter{recentskipremaining}{-1}%
		\else		
			\unless\ifisfirstrecentitem
			\vertlistsep
			\fi
			\isfirstrecentitemfalse
			\textbf{##1}%
		\fi
	}%
	\recentlist
}

\newcommand{\recentpopfront}[1][1]{%
	\typeout{recentpopfront, before: \meaning\recentlist}
	\setcounter{recentskipremaining}{#1}%
	\let\origrecentlist\recentlist
	\clearrecentlist
	\def\do##1{%
		\ifnum\value{recentskipremaining}>0\relax
			\addtocounter{recentskipremaining}{-1}%
		\else		
			\addtorecentlist{##1}%
		\fi
	}%
	\origrecentlist
	\typeout{recentpopfront, after: \meaning\recentlist}
}

\newsavebox\printrecentbox
\savebox\printrecentbox{}
\newsavebox\scratchbox

% \AtBeginDocument{
% \setbox\scratchbox\printrecentbox
% }

\newcommand{\saveprintrecentbox}{%
	\bgroup
		\savebox\printrecentbox{\printrecentlist}%
		\global\setbox\printrecentbox\box\printrecentbox
	\egroup
	% \setbox\scratchbox\printrecentbox
	% \global\setbox\printrecentbox\scratchbox
	% \ifdim\wd\printrecentbox>0.9\textwidth
	% 	\savebox\printrecentbox{\adjustbox{right=0.9\textwidth}{\printrecentlist}}%
	% \else
	% 	\savebox\printrecentbox{\adjustbox{left=0.9\textwidth}{\printrecentlist}}%
	% \fi
}

\newcommand{\shrinkrecentbox}[1]{%
	{\loop
		%\clearrecentlist
		%\saveprintrecentbox
		%(SavedEmptyBox)
		%\iffalse


		\ifdim\wd\printrecentbox>\dimexpr #1\relax
		%
		\recentpopfront[1]%
		\saveprintrecentbox
	\repeat}%
}

% Based on miniframes code
\setbeamertemplate{headline}
{%
	\begin{beamercolorbox}[colsep=1.5pt]{upper separation line head}
	\end{beamercolorbox}
	\begin{beamercolorbox}{section in head/foot}
		\vskip2pt\insertnavigation{\paperwidth}\vskip2pt
	\end{beamercolorbox}%
	%
	\begin{beamercolorbox}[colsep=1.5pt]{middle separation line head}
	\end{beamercolorbox}
	\begin{beamercolorbox}[
		ht=2.5ex,
		dp=1.125ex,
		leftskip=.3cm,rightskip=.3cm plus1fil
		]{subsection in head/foot}
		\usebeamerfont{subsection in head/foot}%\insertsubsectionhead
		% \savebox\printrecentbox{\printrecentlist}%
		% \ifdim\wd\printrecentbox>0.9\textwidth
		% 	\adjustbox{right=0.9\textwidth}{\printrecentlist}%
		% \else
		% 	\adjustbox{left=0.9\textwidth}{\printrecentlist}%
		% \fi
		\saveprintrecentbox
		\ifdim\wd\printrecentbox>0.9\textwidth
			%(Shrinking box)
			%\PackageError{debug}{Width is \the\wd\printrecentbox}{}%
			\shrinkrecentbox{0.6\textwidth}%
		\else
			%(Not shrinking box)
		\fi
		\usebox\printrecentbox
		%\textbullet\ Hey
	\end{beamercolorbox}%
	%
	\begin{beamercolorbox}[colsep=1.5pt]{lower separation line head}
	\end{beamercolorbox}
}

\makeatletter

\NewEnviron{colC}[2][]{%
	\def\setpadd{}%
	\if\relax #1\relax
	\else
		%\def\setpadd{padding={0pt {\dimexpr ((#1)-\height)\relax} {0pt} {0pt}}}%
		\def\setpadd{%
			set depth={\dimexpr (#1)-\height\relax}%
		}
	\fi
	% \def\setparboxargs{}%
	% \if\relax #1\relax
	% \else
	% 	\def\setparboxargs{[t][\dimexpr #1\relax][]}%
	% \fi
	%
	\expandafter\adjustbox\expandafter{\setpadd,
		%margin=0pt,padding=0pt,
	%padding={0pt {\dimexpr (0.4\textheight-\height)/2\relax} {0pt} {\dimexpr (0.4\textheight-\height)/2\relax}},
		fbox=1pt 0pt 0pt,
		valign=M
	}%
	{%
		\parbox{\dimexpr #2-2pt\relax}{%
			\BODY
		}%
	}%
}

\NewEnviron{colT}[2][]{%
	\def\setpadd{}%
	\if\relax #1\relax
	\else
		\def\setpadd{%
			set depth={\dimexpr (#1)-\height\relax}%
		}%
	\fi
	%
	\expandafter\adjustbox\expandafter{\setpadd,
		fbox=1pt 0pt 0pt,
		valign=T
	}%
	{%
		\parbox{\dimexpr #2-2pt\relax}{%
			\BODY
		}%
	}%
}

\makeatother

\newlength\atleastlength


\newenvironment{noindentlist}{
	\begin{list}{\textbullet}{
		\leftmargin=0pt\relax
		\itemindent=0pt\relax
		\setlength{\itemsep}{2pt}
	}
}{
	\end{list}
}


} in je eigenlijke \texttt{.tex}-bestand.
%     Kan je het nog steeds compileren?
% \end{exercise}

% \begin{exercise}[Eigen documentclass]
%     Maak je eigen documentclass zoals aangegeven in de slides. Werkt het als je
%     in het kopie van je vorige inleveropgave de preamble vervangt door\newline
%     \mintinline{tex}{\documentclass{inleveropgave}}?
% \end{exercise}

% \begin{exercise}[aux-directory]\label{ex:auxDir}
%     In plaats van dat alle hulpbestanden zoals \texttt{.aux}, \texttt{.toc}, \texttt{.out}
%     je mapje onoverzichtelijk maken, is het mogelijk de locatie ervoor te veranderen naar een
%     ander mapje. Vraag Vincent als je benieuwd bent.

%     In VS Code, onder het TeX-tabje, gebruik `Clean up auxiliary files' onder `Build LaTeX project'.

%     Typ \mintinline{text}{"latex-workshop.latex.tools"} in je settings.json bestand, en gebruik
%     de auto-complete. Je krijgt een hele lijst met 'tools'. Voeg deze tool toe:
%     \begin{minted}{tex}
%         {
%             "name": "pdflatexDirs",
%             "command": "pdflatex",
%             "args": [
%                 "-synctex=1",
%                 "-interaction=nonstopmode",
%                 "-file-line-error",
%                 "-aux-directory=auxdir",
%                 "%DOC%"
%             ],
%             "env": {}
%         },
%     \end{minted}
%     Typ nu \mintinline{text}{"latex-workshop.latex.recipes"} in je settings.json bestand,
%     en gebruik weer auto-complete. Voeg bovenaan de recipes toe:
%     \begin{minted}{tex}
%         {
%             "name": "pdflatexDirs",
%             "tools": [
%                 "pdflatexDirs"
%             ]
%         },
%     \end{minted}
%     Sla op, en compileer je bestand. Je zou nu een nieuw mapje `auxdir' moeten zien, en
%     hierin staan al je auxiliary files.
% \end{exercise}

% \begin{exercise}[Snelle compilatie]
%     %Zie oefeningen PDF op texnicie.nl.

%     Maak een \texttt{.tex}-bestand en compileer het manueel met het \texttt{pdflatex}-commando
%     in je terminal.

%     Eenmaal dat is gelukt, voeg deze lijn bovenaan je \texttt{.tex}-bestand toe
%     \begin{minted}{text}
%         %&document_format
%         \documentclass{article}

%         ...
%     \end{minted}
%     en voer dit commando uit in je terminal:
%     \begin{minted}{text}
%         pdftex -ini -jobname="document" "&pdflatex" mylatexformat.ltx document.tex
%     \end{minted}

%     Als het goed is zie je nu een \texttt{document.fmt}-bestand. Dit is een cache van het moment
%     dat de preamble helemaal was ingeladen. Als je nu je \texttt{.tex}-bestand weer compileert
%     (kan ook via Visual Studio Code), zou dit veel sneller moeten gaan.

%     Maar let op! De cache kijkt niet of je preamble is veranderd, dus als je je preamble verandert
%     moet je je \texttt{document.fmt}-bestand verwijderen, en opnieuw maken.
%     Je kan het commando voor het maken van dit \texttt{.fmt}-bestand ook instellen in VS~Code.
%     Dit gaat analoog aan \autoref{ex:auxDir}, als
%     \begin{minted}{json}
%         {
%             "name": "mylatexformat",
%             "command": "pdftex",
%             "args": [
%                 "-ini",
%                 "-jobname=\"%DOC%_format\"",
%                 "&pdflatex",
%                 "mylatexformat.ltx",
%                 "%DOC%"
%             ],
%             "env": {}
%         },
%     \end{minted}

%     Let erop dat in je document de eerste regel \texttt{\%\&document\_format} is, en vervang \texttt{document\_format}
%     hier door de naam van je bestand zonder de \texttt{.tex}, en met \texttt{\_format} erachter.

%     Dit is een moeilijke opgave, en het kan zijn dat ik iets mis in de instructies. 
%     Vraag Vincent als het niet meteen lukt.
% \end{exercise}

% Exercises were created by members of the TeXniCie:
% \begin{verbatim}
%     Copyright (c) 2022-2023 Tim Weijers
%     Copyright (c) 2022-2023 Thomas van Maaren
%     Copyright (c) 2022-2023 Hanneke Schroten
%     Copyright (c) 2021-2023 Vincent Kuhlmann
% \end{verbatim}

\end{document}
