\documentclass[a4paper]{article}

\author{TeXniCie}
\date{\today}
\title{Manual for thesis}

%%% Overview of this file in order:
% Packages which don't need options
% Packages which have one or few options
% Geometry package
% Header/footer settings
% Theorem styles
% Enable/disable parindents
% Reference and bibliography settings
% Front/main/back-matter


%%%%%%%%%%%%%%%%%%%%%%%%%%%%%%%%%%%%%%%%%

\usepackage{
		%layout,		% Allow visualisation of all the margins
		subfiles,		% For separate main and sub documents
		graphicx,		% For image modifications and the figure environment
		amsmath,		% For the AMS math styles
		amssymb,		% The extended AMS math symbol list
		amsthm,			% For use of theorems (works together with thmtools)
		fancyhdr,		% For fancy headers and footers on pages
		%gensymb,		% For easy generic symbols (uniform use in math and text mode)
		%sidecap,		% For use of captions next to a float (figure, table, etc)
		subcaption,		% For easy subfigures in a plot (with nice captions)
		tikz,			% Difficult drawing of awesome vector plots
		%listings,		% For listing pieces of code in a nice and neat way
		multicol,		% For easy local multicolumn use
		color,			% For handy colour definitions (used cause of styling)
		%calc,			% To calculate stuff for the back-end
		%mdwlist,		% For customising lists
		thmtools,		% Lets you define your own theorem style (used for all the fancy theorems, definitions etc.)
		etoolbox,		% Allows adjustment of commands (used to reset the claim counter at the end of a proof).
		xspace,			% Makes latex not eat spaces after commands
        hyperref,		% Makes links, references, the Table of Contents, etc. clickable.
        mathtools,      % Extensible symbols, such as brackets, arrows, harpoons, etc.;
        inputenc,       % Allows to use things like ö instead of \"o in your text.
        wrapfig,        % wrap text around a smaller figure
        framed          % allows for boxes around your important equations or theorems
        }

%%%%%%%%%%%%%%%%%%%%%%%%%%%%%%%%%%%%%%%%%%

\usepackage[english]{babel} % Correct language setting, 'british', 'american'='english' or 'dutch'.
\usepackage[autostyle]{csquotes} % Fixes quotes to correspond to the babel language.
% Note the difference between ``quotes'' and ''quotes'' when using different languages.


%%%%%%%%%%%%%%%%%%%%%%%%%%%%%%%%%%%%%%%%%%

 \usepackage[margin=2.5cm]{geometry}
 % Change the shape of a page (custom margins etc.)
 % paper=a4paper slightly changes the style through the whole document.
    %%%% We set the margins for whole document here, except the titlepage. The titlepage uses special margins; see titlepage.tex.


%%%%%%%%%%%%%%%%%%%%%%%%%%%%%%%%%%%%%%%%%%

%%% This is about changing the headers and footers (i.e. Top and bottom of the page)

\pagestyle{fancy}% use fancyheaders with the bar on the top
\fancyhf{} % Clear the normal style
\fancyhead[L]{\bfseries\leftmark} %this places the section number and name in the top left
\fancyhead[R]{\bfseries\thepage}% this places the pagenumber in the top right
	

%%%%%%%%%%%%%%%%%%%%%%%%%%%%%%%%%%%%%%%%%%%%
%%%%		Theorem style

% The set-up is as follows, first you give the 'style' of your theorem. This determines whether it for instance is plain, or italic. Secondly you can give an option for the symbol on the end, normally it is nothing. But you could add some to increase the readability of your text. Finally you can use numberwithin to add the number of your section/theorem before your equations. This is useful if you want to keep the numbers of your equation in check (In this thesis there where over a 100) and keeps in order where the equations are.
%Finally you can use sibling to let different 'theorems' count together. Hence you will get Theorem 1 Definition 2 Claim 3, instead of Theorem 1 Definition 1 Claim 1. This is a matter of taste.

% Theorem definitions
\declaretheorem[style=definition,numberwithin=subsection]{definition} %If you want your theorems to be counted per section instead of subsection, then just remove the sub from the numberwithin
\declaretheorem[style=definition,qed=$\triangle$,sibling=definition]{example}% sibling says with what type of theorems you wan the numbering to count with.

\declaretheorem[style=plain,sibling=definition]{theorem}
\declaretheorem[style=plain,sibling=definition]{lemma}
\declaretheorem[style=plain,sibling=definition]{proposition}
\declaretheorem[style=plain,sibling=definition]{corollary}
\declaretheorem[style=definition]{claim}
\declaretheorem[style=definition,sibling=example]{remark}

\AtEndEnvironment{proof}{\setcounter{claim}{0}} % Sets the claim number to 0 after ending a proof

% You can make short-hands like these. 

\newcommand{\thm}[2]{\begin{theorem} #1 \begin{proof} #2 \end{proof} \end{theorem}}
\newcommand{\lm}[2]{\begin{lemma} #1 \begin{proof} #2 \end{proof} \end{lemma}}
\newcommand{\df}[1]{\begin{definition} #1 \end{definition}}
\newcommand{\clm}[1]{\begin{claim} #1 \end{claim}}


%%%%%%%%%%%%%%%%%%%%%%%%%%%%%%%%%%%%%%%%%%%%

%Let equation numbers be numbered by 1.1, 1.2, 2.1, etc where the first number is the section number. section can also be replaced by chapter when using book class.
\numberwithin{equation}{section}

%%%%%%%%%%%%%%%%%%%%%%%%%%%%%%%%%%%%%%%%%%%%

% Comment/uncomment the following to disable/enable parindents:
\setlength\parindent{0pt}

%%%%%%%%%%%%%%%%%%%%%%%%%%%%%%%%%%%%%%%%%%%%

%%%% Add the bibliography with some settings:
% package:
\usepackage[% Options
style = numeric-comp, % 
% Choose the style of your citations (and correspondingly your bibliography).
% Few examples:
% numeric = [15, 16, 17, 20], numeric-comp = [15-17, 20], numeric-verb = [15]; [16]; [17]; [20], alphabetic = [Jon99, Wil93, BT86, Zil13], authoryear = Jones 99, Wilfred 93, Bohr, Turing 86, Ziltener 13.
%
% List of all: (you probably want a version of numeric, alphabetic or authoryear)
% numeric, numeric-comp, numeric-verb,
% nature (like numeric, but with '23.' instead of '[23]' in the bibliography),
% apa (does not work well with out with only 'year'; really needs a full date)
% alphabetic, alphabetic-verb,
% authoryear, authoryear-comp, authoryear-ibid, authoryear-icomp,
% authortitle, authortitle-comp, authortitle-ibid, authortitle-icomp, authortitle-terse, authortitle-tcomp, authortitle-ticomp,
% verbose, verbose-ibid, verbose-note, verbose-inote, verbose-trad1, verbose-trad2, verbose-trad3,
% reading, (draft, debug)
sorting = none, % 
% Choose how the bibliography is sorted.
% Options: nty, nyt, nyvt, anyt, anyvt, ynt, ydnt, none, (debug)
% Here n = name, t = title, y = year, v = volume, a = alphabetic label, ...d = ... descending
% So nty = sort by name, then title, then year.
backend = biber%
% This is the default, and should almost alway be kept as such. 'backend = bibtex' is legacy.
]{biblatex}

% If you use APA, you will need:
% \DeclareLanguageMapping{english}{english-apa}

% There exist related packages for specific styles like biblatex-chicago (Chicago manual of style citations) or biblatex-jura (German legal citations). You most likely won't need them or use them.

% For more information and a nice matrix with typesupport, see:
% https://en.wikibooks.org/wiki/LaTeX/Bibliography_Management#biblatex

% The source file for you bibliography:
\addbibresource{bibfile.bib}
% It's possible to add multiple bib files and separate them based on label (so to have two different references lists e.g. to seperate main sources from minor sources or books and theses from misc sources); see in 3.7 of the BibLaTeX documentation if you want to do stuff like that.

%%%%%%%%%%%%%%%%%%%%%%%%%%%%%%%%%%%%%%%%%%%%

%%%%% frontmatter/mainmatter/backmatter:
\newcommand\frontmatter{%
    \cleardoublepage
    \pagenumbering{roman}} %small Roman numbers

\newcommand\mainmatter{%
    \cleardoublepage
    \pagenumbering{arabic}} %normal numbers

\newcommand\backmatter{%
    \cleardoublepage %% double page style
    %\clearpage %% single page style
    \pagenumbering{Roman}} %capital Roman numbers
   




\title{\vspace{-65pt} Exercises LaTeX workshop}
\author{\TeX niCie\\
{\small (Thomas, Vincent \& Hanneke)}
%\\{\small (Vincent Kuhlmann)}
}
\date{7 February 2023}

\usepackage{minted}
\setminted{fontsize=\small, autogobble=true, linenos=false, frame=none}
% \setminted[tex]{fontsize=\small, autogobble=true, linenos=false, frame=none}
% \setminted[json]{fontsize=\small, autogobble=true, linenos=false, frame=none}

%\usemintedstyle{pastie}

\usepackage{wrapfig}
%\usepackage{cutwin}

\setcounter{secnumdepth}{0}

\begin{document}
\maketitle

% \CheckBox[]{aaa}

Remember the slides are available on \url{https://texnicie.nl}\quad Also, make sure you have at least these lines in your preamble:
\begin{minted}{tex}
        \usepackage[a4paper,margin=2.54cm]{geometry}
        \usepackage{amsmath,amssymb,amsthm}
        \usepackage{graphicx}
    \end{minted}

\section{Part 1: Text document}
% Zorg dat je steeds minstens deze packages hebt in je preamble:
% \begin{minted}{tex}
%         \usepackage[a4paper,margin=2.54cm]{geometry}
%         \usepackage{amsmath,amssymb,amsthm}
%         \usepackage[bookmarksnumbered]{hyperref}
%     \end{minted}
% \bigskip

\begin{exercise}[first document]
    Create a document with a title and a first line of text. Set the author to
    be your name. Change the paper size to a5paper, and set the margins to 1cm.
\end{exercise}

\begin{exercise}[emphasize]
    Emphasize some text by using \mintinline{tex}{\emph{your text}}. Put some another word
    or phrase in bold.
    
    % Extra: Can you find the difference between \mintinline{tex}{\emph} and \mintinline{tex}{\textit}?
\end{exercise}

\begin{exercise}[flushright]
    Find out what the \mintinline{tex}{\flushright} command does.
\end{exercise}

\begin{exercise}[headings]
    Create headings (section, subsection etc.), and create
    a table of contents for it. The table of contents should be on its own page.
\end{exercise}

\begin{exercise}[spacing]
    Let's make your document very\; s\,p\,a\,c\,i\,o\,u\,s. First, add the following
    lines to your preamble:
    \begin{minted}{tex}
    \usepackage{parskip}

    \setlength{\parskip}{20pt}
    \renewcommand{\baselinestretch}{1.5} 
    \end{minted}

    % They will make sure we are using paragraph spacing, with 20pt between paragraphs
    % (you can also use a value in centimeters if you prefer that). Additionally, the
    % line spacing is
    % set to 1.5.\footnote{If you are interested in how line spacing can be changed mid-document, see \url{https://tex.stackexchange.com/a/241121}}
    
    Check if this increases paragraph spacing and line spacing.

    Next, change the vertical margins to be 4 cm. Refer to the manual of the geometry
    package, or try what the following package options for geometry do: \mintinline{tex}{top=},
    \mintinline{tex}{bottom=}, \mintinline{tex}{vmargin=}.
    
\end{exercise}

\begin{exercise}[hyphenation]
    LaTeX can hyphenate words automatically. For this it needs the \texttt{babel} package,
    with package option \texttt{english} (i.e. \mintinline{tex}{\usepackage[english]{babel}}).
    Try to produce such hyphenation in your document.

    Hint: if you are having difficulty, increase the horizontal margin size, and change the paper
    size to A5 if you haven't already.
\end{exercise}

\begin{exercise}[special characters]
    Reproduce the following text:

    {\itshape When I woke up this morning, the temperature in my room was 13°C with 75\%
    humidity. I wrote down this data on my ``C:\textbackslash{}'' drive, in a file
    called temp\_room.txt. That morning the dollar-to-euro exchange rate was
    \$1.00 is €0.84.%
    % \footnote{This exchange rate was real in February 2021. How
    % times have changed. You don't have to reproduce this footnote, however you could with
    % \mintinline{tex}{\footnote{}}. Or just insert a superscript 1, if you really want that.}
    }

    Hints:
    \begin{itemize}
        \item You can use \mintinline{tex}{\textdegree} instead of pasting in a degree
        symbol, if you use \mintinline{tex}{\usepackage{gensymb}}.
        \item Look at the slide of typing special characters literally.
        \item Use \mintinline{tex}{\usepackage{lmodern}} for a nicer euro symbol.
        (You can enter a euro symbol directly in the code)
        \item For special characters it is often advisable to use \mintinline{tex}{\usepackage[utf8]{inputenc}}
        (which Overleaf includes by default). Then more characters can be typed in directly in code.
    \end{itemize}
\end{exercise}

\begin{exercise}[parskip]
    Add two paragraphs to your document, and observe the difference with \mintinline{tex}{\usepackage{parskip}}
    and without it. Which style do you prefer?
\end{exercise}

\begin{exercise}[manual spacing]
    Find out what the following commands do: \mintinline{tex}{\quad}, \mintinline{tex}{\qquad}, \mintinline{tex}{\hspace{2cm}},
    \mintinline{tex}{\;}, \mintinline{tex}{\!}, \mintinline{tex}{\vspace{2cm}}, \mintinline{tex}{\bigskip}.
\end{exercise}

\begin{exercise}[colors]
    Add package \mintinline{tex}{\usepackage{xcolor}}, produce the following text
    in red and orange colors:
    \textbf{\textcolor{orange}{Hi, I like the color \textcolor{red}{red}.}}
\end{exercise}

\section{Part 2: Formulas and figures}

\begin{exercise}
    \textit{Recreate the following expression in inline mode:}

$$\left(\frac{x^3}{3(x+1)^2}\right)^{\frac{1}{n}}$$

\end{exercise}
\begin{exercise}
\textit{Recreate the following proof by using align:}

    \bgroup\small
	The solution of $ax^2+bx+c=0$ where $a\neq 0$ is

	\begin{align}
		\frac{-b\pm \sqrt{d}}{2a}\text{ where }d=b^2-4ac
	\end{align}
	\begin{proof}
		We see that the equation is equivalent to
		\begin{align}
			ax^2+bx&=-c
			\intertext{Or equivalently}
			-\frac{c}{a}=x^2+\frac{b}{a}x&=x^2+2\frac{b}{2a}x
			\intertext{By adding $\left(\frac{b}{2a})\right)^2$ to both sides we get} 
			\left(\frac{b}{2a}\right)^2-\frac{c}{a}&=x^2+2\frac{b}{2a}+\left(\frac{b}{2a}\right)^2\\
			&=\left(x+\frac{b}{2a}\right)^2
			\intertext{By multiplying $4a^2$ to bot sides we get}
			b^2-4ac&=(2ax+b)^2
			\intertext{So}
			\pm \sqrt{b^2-4ac}&=2ax+b
			\intertext{And therefore}
			\frac{-b \pm \sqrt{b^2-4ac}}{2a}&=x
		\end{align}
	\end{proof}
    \egroup
\end{exercise}

\begin{exercise}[basic image]
    Find an image of your favourite animal species, and upload the image into your
    Overleaf document. First, use a direct \mintinline{tex}{\includegraphics{...}}
    with \mintinline{tex}{...} the name of the image. If this works, wrap
    a proper figure environment around it as seen in the slides.
\end{exercise}

\begin{exercise}[reference]
    Add a reference to a numbered equation and a figure in your text. Use the proper
    \LaTeX{} way of doing this, i.e. with \mintinline{tex}{\label{fig:cuteanimal}}
    and \mintinline{tex}{\ref{fig:cuteanimal}}. This ensures the numbers will stay
    correct.
\end{exercise}

\begin{exercise}[image trimming]
    You can crop an image from within \LaTeX{} using this command:
    \begin{minted}{tex}
        \includegraphics[width=0.9\linewidth,trim=10pt 10pt 10pt 10pt,clip]{example-image-a}
    \end{minted}
    Observe how changing the 4 numbers in the trim option (corresponding to left, bottom,
    right, top respecitvely) affects the cropping. Make sure you have added \mintinline{tex}{\usepackage{graphicx}}
    to your preamble!
\end{exercise}

% \begin{exercise}[PDF-viewer]
%     Ga naar het TeX-tabje in de activity bar links in VS~Code. Probeer de verschillende opties
%     onder `View LaTeX PDF'. Wat vind je het fijnste werken?
%     Verander de \texttt{latex-workshop.view.pdf.viewer} optie in de settings als je
%     een andere default wil.
% \end{exercise}

% \begin{exercise}[Inline math shortcut]
%     Stel een shortcut in voor het invoegen van inline math. Bijvoorbeeld door
%     het volgende toe te voegen aan je \texttt{keybindings.json}:
%     \begin{minted}{json}
%         {
%             "key": "ctrl+shift+m",
%             "when": "editorTextFocus && editorLangId == latex",
%             "command": "editor.action.insertSnippet",
%             "args": {
%                 "snippet": "\\$ ${1:} \\$$0"
%             }
%         },
%     \end{minted}
%     Check dat dit werkt.
% \end{exercise}

% \begin{exercise}[Errors en warnings]
%     Maak een error door een align met een witregel erin, en daarna een warning door
%     \mintinline{tex}{\label} twee keer te gebruiken met hetzelfde argument.
%     Waar zie je de errors en warnings in Visual Studio Code?
% \end{exercise}

% \begin{exercise}[LaTeX Workshop snippets]
%     Ga naar de volgende URL:
    
%     \url{https://github.com/James-Yu/LaTeX-Workshop/wiki/Snippets}.
    
%     Stel de \texttt{editor.suggest.snippetsPreventQuickSuggestions} in zoals aangegeven op de
%     pagina. Probeer vervolgens een figure, een section en een \mintinline{tex}{\textbf} te maken
%     met de default snippets en shortcuts die erop vermeld staan.
% \end{exercise}

% \begin{exercise}[Environment snippet]
%     Stel een snippet in voor het toevoegen van een environment. Kies als default environment
%     naam wat je denkt het meest te zullen gebruiken (bijvoorbeeld align).
% \end{exercise}

% \begin{exercise}[VS Code algemene shortcuts]
%     Ga naar \url{https://code.visualstudio.com/docs}, klik op `Keyboard Shortcut Reference Sheet'
%     en download de PDF voor jouw besturingssysteem.
%     Probeer wat shortcuts uit. Welke zouden voor jou handig kunnen zijn?
% \end{exercise}

% \begin{exercise}[Basisdocument snippet]
%     Maak een snippet die een basisdocument voor LaTeX voorziet, met alle \mintinline{tex}{\usepackage}'s
%     die je meestal nodig hebt.
% \end{exercise}

% \begin{exercise}[Python]
%     Als je Python kent, maak een Python bestand in VS Code, en zoek hoe je het kan
%     uitvoeren. Probeer ook de interactive console.
% \end{exercise}

% \pagebreak
% \section{Deel 2: Effici\"entie in LaTeX code}

% % Zorg dat je steeds minstens deze packages hebt in je preamble:
% % \begin{minted}{tex}
% %         \usepackage[a4paper,margin=2.54cm]{geometry}
% %         \usepackage{amsmath,amssymb,amsthm}
% %         \usepackage{graphicx}
% %         \usepackage{subcaption}
% %         \usepackage{booktabs}
% %         \usepackage[bookmarksnumbered]{hyperref}
% %     \end{minted}
% % \bigskip

% \begin{exercise}[Stelsel in matrix revisited]
% Stelsels lineaire vergelijkingen kunnen opgelost worden door ze te schrijven als een matrix en Gauss
% eliminatie toe te passen.
% Repliceer dit typische stelselmatrix:

% \begin{tabularx}{\textwidth}{Xp{0.7\textwidth}}
% \adjustbox{valign=t}{\small$\displaystyle
%     \left(\begin{array}{rrr|r}
%         2 & 1 & -1 & 8\\
%         -3 & -1 & 2 & -11\\
%         -2 & 1 & 2 & -3
%     \end{array}\right)
% $}&
% \parbox[t]{0.65\textwidth}{\small De eerste rij komt overeen met de vergelijking $ 2x+y-z=8 $.\\
% Getallenvoorbeeld van:\\\url{https://en.wikipedia.org/wiki/Gaussian_elimination}}
% \end{tabularx}

% Maak een environment hiervoor. Zorg dat het environment een argument heeft voor hoeveel
% kolommen er voor de verticale streep staan.

% Kan je dit een optioneel argument maken?
% \end{exercise}

% \begin{exercise}[Vector]
%     Definieer een commando die drie argumenten neemt, en er een kolommatrix van maakt.
% \end{exercise}

% \begin{exercise}[Commando \textbackslash input]
%     Kopieer het \texttt{.tex}-bestand van je vorige inleveropgave, en plaats de preamble ervan
%     in een ander bestand, dat je bijvoorbeeld \texttt{preamble.tex} noemt. Gebruik
%     \mintinline{tex}{\author{TeXniCie}
\date{\today}
\title{Manual for thesis}

%%% Overview of this file in order:
% Packages which don't need options
% Packages which have one or few options
% Geometry package
% Header/footer settings
% Theorem styles
% Enable/disable parindents
% Reference and bibliography settings
% Front/main/back-matter


%%%%%%%%%%%%%%%%%%%%%%%%%%%%%%%%%%%%%%%%%

\usepackage{
		%layout,		% Allow visualisation of all the margins
		subfiles,		% For separate main and sub documents
		graphicx,		% For image modifications and the figure environment
		amsmath,		% For the AMS math styles
		amssymb,		% The extended AMS math symbol list
		amsthm,			% For use of theorems (works together with thmtools)
		fancyhdr,		% For fancy headers and footers on pages
		%gensymb,		% For easy generic symbols (uniform use in math and text mode)
		%sidecap,		% For use of captions next to a float (figure, table, etc)
		subcaption,		% For easy subfigures in a plot (with nice captions)
		tikz,			% Difficult drawing of awesome vector plots
		%listings,		% For listing pieces of code in a nice and neat way
		multicol,		% For easy local multicolumn use
		color,			% For handy colour definitions (used cause of styling)
		%calc,			% To calculate stuff for the back-end
		%mdwlist,		% For customising lists
		thmtools,		% Lets you define your own theorem style (used for all the fancy theorems, definitions etc.)
		etoolbox,		% Allows adjustment of commands (used to reset the claim counter at the end of a proof).
		xspace,			% Makes latex not eat spaces after commands
        hyperref,		% Makes links, references, the Table of Contents, etc. clickable.
        mathtools,      % Extensible symbols, such as brackets, arrows, harpoons, etc.;
        inputenc,       % Allows to use things like ö instead of \"o in your text.
        wrapfig,        % wrap text around a smaller figure
        framed          % allows for boxes around your important equations or theorems
        }

%%%%%%%%%%%%%%%%%%%%%%%%%%%%%%%%%%%%%%%%%%

\usepackage[english]{babel} % Correct language setting, 'british', 'american'='english' or 'dutch'.
\usepackage[autostyle]{csquotes} % Fixes quotes to correspond to the babel language.
% Note the difference between ``quotes'' and ''quotes'' when using different languages.


%%%%%%%%%%%%%%%%%%%%%%%%%%%%%%%%%%%%%%%%%%

 \usepackage[margin=2.5cm]{geometry}
 % Change the shape of a page (custom margins etc.)
 % paper=a4paper slightly changes the style through the whole document.
    %%%% We set the margins for whole document here, except the titlepage. The titlepage uses special margins; see titlepage.tex.


%%%%%%%%%%%%%%%%%%%%%%%%%%%%%%%%%%%%%%%%%%

%%% This is about changing the headers and footers (i.e. Top and bottom of the page)

\pagestyle{fancy}% use fancyheaders with the bar on the top
\fancyhf{} % Clear the normal style
\fancyhead[L]{\bfseries\leftmark} %this places the section number and name in the top left
\fancyhead[R]{\bfseries\thepage}% this places the pagenumber in the top right
	

%%%%%%%%%%%%%%%%%%%%%%%%%%%%%%%%%%%%%%%%%%%%
%%%%		Theorem style

% The set-up is as follows, first you give the 'style' of your theorem. This determines whether it for instance is plain, or italic. Secondly you can give an option for the symbol on the end, normally it is nothing. But you could add some to increase the readability of your text. Finally you can use numberwithin to add the number of your section/theorem before your equations. This is useful if you want to keep the numbers of your equation in check (In this thesis there where over a 100) and keeps in order where the equations are.
%Finally you can use sibling to let different 'theorems' count together. Hence you will get Theorem 1 Definition 2 Claim 3, instead of Theorem 1 Definition 1 Claim 1. This is a matter of taste.

% Theorem definitions
\declaretheorem[style=definition,numberwithin=subsection]{definition} %If you want your theorems to be counted per section instead of subsection, then just remove the sub from the numberwithin
\declaretheorem[style=definition,qed=$\triangle$,sibling=definition]{example}% sibling says with what type of theorems you wan the numbering to count with.

\declaretheorem[style=plain,sibling=definition]{theorem}
\declaretheorem[style=plain,sibling=definition]{lemma}
\declaretheorem[style=plain,sibling=definition]{proposition}
\declaretheorem[style=plain,sibling=definition]{corollary}
\declaretheorem[style=definition]{claim}
\declaretheorem[style=definition,sibling=example]{remark}

\AtEndEnvironment{proof}{\setcounter{claim}{0}} % Sets the claim number to 0 after ending a proof

% You can make short-hands like these. 

\newcommand{\thm}[2]{\begin{theorem} #1 \begin{proof} #2 \end{proof} \end{theorem}}
\newcommand{\lm}[2]{\begin{lemma} #1 \begin{proof} #2 \end{proof} \end{lemma}}
\newcommand{\df}[1]{\begin{definition} #1 \end{definition}}
\newcommand{\clm}[1]{\begin{claim} #1 \end{claim}}


%%%%%%%%%%%%%%%%%%%%%%%%%%%%%%%%%%%%%%%%%%%%

%Let equation numbers be numbered by 1.1, 1.2, 2.1, etc where the first number is the section number. section can also be replaced by chapter when using book class.
\numberwithin{equation}{section}

%%%%%%%%%%%%%%%%%%%%%%%%%%%%%%%%%%%%%%%%%%%%

% Comment/uncomment the following to disable/enable parindents:
\setlength\parindent{0pt}

%%%%%%%%%%%%%%%%%%%%%%%%%%%%%%%%%%%%%%%%%%%%

%%%% Add the bibliography with some settings:
% package:
\usepackage[% Options
style = numeric-comp, % 
% Choose the style of your citations (and correspondingly your bibliography).
% Few examples:
% numeric = [15, 16, 17, 20], numeric-comp = [15-17, 20], numeric-verb = [15]; [16]; [17]; [20], alphabetic = [Jon99, Wil93, BT86, Zil13], authoryear = Jones 99, Wilfred 93, Bohr, Turing 86, Ziltener 13.
%
% List of all: (you probably want a version of numeric, alphabetic or authoryear)
% numeric, numeric-comp, numeric-verb,
% nature (like numeric, but with '23.' instead of '[23]' in the bibliography),
% apa (does not work well with out with only 'year'; really needs a full date)
% alphabetic, alphabetic-verb,
% authoryear, authoryear-comp, authoryear-ibid, authoryear-icomp,
% authortitle, authortitle-comp, authortitle-ibid, authortitle-icomp, authortitle-terse, authortitle-tcomp, authortitle-ticomp,
% verbose, verbose-ibid, verbose-note, verbose-inote, verbose-trad1, verbose-trad2, verbose-trad3,
% reading, (draft, debug)
sorting = none, % 
% Choose how the bibliography is sorted.
% Options: nty, nyt, nyvt, anyt, anyvt, ynt, ydnt, none, (debug)
% Here n = name, t = title, y = year, v = volume, a = alphabetic label, ...d = ... descending
% So nty = sort by name, then title, then year.
backend = biber%
% This is the default, and should almost alway be kept as such. 'backend = bibtex' is legacy.
]{biblatex}

% If you use APA, you will need:
% \DeclareLanguageMapping{english}{english-apa}

% There exist related packages for specific styles like biblatex-chicago (Chicago manual of style citations) or biblatex-jura (German legal citations). You most likely won't need them or use them.

% For more information and a nice matrix with typesupport, see:
% https://en.wikibooks.org/wiki/LaTeX/Bibliography_Management#biblatex

% The source file for you bibliography:
\addbibresource{bibfile.bib}
% It's possible to add multiple bib files and separate them based on label (so to have two different references lists e.g. to seperate main sources from minor sources or books and theses from misc sources); see in 3.7 of the BibLaTeX documentation if you want to do stuff like that.

%%%%%%%%%%%%%%%%%%%%%%%%%%%%%%%%%%%%%%%%%%%%

%%%%% frontmatter/mainmatter/backmatter:
\newcommand\frontmatter{%
    \cleardoublepage
    \pagenumbering{roman}} %small Roman numbers

\newcommand\mainmatter{%
    \cleardoublepage
    \pagenumbering{arabic}} %normal numbers

\newcommand\backmatter{%
    \cleardoublepage %% double page style
    %\clearpage %% single page style
    \pagenumbering{Roman}} %capital Roman numbers
   


} in je eigenlijke \texttt{.tex}-bestand.
%     Kan je het nog steeds compileren?
% \end{exercise}

% \begin{exercise}[Eigen documentclass]
%     Maak je eigen documentclass zoals aangegeven in de slides. Werkt het als je
%     in het kopie van je vorige inleveropgave de preamble vervangt door\newline
%     \mintinline{tex}{\documentclass{inleveropgave}}?
% \end{exercise}

% \begin{exercise}[aux-directory]\label{ex:auxDir}
%     In plaats van dat alle hulpbestanden zoals \texttt{.aux}, \texttt{.toc}, \texttt{.out}
%     je mapje onoverzichtelijk maken, is het mogelijk de locatie ervoor te veranderen naar een
%     ander mapje. Vraag Vincent als je benieuwd bent.

%     In VS Code, onder het TeX-tabje, gebruik `Clean up auxiliary files' onder `Build LaTeX project'.

%     Typ \mintinline{text}{"latex-workshop.latex.tools"} in je settings.json bestand, en gebruik
%     de auto-complete. Je krijgt een hele lijst met 'tools'. Voeg deze tool toe:
%     \begin{minted}{tex}
%         {
%             "name": "pdflatexDirs",
%             "command": "pdflatex",
%             "args": [
%                 "-synctex=1",
%                 "-interaction=nonstopmode",
%                 "-file-line-error",
%                 "-aux-directory=auxdir",
%                 "%DOC%"
%             ],
%             "env": {}
%         },
%     \end{minted}
%     Typ nu \mintinline{text}{"latex-workshop.latex.recipes"} in je settings.json bestand,
%     en gebruik weer auto-complete. Voeg bovenaan de recipes toe:
%     \begin{minted}{tex}
%         {
%             "name": "pdflatexDirs",
%             "tools": [
%                 "pdflatexDirs"
%             ]
%         },
%     \end{minted}
%     Sla op, en compileer je bestand. Je zou nu een nieuw mapje `auxdir' moeten zien, en
%     hierin staan al je auxiliary files.
% \end{exercise}

% \begin{exercise}[Snelle compilatie]
%     %Zie oefeningen PDF op texnicie.nl.

%     Maak een \texttt{.tex}-bestand en compileer het manueel met het \texttt{pdflatex}-commando
%     in je terminal.

%     Eenmaal dat is gelukt, voeg deze lijn bovenaan je \texttt{.tex}-bestand toe
%     \begin{minted}{text}
%         %&document_format
%         \documentclass{article}

%         ...
%     \end{minted}
%     en voer dit commando uit in je terminal:
%     \begin{minted}{text}
%         pdftex -ini -jobname="document" "&pdflatex" mylatexformat.ltx document.tex
%     \end{minted}

%     Als het goed is zie je nu een \texttt{document.fmt}-bestand. Dit is een cache van het moment
%     dat de preamble helemaal was ingeladen. Als je nu je \texttt{.tex}-bestand weer compileert
%     (kan ook via Visual Studio Code), zou dit veel sneller moeten gaan.

%     Maar let op! De cache kijkt niet of je preamble is veranderd, dus als je je preamble verandert
%     moet je je \texttt{document.fmt}-bestand verwijderen, en opnieuw maken.
%     Je kan het commando voor het maken van dit \texttt{.fmt}-bestand ook instellen in VS~Code.
%     Dit gaat analoog aan \autoref{ex:auxDir}, als
%     \begin{minted}{json}
%         {
%             "name": "mylatexformat",
%             "command": "pdftex",
%             "args": [
%                 "-ini",
%                 "-jobname=\"%DOC%_format\"",
%                 "&pdflatex",
%                 "mylatexformat.ltx",
%                 "%DOC%"
%             ],
%             "env": {}
%         },
%     \end{minted}

%     Let erop dat in je document de eerste regel \texttt{\%\&document\_format} is, en vervang \texttt{document\_format}
%     hier door de naam van je bestand zonder de \texttt{.tex}, en met \texttt{\_format} erachter.

%     Dit is een moeilijke opgave, en het kan zijn dat ik iets mis in de instructies. 
%     Vraag Vincent als het niet meteen lukt.
% \end{exercise}

% Exercises were created by members of the TeXniCie:
% \begin{verbatim}
%     Copyright (c) 2022-2023 Tim Weijers
%     Copyright (c) 2022-2023 Thomas van Maaren
%     Copyright (c) 2022-2023 Hanneke Schroten
%     Copyright (c) 2021-2023 Vincent Kuhlmann
% \end{verbatim}

\end{document}
