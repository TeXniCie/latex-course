\usepackage[utf8]{inputenc}
\usepackage[a4paper,margin=2.54cm,top=2cm]{geometry}

\usepackage{amsmath,amssymb,amsthm}
\usepackage{commath,mathtools}
\usepackage{parskip}
\usepackage{graphicx}
\usepackage{xcolor}
\usepackage{subcaption}
\usepackage{fancyhdr}
\usepackage{thmtools}
\usepackage{enumerate}
\usepackage{enumitem}
\usepackage{tabularx}
\usepackage[english]{babel}
\usepackage{adjustbox}
\usepackage{lmodern}
\usepackage[bookmarksnumbered]{hyperref}

% \pagestyle{fancy}
% \fancyhf{}
% \fancyhead[L]{\bfseries\leftmark}
% \fancyhead[R]{\bfseries\thepage}

\makeatletter

\newlength\uprightIndent
\setlength\uprightIndent{25pt}

% Using https://tex.stackexchange.com/questions/67242/a-theoremstyle-with-complete-indentation-using-amsthm
% and package documentation
\newtheoremstyle{uprightIndented}
{3pt}
{3pt}
{
\addtolength\@totalleftmargin\uprightIndent
\addtolength\linewidth{-\uprightIndent}
\parshape 1
%0pt \textwidth
\uprightIndent \dimexpr\textwidth-\uprightIndent\relax
}
{-\uprightIndent}% indent
{\bfseries}% header font
{.}
{\newline%.5em
}
{}

\newtheoremstyle{uprightIndentedNoNewline}
{2.3pt}
{2.3pt}
{
\addtolength\@totalleftmargin\uprightIndent
\addtolength\linewidth{-\uprightIndent}
\parshape 1
%0pt \textwidth
\uprightIndent \dimexpr\textwidth-\uprightIndent\relax
}
{-\uprightIndent}% indent
{\bfseries}% header font
{.}
{0.5em%3em
}
{}

\newlength\slantedIndent
\setlength\slantedIndent{15pt}

% Using https://tex.stackexchange.com/questions/67242/a-theoremstyle-with-complete-indentation-using-amsthm
% and package documentation
\newtheoremstyle{slantedIndented}
{3pt}
{3pt}
{
\addtolength\@totalleftmargin\slantedIndent
\addtolength\linewidth{-\slantedIndent}
\parshape 1
%0pt \textwidth
\slantedIndent \dimexpr\textwidth-\slantedIndent\relax
\itshape
}
{-\slantedIndent}% indent
{\bfseries}% header font
{.}
{\newline%.5em
}
{}

\newtheoremstyle{slantedIndentedNoNewline}
{3pt}
{3pt}
{
\addtolength\@totalleftmargin\slantedIndent
\addtolength\linewidth{-\slantedIndent}
\parshape 1
%0pt \textwidth
\slantedIndent \dimexpr\textwidth-\slantedIndent\relax
\itshape
}
{-\slantedIndent}% indent
{\bfseries}% header font
{.}
{1.5em
}
{}

\makeatother

\declaretheorem[style=uprightIndentedNoNewline,name={\ensuremath{\square} Oefening}]{exercise} % If you want your theorems to be counted per section instead of subsection, then just remove the sub from the numberwithin
\declaretheorem[style=uprightIndented,sibling=exercise,name=Oefening]{exerciseS}

\declaretheorem[style=slantedIndented,sibling=exercise]{exerciseSlanted}

\addto\extrasbritish{%
    \def\exercisename{Exercise}%
    \def\exerciseautorefname{Exercise}%
}


\addto\extrasenglish{%
    \def\exercisename{Exercise}%
    \def\exerciseautorefname{Exercise}%
}


\addto\extrasdutch{%
    \def\exercisename{Oefening}%
    \def\exerciseautorefname{Oefening}%
}

\newcommand{\R}{\mathbb{R}}
\newcommand{\Z}{\mathbb{Z}}
\newcommand{\N}{\mathbb{N}}
\newcommand{\Q}{\mathbb{Q}}
\newcommand{\C}{\mathbb{C}}
\renewcommand{\P}{\mathbb{P}}
\newcommand{\half}{\frac{1}{2}}



