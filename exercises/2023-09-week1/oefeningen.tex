\documentclass[a4paper, 11pt]{article}
\usepackage[top=3cm]{geometry}
\usepackage{parskip}
\usepackage[english]{babel}
\usepackage{amsmath, amssymb}
\usepackage{paralist}
\usepackage{minted}
\usepackage{color}
\usepackage{graphicx}

\title{Oefeningen bij \LaTeX cursus deel 1}
\date{\today}
\author{\TeX niCie}
\begin{document}
\maketitle
\begin{compactenum}
\item make a simple document of the article class with 1 paragraph of text. Use a5paper and a font size of 10pt.
\item emphasise some text by using the command \mintinline{tex}{\emph{your text}}.
\item find out what the \mintinline{tex}{\newline} command does.
\item LaTeX can hyphenate words automatically. For this it needs the \mintinline{tex}{babel} package, with optional argument \mintinline{tex}{english}. Import this package now.
\item Try the \mintinline{tex}{\quad} command in your document.
\item Add two new paragraphs to your document.
\item Now import the \mintinline{tex}{parskip} package and check the results.
\item Add headings to your paragraphs by using the \mintinline{tex}{section} and \mintinline{tex}{subsection} commands.
\item Make a title for your document, with an author and date.
\item Add the following text to your document, pay attention to special characters:
    \begin{compactitem}
         \item[] \{ \} \quad \textgreater \quad \& \quad 100\% \quad €40 \quad \$2 \quad C:\textbackslash Program\_Files 
    \end{compactitem}
\item Recreate the following text color effects. First import the \mintinline{tex}{xcolor} package by adding the line: \mintinline{tex}{\usepackage{xcolor}} to the preamble.
    \begin{compactitem}
        \item \textcolor{red}{red} \textcolor{green}{green} \textcolor{yellow}{yellow}
    \end{compactitem}
\item Write the pythagorean theorem in inline mode.
\item Use the \mintinline{tex}{\frac} and \mintinline{tex}{\blacksquare} command to write the following inline math: From \(z = \frac{x^3}{3(y+1)^2}\) and \(x>0\) it follows that \(z\) is positive \(\blacksquare\).
Make sure to import the \mintinline{tex}{amssymb} package first.
\item go to the wikipedia page about the dot-product (in english) and copy-paste the algebraic definition of the dot product into Overleaf in an align environment. You can copy all the mathematics on wikipedia directly into LaTeX.
\item Write this in an align environment $\left(\sum_{i=1}^\infty a_i\right)\left(\sum_{i=1}^\infty\frac{1}{a_i}\right)\geq n^2$
\item State the distributive laws for sets, aligned on the = sign (Ask a neighbour or internet if you don't know what it should say.)
\item State Wilsons theorem (Ask a neighbour or internet if you don't know the theorem.)
\item recreate the following math
\end{compactenum}
\includegraphics[width=0.9\textwidth]{plaatje.pdf}
\end{document}
