\documentclass[a4paper]{article}

\newif\ifishandout
\ishandoutfalse
%\ishandouttrue

\ifishandout
\documentclass[handout,aspectratio=32]{beamer}
\else
\documentclass[aspectratio=32]{beamer}
\fi

\usepackage[tabsize=4]{highlightlatex}

\setbeamertemplate{caption}[numbered]

\usecolortheme{rose}
%\useinnertheme[shadow]{rounded}
\useinnertheme{rounded}

\usetheme{Dresden}
\usecolortheme{dolphin}
\useoutertheme{miniframes}

\usepackage{subfiles}
\usepackage{amsmath,amssymb,amsthm,commath,mathtools}
\usepackage{esint}
\usepackage{enumerate}
\usepackage{subcaption}
\usepackage{graphicx}
\usepackage{xcolor}
\usepackage{adjustbox}
\usepackage{soul}
\usepackage{booktabs}
\usepackage{tabularx}
\usepackage{environ}
\usepackage[dutch]{babel}
\usepackage[utf8]{inputenc}
\usepackage{fancyvrb}
\usepackage{marvosym}
\usepackage{csquotes}
\usepackage[style=numeric]{biblatex}
\usepackage{textcomp}
%\usepackage{enumitem}
\usepackage{hyperref}
\usepackage{xkeyval}

\addbibresource{\subfix{assets/fakebib.bib}}

\DeclareMathOperator{\Image}{Image}

% Source: https://tex.stackexchange.com/questions/41683/why-is-it-that-coloring-in-soul-in-beamer-is-not-visible
\let\UL\ul
\makeatletter
\renewcommand\ul{
	\let\set@color\beamerorig@set@color
	\let\reset@color\beamerorig@reset@color
	\UL
}

\let\ST\st
\makeatletter
\def\st#1{
	\begingroup
	\let\set@color\beamerorig@set@color
	\let\reset@color\beamerorig@reset@color
	\def\SOUL@uleverysyllable{%
		\rlap{%
			%\color{red}
			\the\SOUL@syllable
			\SOUL@setkern\SOUL@charkern}%
		\SOUL@ulunderline{%
			\phantom{\the\SOUL@syllable}}%
	}%
	\ST{#1}%
	\endgroup
}
\makeatother
% https://tex.stackexchange.com/questions/71051/strikeout-in-different-color-appears-behind-letters-not-on-top-of-them

\setulcolor{red}
\setstcolor{red}

% Override if you want. Else you can delete it.
%\colorlet{curlyBrackets}{red!50!blue}
%\colorlet{squareBrackets}{blue!50!white}
%\colorlet{codeBackground}{gray!10!white}
%\colorlet{comment}{green!40!black}

\updatehighlight{
	name = default,
	color = {blue!90!black},
	add = {
		\knowncommand, \figref, \textcolor, \maketitle, \subsubsection,
		\textasciigrave, \textasciiacute, \tag, \middle, \mathbb, \abs,
		\mathcal, \middle, \dfrac, \subfile, \autoref, \eqref, \cites,
		\tableofcontents, \printbibliography, \fullcite, \parencite,
		\addbibresource, \DeclareLanguageMapping, \textcite, \intertext,
		\sum, \dif, \norm, \text, \dod, \dpd, \int, \partial,
		\DeclareMathOperator
	},
	name = structure,
	add = {
	},
}

\updatehighlight{
	name = greenDollar,
	style = {\itshape\color{green!70!black}},
	add = {
		% The dollar sign is provided an extra time just to
		% calm down TeXstudio's code highlighting.
		$, $
	},
	name = accentA,
	color = green!60!black,
	add = {
		\inAccA
	},
	%
	name = accentB,
	color = red!60!black,
	add = {
		\inAccB, \includegraphics
	},
	%
	name = accentC,
	color = orange!100!black,
	add = {
		\inAccC
	}
}

\lstset{tabsize=4}
\def\defaultgobble{8}

%\hllconfigure{
%	gobbletabs=3,
%}

\def\Zphantomconceal#1#2{%
	\only<#2->{\rlap{#1}}\phantom{#1}%
	%\only<#2->{#3}\unless\ifishandout\only<-#1>{\phantom{#3}}\fi
}

\def\phantomconceal#1#2{%
	\Zphantomconceal{#1}{#2}%
}

\newcommand\hideformula[2][2]{%
	%\hll|$| \only<2->{\hll|\\sqrt\{2\}|}\only<-1>{??} \hll|$|
	\hll|$| \phantomconceal{\hll|#2|}{#1} \hll|$|
}

\newcommand\hidelatex[2][2]{%
	\phantomconceal{\hll|#2|}{#1}
}%

\newcount\showcount

%\newcommand\showformula[2]{%
%	#1 & %
%	\expandafter\hideformula\expandafter[\the\showcount]{#2}%
%}
%
%\newcommand\showformula[2]{%
%	\global\showcount=\numexpr\showcount + 1\relax
%	\showformula*{#1}{#2}%
%}

\makeatletter

\def\showformula@i#1#2{%
	#1 & %
	\expandafter\hideformula\expandafter[\the\showcount]{#2}%
}

%\def\showformula{%
%	\@ifstar{%
%		\global\showcount=\numexpr\showcount + 1\relax
%		\showformula@i
%	}{%
%		\showformula@i
%	}%
%}

\def\showformula#1#2{
	#1 & \global\showcount=\numexpr\showcount + 1\relax
	\expandafter\hideformula\expandafter[\the\showcount]{#2}%
}

\def\showformulaa#1#2{
	#1 & %
	\expandafter\hideformula\expandafter[\the\showcount]{#2}%
}

\def\showlatex#1#2{
	#1 & \global\showcount=\numexpr\showcount + 1\relax
	\expandafter\hidelatex\expandafter[\the\showcount]{#2}%
}

\def\showlatexx#1#2{
	#1 & %
	\expandafter\hidelatex\expandafter[\the\showcount]{#2}%
}

\makeatother

\newlength{\naturalwidth}
\newlength{\minimumwidth}
\newbox\naturalsizebox
\newcommand{\atleastwidth}[2][2cm]{%
	\savebox\naturalsizebox{#2}%
	\settowidth\naturalwidth{#2}%
	\naturalwidth=\wd\naturalsizebox
	\minimumwidth=\dimexpr #1\relax
	\leavevmode%(\the\naturalwidth, \the\minimumwidth)%
	\ifdim\naturalwidth<\minimumwidth\relax
	\makebox[\minimumwidth][l]{\usebox{\naturalsizebox}}%
	\else
	\usebox{\naturalsizebox}%
	\fi
}

\newcommand{\atleastwidthr}[2][2cm]{%
	\savebox\naturalsizebox{#2}%
	\settowidth\naturalwidth{#2}%
	\naturalwidth=\wd\naturalsizebox
	\minimumwidth=\dimexpr #1\relax
	\leavevmode%(\the\naturalwidth, \the\minimumwidth)%
	\ifdim\naturalwidth<\minimumwidth\relax
	\makebox[\minimumwidth][r]{\usebox{\naturalsizebox}}%
	\else
	\usebox{\naturalsizebox}%
	\fi
}

\lstset{framexleftmargin=0.25em,xleftmargin=0.25em}

\NewEnviron{bluebox}{
	\begingroup
		\adjustbox{cfbox=blue!40!white 2pt 10pt,valign=t,bgcolor=blue!5!white}{%
			\begin{minipage}[t]{\dimexpr\linewidth-24pt\relax}
				\BODY
			\end{minipage}%
		}%
	\endgroup
}

\newcounter{maxrecentdisplay}
\setcounter{maxrecentdisplay}{27}

\newcounter{recentcount}
\setcounter{recentcount}{0}

\newcounter{recentskipremaining}

\def\vertlistsep{\hspace{2em}\textcolor{white!100!black}{\vrule width 0.5pt height 0.7\baselineskip\relax}\hspace{2em}}

\def\recentlist{}

%\newcommand{\addtorecentlist}[1]{%
%	\let\do\relax
%	\xdef\recentlist{\recentlist\do{#1}}%
%}

\newcommand{\addtorecentlist}[1]{%
	\bgroup
		\let\do\relax
		\expandafter\gdef\expandafter\recentlist\expandafter{\recentlist\do{#1}}%
		\addtocounter{recentcount}{1}%
	\egroup
	%
	%\xdef\recentlist{\recentlist\do{#1}}%
}

\newcommand{\clearrecentlist}{%
	\gdef\recentlist{}%
	\setcounter{recentcount}{0}%
}

\newif\ifisfirstrecentitem
\newcommand{\printrecentlist}{%
	\setcounter{recentskipremaining}{0}%
	\ifnum\value{recentcount}>\value{maxrecentdisplay}
		\setcounter{recentskipremaining}{\value{recentcount}-\value{maxrecentdisplay}}
	\fi
	%(\therecentskipremaining)
	%(\meaning\recentlist)
	\isfirstrecentitemtrue
	\def\do##1{%
		\ifnum\value{recentskipremaining}>0\relax
			\addtocounter{recentskipremaining}{-1}%
		\else		
			\unless\ifisfirstrecentitem
			\vertlistsep
			\fi
			\isfirstrecentitemfalse
			\textbf{##1}%
		\fi
	}%
	\recentlist
}

\newcommand{\recentpopfront}[1][1]{%
	\typeout{recentpopfront, before: \meaning\recentlist}
	\setcounter{recentskipremaining}{#1}%
	\let\origrecentlist\recentlist
	\clearrecentlist
	\def\do##1{%
		\ifnum\value{recentskipremaining}>0\relax
			\addtocounter{recentskipremaining}{-1}%
		\else		
			\addtorecentlist{##1}%
		\fi
	}%
	\origrecentlist
	\typeout{recentpopfront, after: \meaning\recentlist}
}

\newsavebox\printrecentbox
\savebox\printrecentbox{}
\newsavebox\scratchbox

% \AtBeginDocument{
% \setbox\scratchbox\printrecentbox
% }

\newcommand{\saveprintrecentbox}{%
	\bgroup
		\savebox\printrecentbox{\printrecentlist}%
		\global\setbox\printrecentbox\box\printrecentbox
	\egroup
	% \setbox\scratchbox\printrecentbox
	% \global\setbox\printrecentbox\scratchbox
	% \ifdim\wd\printrecentbox>0.9\textwidth
	% 	\savebox\printrecentbox{\adjustbox{right=0.9\textwidth}{\printrecentlist}}%
	% \else
	% 	\savebox\printrecentbox{\adjustbox{left=0.9\textwidth}{\printrecentlist}}%
	% \fi
}

\newcommand{\shrinkrecentbox}[1]{%
	{\loop
		%\clearrecentlist
		%\saveprintrecentbox
		%(SavedEmptyBox)
		%\iffalse


		\ifdim\wd\printrecentbox>\dimexpr #1\relax
		%
		\recentpopfront[1]%
		\saveprintrecentbox
	\repeat}%
}

% Based on miniframes code
\setbeamertemplate{headline}
{%
	\begin{beamercolorbox}[colsep=1.5pt]{upper separation line head}
	\end{beamercolorbox}
	\begin{beamercolorbox}{section in head/foot}
		\vskip2pt\insertnavigation{\paperwidth}\vskip2pt
	\end{beamercolorbox}%
	%
	\begin{beamercolorbox}[colsep=1.5pt]{middle separation line head}
	\end{beamercolorbox}
	\begin{beamercolorbox}[
		ht=2.5ex,
		dp=1.125ex,
		leftskip=.3cm,rightskip=.3cm plus1fil
		]{subsection in head/foot}
		\usebeamerfont{subsection in head/foot}%\insertsubsectionhead
		% \savebox\printrecentbox{\printrecentlist}%
		% \ifdim\wd\printrecentbox>0.9\textwidth
		% 	\adjustbox{right=0.9\textwidth}{\printrecentlist}%
		% \else
		% 	\adjustbox{left=0.9\textwidth}{\printrecentlist}%
		% \fi
		\saveprintrecentbox
		\ifdim\wd\printrecentbox>0.9\textwidth
			%(Shrinking box)
			%\PackageError{debug}{Width is \the\wd\printrecentbox}{}%
			\shrinkrecentbox{0.6\textwidth}%
		\else
			%(Not shrinking box)
		\fi
		\usebox\printrecentbox
		%\textbullet\ Hey
	\end{beamercolorbox}%
	%
	\begin{beamercolorbox}[colsep=1.5pt]{lower separation line head}
	\end{beamercolorbox}
}

\makeatletter

\NewEnviron{colC}[2][]{%
	\def\setpadd{}%
	\if\relax #1\relax
	\else
		%\def\setpadd{padding={0pt {\dimexpr ((#1)-\height)\relax} {0pt} {0pt}}}%
		\def\setpadd{%
			set depth={\dimexpr (#1)-\height\relax}%
		}
	\fi
	% \def\setparboxargs{}%
	% \if\relax #1\relax
	% \else
	% 	\def\setparboxargs{[t][\dimexpr #1\relax][]}%
	% \fi
	%
	\expandafter\adjustbox\expandafter{\setpadd,
		%margin=0pt,padding=0pt,
	%padding={0pt {\dimexpr (0.4\textheight-\height)/2\relax} {0pt} {\dimexpr (0.4\textheight-\height)/2\relax}},
		fbox=1pt 0pt 0pt,
		valign=M
	}%
	{%
		\parbox{\dimexpr #2-2pt\relax}{%
			\BODY
		}%
	}%
}

\NewEnviron{colT}[2][]{%
	\def\setpadd{}%
	\if\relax #1\relax
	\else
		\def\setpadd{%
			set depth={\dimexpr (#1)-\height\relax}%
		}%
	\fi
	%
	\expandafter\adjustbox\expandafter{\setpadd,
		fbox=1pt 0pt 0pt,
		valign=T
	}%
	{%
		\parbox{\dimexpr #2-2pt\relax}{%
			\BODY
		}%
	}%
}

\makeatother

\newlength\atleastlength


\newenvironment{noindentlist}{
	\begin{list}{\textbullet}{
		\leftmargin=0pt\relax
		\itemindent=0pt\relax
		\setlength{\itemsep}{2pt}
	}
}{
	\end{list}
}




\title{\vspace{-65pt} Oefeningen \LaTeX-cursus Week 2}
% \author{\TeX niCie\\{\small (Vincent Kuhlmann)}}
\author{\TeX niCie}
\date{5 oktober 2023}

\usepackage{minted}
\setminted[tex]{fontsize=\small, autogobble=true, linenos=false, frame=none}

\usemintedstyle{pastie}

\usepackage{wrapfig}
\usepackage{cutwin}

\setcounter{secnumdepth}{0}

\begin{document}
\maketitle

% \CheckBox[]{aaa}

\section{Deel 1}
Zorg dat je steeds minstens deze packages hebt in je preamble:
\begin{minted}{tex}
        \usepackage[a4paper,margin=2.54cm]{geometry}
        \usepackage{amsmath,amssymb,amsthm}
        \usepacakge{commath}
    \end{minted}
\bigskip


\begin{exercise}
    Maak de volgende formule na in \LaTeX, gebruik \emph{geen} \mintinline{tex}{\frac}, maar \mintinline{tex}{\dod}:
    \begin{align*}
        \int_{0}^{\pi/2}\cos(x)\dif x
        =
        \int_{0}^{\pi/2}\dod{\sin(x)}{x}\dif x
        =
        \left.\left[\sin(x)\right]\right|_{x=0}^{x=\pi/2}
        =
        1
    \end{align*}
\end{exercise}

\begin{exercise}
    Maak de volgende formules na in \LaTeX:
    \begin{align*}
        e^{x} = \lim_{N\to\infty}\sum_{n=0}^{N}\frac{x^n}{n!}
        \hspace{2cm}
        \lim_{x\uparrow 4}\lfloor x\rfloor = 3
        \hspace{2cm}
        \left\{2x_{*}\mid x_{*}\in \mathbb{Z}_{\geq 0}\right\}
    \end{align*}
\end{exercise}



% \begin{exercise}[TOC]
%     Voeg een paar \mintinline{tex}{\section}'s toe aan je bestand, en een
%     table of contents op een aparte pagina.

%     Het \mintinline{tex}{\section} commando laat een optioneel argument toe.
%     Voeg een section ermee toe, bijvoorbeeld
%     \mintinline{tex}{\section[Intro]{Introductie}} en kijk wat er gebeurt in je table of
%     contents.
% \end{exercise}

% \begin{exercise}[PDF TOC]
%     Voeg \mintinline{tex}{\usepackage[bookmarksnumbered]{hyperref}} toe aan
%     je preamble. Download je document als PDF en open het. Kijk of je de
%     table of contents van je PDF-lezer kan vinden.
%     Wat gebeurt er als je \mintinline{tex}{bookmarksopen} toevoegt als option voor hyperref?
% \end{exercise}

% \begin{exercise}[URL's]
%     Voeg de volgende link toe aan je bestand:\\
%     \nolinkurl{https://en.wikipedia.org/wiki/Electromagnetic_tensor}
%     \begin{enumerate}[label=\alph*)]
%         \item Wat gebeurt er als je de link direct in je code plakt?
%               Kan je die foutmelding fixen?
%         \item Plak nu dezelfde link in het argument van het \mintinline{tex}{\url}-commando
%               van hyperref: \mintinline{tex}{\url{...}}. Heb je dezelfde fix nog nodig?
%         \item Wat gebeurt er als je de \texttt{https://} weglaat?
%     \end{enumerate}
% \end{exercise}


\begin{exercise}[Geometry]\label{ex:aaaa}
    Maak een A6-document in landscape met voorbeeldtekst van \nolinkurl{lipsum.com}. Zet de
    horizontale marges op $ 2\text{ cm} $ en vertical marges op $ 3\text{ cm} $.

    Hint: De volgende opties van geometry kunnen van pas komen: left, right, top, bottom,
    vmargin, hmargin, landscape, a6paper.\\
    Documentatie over gebruik van geometry package: \url{https://ctan.org/pkg/geometry}
\end{exercise}

\begin{exercise}[\textbackslash eqref]
    De amsmath package definieert het commando \mintinline{tex}{\eqref{...}}.
    Voeg een genummerde vergelijking toe aan je document, met een label, en
    kijk wat het verschil is tussen \mintinline{tex}{\ref{...}}
    en \mintinline{tex}{\eqref{...}}.
\end{exercise}

\begin{exercise}[Stelling met bewijs]
    Voeg een stelling met bewijs (environment proof) toe in je bestand.
    %  voor je favoriete stelling of bewijs.
\end{exercise}

% \begin{exercise}[Labels]
%     Wat gebeurt er als je aan een niet-bestaande label refereert?
% \end{exercise}

% \begin{exercise}[Definitie]
%     Voeg een `Definitie' toe aan je bestand, en refereer eraan in je bestand.
% \end{exercise}

\begin{exercise}[\textbackslash theoremstyle]\label{ex:theoremStyle}
Cre\"eer een nieuw bestand met de template van\\
\href{https://vkuhlmann.com/latex/example}{\nolinkurl{vkuhlmann.com/latex/example}}
(zet de `Include Theorem, Lemma etc.' aan). Wat is het verschil in stijl tussen
\mintinline{tex}{\begin{theorem}},
\mintinline{tex}{\begin{definition}} en \mintinline{tex}{\begin{remark}}?
Probeer deze stijlen nu te veranderen door \mintinline{tex}{\theoremstyle{...}} commando's
toe te voegen, te verplaatsen en/of te verwijderen.
\end{exercise}

\begin{exercise}[Theorem nummering]%[Theorem numberwithin]
    %Probeer uit te vinden welk effect elk van de volgende codewijzigingen hebben:
    Welk effect heeft elk van de volgende codewijzigingen?\\
    (gebruik hetzelfde bestand als bij \autoref{ex:theoremStyle})
    \begin{enumerate}[label=\alph*)]
        \item \begin{minted}{tex}
            \newtheorem{theorem}{Theorem}[section] --> \newtheorem{theorem}{Theorem}
        \end{minted}
        \item \begin{minted}{tex}
            \newtheorem{lemma}[theorem]{Lemma} --> \newtheorem{lemma}{Lemma}
        \end{minted}
    \end{enumerate}
\end{exercise}

\newpage
\section{Deel 2}

Zorg dat je steeds minstens deze packages hebt in je preamble:
\begin{minted}{tex}
        \usepackage[a4paper,margin=2.54cm]{geometry}
        \usepackage{amsmath,amssymb,amsthm}
        \usepackage{graphicx}
        \usepackage{subcaption}
        \usepackage{booktabs}
    \end{minted}
\bigskip

% \begin{exercise}[Figure]
%     Is het mogelijk een figure environment te maken zonder \mintinline{tex}{\includegraphics}?
%     Kan je in plaats ervan tekst, een inline formule of een tabel hebben?
% \end{exercise}

\begin{exercise}
    Wat zijn `Table of contents', `Proof', `Table' en `Figure' in het Spaans?
    Zoek het uit in \LaTeX, zonder een vertalingsapp te gebruiken!

    Hint: \mintinline{tex}{\usepackage[spanish]{babel}}
\end{exercise}

\begin{exercise}
    Ga naar \url{https://opendata.cbs.nl}, klik op
    `Naar thema's', en klik tot je een leuke tabel krijgt.
    In plaats van deze tabel na te maken in \LaTeX{} (veel werk!),
    maken we een screenshot van de tabel.

    Terug in je document, maak een table environment, en gebruik
    een \mintinline{tex}{\includegraphics}
    om de screenshot in te laden. 
    % Maak een table environment met daarin een afbeelding i.p.v. tabel.
    De caption moet nog steeds `Table 1' zeggen.
\end{exercise}

% \begin{exercise}[Figuurplaatsing]
%     Cre\"eer een scenario waarbij \LaTeX{} je figuurplaatsingsadvies niet opvolgt.
% \end{exercise}

% \begin{exercise}[Matrix]
%     Maak een matrix met een verticale streep langs beide kanten i.p.v. haakjes, zoals de
%     notatie voor determinant van een matrix. Kan je vinden welke environment (eindigend op
%     matrix) dit al standaard doet?
% \end{exercise}

% \begin{tabular}{p{0.4\textwidth}l}
%     \begin{minipage}{\linewidth}
\begin{exercise}[Stelsel in matrix]
    % \begin{cutout}{2}{20pt}{\dimexpr\linewidth-2.5cm\relax}{6}
    %     AAA
    %   \end{cutout}
Stelsels lineaire vergelijkingen kunnen opgelost worden door ze te schrijven als een matrix en Gauss
eliminatie toe te passen.
% \begin{cutout}{3}{50pt}{\dimexpr\linewidth-50pt\relax}{6}
%     AAA
% \end{cutout}
%Een vergelijking $ 2x+y-z = 8 $ wordt dan de rij $ (2, 1, -1, 8) $.
%Dat het laatste getal geen eigenlijke variabele is, maken we in de wiskunde soms duidelijk door
%de extra kolom af te bakenen met een verticale streep.
Repliceer dit typische stelselmatrix:

\begin{tabularx}{\textwidth}{Xp{0.7\textwidth}}
\adjustbox{valign=t}{\small$\displaystyle
    \left(\begin{array}{rrr|r}
        2 & 1 & -1 & 8\\
        -3 & -1 & 2 & -11\\
        -2 & 1 & 2 & -3
    \end{array}\right)
$}&
\parbox[t]{0.65\textwidth}{\small De eerste rij komt overeen met de vergelijking $ 2x+y-z=8 $.\\
Getallenvoorbeeld van:\\\url{https://en.wikipedia.org/wiki/Gaussian_elimination}}
\end{tabularx}
% \end{exercise}
% \end{minipage}
% &
% {\small$\displaystyle
%     \left(\begin{array}{rrr|r}
%         2 & 1 & -1 & 8\\
%         -3 & -1 & 2 & 11\\
%         -2 & 1 & 2 & -3
%     \end{array}\right)
% $}

Hint: in wiskundemodus heb je de \mintinline{tex}{array} environment, die
werkt net zoals \mintinline{tex}{tabular}.
\end{exercise}

% \begin{exercise}[Align]
%     Hoe gedraagt het align-environment zich als je meer dan twee `kolommen' hebt?
% \end{exercise}

\begin{exercise}[Wiskunde in tabellen]
    Maak een simpele tabel met wat woorden, nummers en wiskundige symbolen erin (bijvoorbeeld $ \sqrt{2} $).
\end{exercise}

\begin{exercise}[Kolomscheidingen]
    Wat gebeurt er als een regel te veel kolommen heeft? En wat als het te
    weinig kolommen heeft?
\end{exercise}

\begin{exercise}[Subfigure]
    Maak een figure met veel subfigures erin, gebaseerd op de code in de slides. Kijk wat de verschillende parameters
    doen. Wat doet de \mintinline{tex}{0.45\textwidth}? Wat doet de \mintinline{tex}{[b]}?%
    \footnote{Hint: Vervang de [b] door [t] of [c] en geef de afbeeldingen in de subfigures
        ongelijke hoogtes.}
\end{exercise}

\begin{exercise}[Alignering]
    Zoek op wat de mogelijke aligneringen zijn voor een kolom in een tabular
    en probeer ze uit.
\end{exercise}

\begin{exercise}[Booktabs]
    Maak een tabel waarbij je \mintinline{tex}{\toprule}, \mintinline{tex}{\midrule}
    en \mintinline{tex}{\bottom} van booktabs gebruikt (zie slides), om
    een goed uitziende tabel te krijgen. Je kan \mintinline{tex}{\cmidrule} gebruiken om een
    gedeeltelijke horizontale lijn te krijgen.
\end{exercise}

\begin{exercise}[Excellent]
    Gebruik \mintinline{tex}{\multicolumn} om op een rij twee kolommen samen te voegen
    (zoek op hoe het commando werkt, of deduceer het van de slides). Als je
    wil kan je ook het \mintinline{tex}{\multirow}-commando van het package multirow
    uitproberen.
\end{exercise}

% \begin{exercise}[\textbackslash autoref]
%     Waarin verschilt het commando \mintinline{tex}{\autoref} (gedefinieerd door
%     \mintinline{tex}{hyperref}) van het simpele \mintinline{tex}{\ref}?
% \end{exercise}

% \begin{exercise}[Babel]
%     Voeg een table of contents toe, een \mintinline{tex}{\autoref} referentie
%     naar een vergelijking, en een figuur. Kijk welke automatische termen
%     babel ervoor geeft in verschillende talen.
% \end{exercise}

\section{Extra oefeningen} %\url{https://vkuhlmann.com/latex/exercises/2022-09-cursus/Week2_Vincent/uitbreiding}

Ga naar \href{https://vkuhlmann.com/go/d98d48}{\nolinkurl{vkuhlmann.com/go/d98d48}}
voor een extra set uitdagende oefeningen :)

% \section{Uitbreiding}

% \begin{exercise}[tabularx]
%     In de slides van tabellen is er een code voorbeeld die tabularx gebruikt. Wat zijn de verschillen
%     met de slide ervoor? Wat kan tabularx dus doen?
% \end{exercise}

% \begin{exercise}[Referentie naar subfigure]
%     Kan je een \mintinline{tex}{\label{...}} toevoegen aan de caption van een subfigure?
%     Hoe ziet een referentie daaraan eruit?
% \end{exercise}

% % \begin{exercise}[Subfigure nesting]
% %     Plaats een subfigure binnen een andere subfigure. Hoe ziet dit eruit?
% %     Hoe vreselijk voelt dat?
% % \end{exercise}

% \begin{exercise}[\textbackslash hfill]
%     Maak een figure met kleine subfigures erin. Wat gebeurt er als je
%     \mintinline{tex}{\hfill} toevoegt tussen de subfigures?
% \end{exercise}

% \begin{exercise}[\textbackslash texorpdfstring]
%     Wat doet het command \mintinline{tex}{\texorpdfstring{}{}} van het hyperref package?
%     Zoek het op in de documentatie van de package.
% \end{exercise}

% \begin{exercise}[pageref]

% \end{exercise}

% \begin{exercise}
%     Voeg een aantal alinea's van \href{https://lipsum.com}{lipsum.com} toe aan je bestand, en spreidt
%     ze uit over meerdere pagina's:
%     \begin{itemize}
%         \item Pagina 1: Landscape A5-papier met marges $ 2\text{cm} $
%         \item Pagina 2: Papier van dimensies $ 100\text{mm}\times 100\text{mm} $
%               met marges $ 1\text{cm} $ en $ 2\text{cm} $ langs de onderkant
%         \item Pagina 3: Portrait A6-papier met marges $ 0\text{cm} $
%     \end{itemize}

%     Hint: \url{https://tex.stackexchange.com/a/528245/242407}

%     Opmerking: de pagina grootte van A6-papier is 105mm:148mm
% \end{exercise}
\end{document}