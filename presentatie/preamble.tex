\newif\ifishandout
\ishandoutfalse

\ifishandout
\documentclass[handout,aspectratio=32]{beamer}
\else
\documentclass[aspectratio=32]{beamer}
\fi

\usepackage[tabsize=4]{highlightlatex}

\setbeamertemplate{caption}[numbered]

\usetheme{Dresden}
\usecolortheme{dolphin}
\useoutertheme{miniframes}

\usepackage{subfiles}
\usepackage{amsmath,amssymb,amsthm,commath,mathtools}
\usepackage{enumerate}
\usepackage{subcaption}
\usepackage{graphicx}
\usepackage{xcolor}
\usepackage{adjustbox}
\usepackage{soul}
\usepackage{booktabs}
\usepackage{tabularx}
\usepackage{environ}
\usepackage[dutch]{babel}
\usepackage[utf8]{inputenc}
\usepackage{fancyvrb}
\usepackage{hyperref}

% Source: https://tex.stackexchange.com/questions/41683/why-is-it-that-coloring-in-soul-in-beamer-is-not-visible
\let\UL\ul
\makeatletter
\renewcommand\ul{
	\let\set@color\beamerorig@set@color
	\let\reset@color\beamerorig@reset@color
	\UL
}

\let\ST\st
\makeatletter
\def\st#1{
	\begingroup
	\let\set@color\beamerorig@set@color
	\let\reset@color\beamerorig@reset@color
	\def\SOUL@uleverysyllable{%
		\rlap{%
			%\color{red}
			\the\SOUL@syllable
			\SOUL@setkern\SOUL@charkern}%
		\SOUL@ulunderline{%
			\phantom{\the\SOUL@syllable}}%
	}%
	\ST{#1}%
	\endgroup
}
\makeatother
% https://tex.stackexchange.com/questions/71051/strikeout-in-different-color-appears-behind-letters-not-on-top-of-them

\setulcolor{red}
\setstcolor{red}

% Override if you want. Else you can delete it.
%\colorlet{curlyBrackets}{red!50!blue}
%\colorlet{squareBrackets}{blue!50!white}
%\colorlet{codeBackground}{gray!10!white}
%\colorlet{comment}{green!40!black}

\updatehighlight{
	name = default,
	color = {blue!90!black},
	add = {
		\knowncommand, \figref, \textcolor, \maketitle, \subsubsection
	},
	name = structure,
	add = {
	},
}

\updatehighlight{
	name = greenDollar,
	style = {\itshape\color{green!70!black}},
	add = {
		% The dollar sign is provided an extra time just to
		% calm down TeXstudio's code highlighting.
		$, $
	},
	name = accentA,
	color = green!60!black,
	add = {
		\inAccA
	},
	%
	name = accentB,
	color = red!60!black,
	add = {
		\inAccB, \includegraphics
	},
	%
	name = accentC,
	color = orange!100!black,
	add = {
		\inAccC
	}
}

\lstset{tabsize=4}
\def\defaultgobble{8}

%\hllconfigure{
%	gobbletabs=3,
%}

\def\Zphantomconceal#1#2{%
	\only<#2->{\rlap{#1}}\phantom{#1}%
	%\only<#2->{#3}\unless\ifishandout\only<-#1>{\phantom{#3}}\fi
}

\def\phantomconceal#1#2{%
	\Zphantomconceal{#1}{#2}%
}

\newcommand\hideformula[2][2]{%
	%\hll|$| \only<2->{\hll|\\sqrt\{2\}|}\only<-1>{??} \hll|$|
	\hll|$| \phantomconceal{\hll|#2|}{#1} \hll|$|
}

\newcommand\hidelatex[2][2]{%
	\phantomconceal{\hll|#2|}{#1}
}%

\newcount\showcount

%\newcommand\showformula[2]{%
%	#1 & %
%	\expandafter\hideformula\expandafter[\the\showcount]{#2}%
%}
%
%\newcommand\showformula[2]{%
%	\global\showcount=\numexpr\showcount + 1\relax
%	\showformula*{#1}{#2}%
%}

\makeatletter

\def\showformula@i#1#2{%
	#1 & %
	\expandafter\hideformula\expandafter[\the\showcount]{#2}%
}

%\def\showformula{%
%	\@ifstar{%
%		\global\showcount=\numexpr\showcount + 1\relax
%		\showformula@i
%	}{%
%		\showformula@i
%	}%
%}

\def\showformula#1#2{
	#1 & \global\showcount=\numexpr\showcount + 1\relax
	\expandafter\hideformula\expandafter[\the\showcount]{#2}%
}

\def\showformulaa#1#2{
	#1 & %
	\expandafter\hideformula\expandafter[\the\showcount]{#2}%
}

\def\showlatex#1#2{
	#1 & \global\showcount=\numexpr\showcount + 1\relax
	\expandafter\hidelatex\expandafter[\the\showcount]{#2}%
}

\def\showlatexx#1#2{
	#1 & %
	\expandafter\hidelatex\expandafter[\the\showcount]{#2}%
}

\makeatother

\newlength{\naturalwidth}
\newlength{\minimumwidth}
\newbox\naturalsizebox
\newcommand{\atleastwidth}[2][2cm]{%
	\savebox\naturalsizebox{#2}%
	\settowidth\naturalwidth{#2}%
	\naturalwidth=\wd\naturalsizebox
	\minimumwidth=\dimexpr #1\relax
	\leavevmode%(\the\naturalwidth, \the\minimumwidth)%
	\ifdim\naturalwidth<\minimumwidth\relax
	\makebox[\minimumwidth][l]{\usebox{\naturalsizebox}}%
	\else
	\usebox{\naturalsizebox}%
	\fi
}

\newcommand{\atleastwidthr}[2][2cm]{%
	\savebox\naturalsizebox{#2}%
	\settowidth\naturalwidth{#2}%
	\naturalwidth=\wd\naturalsizebox
	\minimumwidth=\dimexpr #1\relax
	\leavevmode%(\the\naturalwidth, \the\minimumwidth)%
	\ifdim\naturalwidth<\minimumwidth\relax
	\makebox[\minimumwidth][r]{\usebox{\naturalsizebox}}%
	\else
	\usebox{\naturalsizebox}%
	\fi
}

\lstset{framexleftmargin=0.25em,xleftmargin=0.25em}
