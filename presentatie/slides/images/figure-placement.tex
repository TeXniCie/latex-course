\addtorecentlist{htbp}

\begin{frame}
	\frametitle{\lang,Figure placement,Figuurplaatsing,}

	% Een figuur wordt ergens verderop geplaatst. Soms is het dus handig om een figuur
	% een paragraaf of twee the plaatsen voor de plek waar je hem gebruikt!
	
	\begin{itemize}
		\item h \textsc{(here)}: \lang,Figure can come here.,Figuur mag hier.,
		\item t \textsc{(top)}: \lang,Figure can come at the top of the page.,%
			Figuur mag bovenaan een pagina.,
		\item b \textsc{(bottom)}: \lang,Figure can come at the bottom of the
			page,Figuur mag onderaan een pagina.,
		\item p \textsc{(page)}: \lang,Figure can come on a special page for figures.,%
			Figuur mag op aparte pagina voor figuren.,
		\item !: \lang,Override internal parameters for floats.,Override interne parameters voor floats.,
		\item H \textsc{(here)}: \lang{No floating, always here.}
		{Geen floating, altijd hier.} (\hll|\\usepackage\{float\}|)
	\end{itemize}

	\medskip
	% \lang
	% {Figure appearing too late? Try placing \hll|figure| to a point earlier in the code.}
	% {Te laat in output? Verplaats \hll|figure| naar voren in je bestand.}

	\uncover<1->{\lang,When working with images,Wanneer je werkt met afbeeldingen,:
	\hll|\\usepackage\{graphicx\}|}
\end{frame}
