
\begin{frame}[fragile]{Commando's}
    \begin{columns}
        \begin{column}{0.5\textwidth}
            \begin{minted}[fontsize=\scriptsize]{tex}
                \newcommand\fp{$ \pi/2 $-faseplaatje}
                \newcommand\co{CO$_2$}
        
                \begin{document}
                    Benodigdheden: laser, camera, lenzen, \fp.

                    Het \fp zorgt voor ...

                    Het \fp{} zorgt voor ...
        
                    En nu iets anders: \co. Dit zit in onze
                    atmosfeer.
                \end{document}
            \end{minted}
        \end{column}
        \begin{column}{0.5\textwidth}
            \begin{demobox}\small\setlength\parskip{5pt}
                \def\fp{$ \pi/2 $-faseplaatje}
                \def\co{CO$_2$}
    
                Benodigdheden: laser, camera, lenzen, \fp.

                Het \fp zorgt voor ...

                Het \fp{} zorgt voor ...
    
                En nu iets anders: \co. Dit zit in onze
                atmosfeer.
            \end{demobox}
        \end{column}
    \end{columns}
\end{frame}

\let\exampleTerm\somethingundefined
\newcommand\exampleTerm[1]{\textcolor{blue}{\textit{#1}}}

\begin{frame}[fragile]{Commando's}
    \begin{columns}
        \begin{column}{0.5\textwidth}
            \begin{minted}[fontsize=\scriptsize]{tex}
                \newcommand\term[1]{\textcolor{blue}{\textit{#1}}}
        
                \begin{document}
                    We noemen een groep \term{abels} of
                    \term{commutatief} als voor elk
                    paar elementen van de groep
                    $ a, b $ er is $ a\cdot b = b\cdot a $.
                \end{document}
            \end{minted}
        \end{column}
        \begin{column}{0.5\textwidth}
            \begin{demobox}\small\setlength\parskip{5pt}        
                We noemen een groep \exampleTerm{abels} of
                \exampleTerm{commutatief} als
                voor elk paar elementen van de groep $ a, b $
                er is $ a\cdot b = b\cdot a $.
            \end{demobox}
        \end{column}
    \end{columns}    
\end{frame}

\let\diag\somethingundefined
% \newcommand\diag[1]{\begin{pmatrix}
%     #1 & 0\\
%     0 & #1
% \end{pmatrix}}
\newcommand\diag[2]{\begin{pmatrix}
    #2 & #1\\
    #1 & #2
\end{pmatrix}}

\begin{frame}[fragile]{Commando's}
    \begin{columns}
        \begin{column}{0.5\textwidth}
            \begin{minted}[fontsize=\scriptsize,escapeinside=||]{tex}
                \newcommand\diag[2]{\begin{pmatrix}
                    #2 & #1\\
                    #1 & #2
                \end{pmatrix}}
        
                \begin{document}
                    De identiteitsmatrix is $\diag{0}{1}$.
                    We zien
                    \begin{align*}
                        2\cdot\diag{0}{1} = \diag{0}{2}.
                    \end{align*}

                    Verder
                    \begin{align*}
                        \diag{5}{0} + \diag{2}{0} = \diag{7}{0}.
                    \end{align*}
                \end{document}
            \end{minted}
        \end{column}
        \begin{column}{0.5\textwidth}
            \begin{demobox}\small\setlength\parskip{5pt}    
                De identiteitsmatrix is $\diag{0}{1}$.
                We zien
                \begin{align*}
                    2\cdot\diag{0}{1} = \diag{0}{2}.
                \end{align*}

                Verder
                \begin{align*}
                    \diag{5}{0} + \diag{2}{0} = \diag{7}{0}.
                \end{align*}
            \end{demobox}
        \end{column}
    \end{columns}
\end{frame}

\let\diag\somethingundefined
% \newcommand\diag[1]{\begin{pmatrix}
%     #1 & 0\\
%     0 & #1
% \end{pmatrix}}
\newcommand\diag[2][0]{\begin{pmatrix}
    #2 & #1\\
    #1 & #2
\end{pmatrix}}

\begin{frame}[fragile]{Commando's}
    \begin{columns}
        \begin{column}{0.5\textwidth}
            \begin{minted}[fontsize=\scriptsize,escapeinside=||]{tex}
                \newcommand\diag[2][0]{\begin{pmatrix}
                    #2 & #1\\
                    #1 & #2
                \end{pmatrix}}
        
                \begin{document}
                    De identiteitsmatrix is $\diag{1}$.
                    We zien
                    \begin{align*}
                        2\cdot\diag{1} = \diag{2}.
                    \end{align*}

                    Verder
                    \begin{align*}
                        \diag[5]{0} + \diag[2]{0} = \diag[7]{0}.
                    \end{align*}
                \end{document}
            \end{minted}
        \end{column}
        \begin{column}{0.5\textwidth}
            \begin{demobox}\small\setlength\parskip{5pt}    
                De identiteitsmatrix is $\diag{1}$.
                We zien
                \begin{align*}
                    2\cdot\diag{1} = \diag{2}.
                \end{align*}

                Verder
                \begin{align*}
                    \diag[5]{0} + \diag[2]{0} = \diag[7]{0}.
                \end{align*}
            \end{demobox}
        \end{column}
    \end{columns}
\end{frame}

