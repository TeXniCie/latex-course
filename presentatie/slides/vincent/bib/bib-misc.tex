\begin{saveblock}{codebox}
	\begin{highlightblock}[linewidth=\dimexpr\linewidth - 21.9pt\relax]
		\addcontentsline{toc}{section}{References}
	\end{highlightblock}
\end{saveblock}

\begin{frame}{Opmerkingen}%Goed om te weten I}
	\begin{itemize}[label=\textbullet]
		\item Referentielijst is, net zoals \hll|\\tableofcontents|, niet standaard opgenomen in je
		inhoudstabel. Dit fix je met
		\useblock{codebox}
		
		\item Enkel citaties die je hebt gebruikt verschijnen in je \hll|\\printbibliography|.
		\item Mocht je toch alle citaties uit je \hll|.bib|-bestand in referentielijst willen:\\
		Gebruik \hll|\\nocite\{*\}|, of specifiek item in plaats van ster.
	\end{itemize}
\end{frame}

% \begin{frame}{Goed om te weten II}
% 	\hll|biber| verzorgt een groot deel van de refentielijst, maar wordt niet bij elke compilatie
% 	aangeroepen. Het wordt aangeroepen als	
% 	\begin{itemize}
% 		\item De hulpbestanden (.aux, .bbl, ...) nog genoeg missen.
% 		\item Je in TeXstudio gebruikt \hll|Tools > Bibliography| (F8).
% 		\item Je een nieuwe bron gebruikt in je \hll|.tex|-bestand.
% 		\item TeXstudio ziet dat \hll|.bib|-bestand aangepast is.
% 	\end{itemize}

% 	Maar dus \emph{niet} gewoon omdat je de een paragraaf verwijdert die de laatste citatie van een
% 	referentie had. Doe je zoiets op het laatste moment voor inleveren, compileer, F8, en compileer
% 	nogmaals.
% \end{frame}
