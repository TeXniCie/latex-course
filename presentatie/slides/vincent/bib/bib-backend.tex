
%\section{Bib: Configuratie}

\begin{saveblock}{codebox}
	\begin{highlightblock}
		\usepackage[backend=biber]{biblatex}
	\end{highlightblock}
\end{saveblock}

\begin{saveblock}{codeboxB}
	\begin{highlightblock}
		% !BIB TS-program = biber
	\end{highlightblock}
\end{saveblock}

\begin{frame}{Configuratie}
	\useblock{codebox}

	\bigskip

	Op ge\"installeerde versies soms beetje configuratie nodig.
	Lukt het niet? Vraag het ons. (tijdens cursus of stuur mailtje)
	
	% meer configuratie nodig.
	% Vraag ons als het niet meteen werkt voor je.
	%Zie extra documentatie.

\end{frame}

% \begin{frame}{Configuratie}
% 	De bibliografie wordt geregeld door het package \hll|biblatex|:
% 	%\par\medskip
% 	\useblock{codebox}
% 	%\par\medskip
% 	... samen met backend Biber.
	
% 	\pause

% 	\bigskip

% 	Archa\"isch systeem: Natbib met backend Bibtex. Niet compatibel.

% 	\begin{center}
% 		{\Large\textbf{Biber expliciet kiezen!}}
% 	\end{center}
	
% 	% \begin{alertblock}{Biber expliciet kiezen}		
% 	% 	\adjustbox{width=\linewidth}{Maar: TeXstudio gebruikt Bibtex als standaard! (ook met \hll|backend=biber|)}
		
% 	% 	\only<3->{Provisionele oplossing: magic comments:
% 	% 		\par\medskip
% 	% 		\useblock{codeboxB}
			
% 	% 		\'Echte oplossing: \hll|Options > Configure TeXstudio > Build > Default Bibliography Tool|, zet op \hll|txs:///biber|.
% 	% 	}
% 	% \end{alertblock}
% 	% }
% \end{frame}

% \begin{frame}{Configuratie}
% 	Archa\"isch systeem: Natbib met backend Bibtex. Niet compatibel.

% 	\begin{alertblock}{Biber expliciet kiezen}		
% 		\adjustbox{width=\linewidth}{Maar: TeXstudio gebruikt Bibtex als standaard! (ook met \hll|backend=biber|)}
		
% 		%\only<1->{
% 		Provisionele oplossing: magic comments:
% 		\par\medskip
% 		\useblock{codeboxB}
		
% 		\'Echte oplossing: \hll|Options > Configure TeXstudio > Build > Default Bibliography Tool|, zet op \hll|txs:///biber|.
% 		%}
% 	\end{alertblock}
% \end{frame}
