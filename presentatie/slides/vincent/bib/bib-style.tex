
\begin{frame}{Bibliografiestijl}
	% Bij bibliografie\"en is er een wildernis aan verschillende stijlen:
	Verschillende stijlen zijn mogelijk, onder meer:
	
	\begin{itemize}
		\item \hll|numeric|: This is well-known [2], but also controversial [5, 6].\par
		\includegraphics[width=0.8\textwidth]{assets/biblatexStyles/numericReferences.png}
		
		\item\hll|alphabetic|: This is well-known [GMS94], but also controversial [Gon01, Ham97].
		\item\hll|authoryear|: This is well-known John 2003, ...
		\item\hll|apa|: This is well-known (Lambert, 1993), bb ...
	\end{itemize}
	In APA: \hll|\\cite| en \hll|\\parencite| verschillen
\end{frame}

\updatehighlight{
	name=accentC,
	add={style=numeric}
}

\begin{saveblock}{codebox}
	%\highlightC{style=numeric}%
	\begin{highlightblock}
		\usepackage[style=numeric]{biblatex}
	\end{highlightblock}
\end{saveblock}

\updatehighlight{
	name=accentC,
	remove={style=numeric}
}

\begin{saveblock}{codeboxD}%
	\begin{highlightblock}
		\DeclareLanguageMapping{english}{english-apa}
	\end{highlightblock}
\end{saveblock}

\begin{frame}{Bibliografiestijl}
	%En er zijn nog veel meer stijlen!
	Vaak gebruikt in exacte wetenschappen: numeric.
	\useblock{codebox}
	
	\medskip
	Voor APA-stijl heb je daarnaast nodig:
	\useblock{codeboxD}

	\bigskip
	
	{\large\itshape\bfseries Vraag je begeleider welke stijl!}
\end{frame}
