% \begin{saveblock}{align}
%     \begin{highlightblock}[gobble=8,linewidth=\textwidth,
%         framexleftmargin=0.25em,xleftmargin=0.25em]
%         De verdubbelingsformule herschrijven we nu als
%         \begin{align*}
%             \cos(2\theta) &= \cos^2(\theta) - \sin^2(\theta)\\
%             &= 2\cos^2(\theta)-1.
%         \end{align*}
%     \end{highlightblock}
% \end{saveblock}

% \begin{saveblock}{alignEN}
%     \begin{highlightblock}[gobble=8,linewidth=\textwidth,
%         framexleftmargin=0.25em,xleftmargin=0.25em]
%         The double-angle formula can now be rewritten as
%         \begin{align*}
%             \cos(2\theta) &= \cos^2(\theta) - \sin^2(\theta)\\
%             &= 2\cos^2(\theta)-1.
%         \end{align*}
%     \end{highlightblock}
% \end{saveblock}

% \addtorecentlist{align*}

\begin{frame}[fragile]{Align}
    \begin{minted}[fontsize=\small,escapeinside=~~]{tex}
        De verdubbelingsformule herschrijven we nu als
        \begin{align*}
            \cos(2\theta) &= \cos^2(\theta) - \sin^2(\theta)\\
            &= 2\cos^2(\theta)-1.
        \end{align*}
    \end{minted}
    \bigskip

    \begin{adjustbox}{frame=1pt 10pt}%
        \begin{minipage}{0.7\textwidth-22pt}
            De verdubbelingsformule herschrijven we nu als
            \begin{align*}
                \cos(2\theta) &= \cos^2(\theta) - \sin^2(\theta)\\
                &= 2\cos^2(\theta)-1.
            \end{align*}
        \end{minipage}
    \end{adjustbox}
    % \includegraphics[width=\linewidth,height=0.4\textheight,keepaspectratio]{%
    %     assets/mathAlignNoNumbers\langsuffix.pdf}
\end{frame}
