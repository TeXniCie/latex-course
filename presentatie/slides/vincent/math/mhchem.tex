
\begin{saveblock}{chem}
	\begin{highlightblock}[gobble=8,linewidth=\textwidth,
		framexleftmargin=0.25em,xleftmargin=0.25em]
		\ce{CO2 + C -> 2 CO}\\
		$\ce{CO2 + C -> 2 CO}$\\
		\ce{CH4 + 2 $\left(\ce{O2 + 79/21 N2}\right)$}
		%$\ce{CH4 + 2 \left(\ce{O2 + 79/21 N2}\right)}$ % Error
	\end{highlightblock}
\end{saveblock}

\begin{frame}
	{\lang,Chemical formulas,Chemische formules, \hll|\\usepackage\{mhchem\}|}
	\useblock{chem}

	\medskip

	\ce{CO2 + C -> 2 CO}\\
	$\ce{CO2 + C -> 2 CO}$\\
	\ce{CH4 + 2 $\left(\ce{O2 + 79/21 N2}\right)$}
	%$\ce{CH4 + 2 \left(\ce{O2 + 79/21 N2}\right)}$ % Error

	{\small Some examples are taken from the \hll|mhchem| package documentation (see below)}
	% (
	% 	\url{http://mirrors.ctan.org/macros/latex/contrib/mhchem/mhchem.pdf}
	% )

	%\vspace{1cm}

	\begin{center}
		\lang{
			More example can be found in the documentation of \hll|mhchem|, see
		}{
			Meer voorbeelden in documentatie van \hll|mhchem|, zie
		}\par
		%\url{https://ctan.org/pkg/mhchem}
		\href{https://ctan.org/pkg/mhchem}{\ul{\texttt{https://ctan.org/pkg/mhchem}}}
	\end{center}

	%\href{https://vkuhlmann.com/latex/example}{\ul{\texttt{vkuhlmann.com/latex/example}}}
\end{frame}
