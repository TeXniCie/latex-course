\copyrightVincent

\updatehighlight{
    name=accentC,
    add={align},
    %
    name=default,
    add={}
}

\begin{saveblock}{align}
    \begin{highlightblock}[gobble=8,linewidth=\textwidth,
        framexleftmargin=0.25em,xleftmargin=0.25em]
        De verdubbelingsformule herschrijven we nu als
        \begin{align}
            \cos(2\theta) &= \cos^2(\theta) - \sin^2(\theta)\\
            &= 2\cos^2(\theta)-1.
        \end{align}
    \end{highlightblock}
\end{saveblock}

\begin{saveblock}{alignEN}
    \begin{highlightblock}[gobble=8,linewidth=\textwidth,
        framexleftmargin=0.25em,xleftmargin=0.25em]
        The double-angle formula can now be rewritten as
        \begin{align}
            \cos(2\theta) &= \cos^2(\theta) - \sin^2(\theta)\\
            &= 2\cos^2(\theta)-1.
        \end{align}
    \end{highlightblock}
\end{saveblock}

\begin{frame}{Align}
    \useblock{align\langsuffix}

    \includegraphics[width=\linewidth,height=0.4\textheight,keepaspectratio]{%
        assets/mathAlignDoubleNumber\langsuffix.pdf}
\end{frame}

\updatehighlight{
    name=accentC,
    remove={align},
    add={\nonumber},
    %
    name=default,
    add={},
    %
    name=accentBlack,
    color=black,
    add={align}
}

\providebool{skipNonumber}
\unless\ifskipNonumber
    \begin{saveblock}{align}
        \begin{highlightblock}[gobble=12,linewidth=\textwidth,
            framexleftmargin=0.25em,xleftmargin=0.25em]
            De verdubbelingsformule herschrijven we nu als
            \begin{align}
                \cos(2\theta) &= \cos^2(\theta) - \sin^2(\theta)
                \nonumber\\
                &= 2\cos^2(\theta)-1.
            \end{align}
        \end{highlightblock}
    \end{saveblock}

    \begin{saveblock}{alignEN}
        \begin{highlightblock}[gobble=12,linewidth=\textwidth,
            framexleftmargin=0.25em,xleftmargin=0.25em]
            The double-angle formula can now be rewritten as
            \begin{align}
                \cos(2\theta) &= \cos^2(\theta) - \sin^2(\theta)
                \nonumber\\
                &= 2\cos^2(\theta)-1.
            \end{align}
        \end{highlightblock}
    \end{saveblock}

    \addtorecentlist{\textbackslash nonumber}

    \begin{frame}{Align}
        \useblock{align\langsuffix}

        \includegraphics[width=\linewidth,height=0.4\textheight,keepaspectratio]{%
            assets/mathAlignSecondNumbered\langsuffix.pdf}
    \end{frame}
\fi

\updatehighlight{
    name=accentC,
    remove={\nonumber},
    add={*},
    %
    name=default,
    add={\nonumber}
}

\begin{saveblock}{align}
    \begin{highlightblock}[gobble=8,linewidth=\textwidth,
        framexleftmargin=0.25em,xleftmargin=0.25em]
        De verdubbelingsformule herschrijven we nu als
        \begin{align*}
            \cos(2\theta) &= \cos^2(\theta) - \sin^2(\theta)\\
            &= 2\cos^2(\theta)-1.
        \end{align*}
    \end{highlightblock}
\end{saveblock}

\begin{saveblock}{alignEN}
    \begin{highlightblock}[gobble=8,linewidth=\textwidth,
        framexleftmargin=0.25em,xleftmargin=0.25em]
        The double-angle formula can now be rewritten as
        \begin{align*}
            \cos(2\theta) &= \cos^2(\theta) - \sin^2(\theta)\\
            &= 2\cos^2(\theta)-1.
        \end{align*}
    \end{highlightblock}
\end{saveblock}

\addtorecentlist{align*}

\begin{frame}{Align}
    \useblock{align\langsuffix}

    \includegraphics[width=\linewidth,height=0.4\textheight,keepaspectratio]{%
        assets/mathAlignNoNumbers\langsuffix.pdf}
\end{frame}

\updatehighlight{
    name=accentC,
    remove={*},
    %
    name=accentBlack,
    color=black,
    add={*}
}
