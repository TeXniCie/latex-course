\copyrightVincent

% \begin{saveblock}{align}
%     \begin{highlightblock}[gobble=8,linewidth=\textwidth,
%         framexleftmargin=0.25em,xleftmargin=0.25em]
%         De verdubbelingsformule herschrijven we nu als
%         \begin{align}
%             \cos(2\theta) &= \cos^2(\theta) - \sin^2(\theta)\\
%             &= 2\cos^2(\theta)-1.
%         \end{align}
%     \end{highlightblock}
% \end{saveblock}

% \begin{saveblock}{alignEN}
%     \begin{highlightblock}[gobble=8,linewidth=\textwidth,
%         framexleftmargin=0.25em,xleftmargin=0.25em]
%         The double-angle formula can now be rewritten as
%         \begin{align}
%             \cos(2\theta) &= \cos^2(\theta) - \sin^2(\theta)\\
%             &= 2\cos^2(\theta)-1.
%         \end{align}
%     \end{highlightblock}
%\end{saveblock}

% \begin{frame}[fragile]{Referenties}
%     \begin{columns}
%         \begin{column}{0.5\textwidth}
%             {\tiny De Taylorexpansie van $ e^{ix}+e^{-ix} $ is
%             \begin{align*}
%                 e^{ix}+e^{-ix}
%                 %&= \sum_{n=0}^{\infty}\frac{1}{n!}x^{n}+\sum_{n=0}^{\infty}\frac{1}{n!}(-ix)^{n}\\
%                 &= \sum_{n=0}^{\infty}\frac{1}{n!}(i^{n}+(-i)^n)x^n\\
%                 &= \sum_{n=0}^{\infty}\frac{x^n}{n!}\begin{cases}
%                     2&\mbox{ als $ n=0\text{ (mod 4)} $}\\
%                     0&\mbox{ als $ n=1\text{ (mod 4)} $}\\
%                     -2&\mbox{ als $ n=2\text{ (mod 4)} $}\\
%                     0&\mbox{ als $ n=3\text{ (mod 4)} $}
%                 \end{cases}\\
%                 &= 2\cdot(1 - \frac{1}{2!}x^2 + \frac{1}{4!}x^4 + \dots)\\
%                 &= 2\cdot\cos(x)
%             \end{align*}}
%         \end{column}
%         \begin{column}{0.5\textwidth}
%             {\tiny
%             % De Taylor-reeks van $ e^{ax} $ is
%             % \begin{align}
%             %     e^{ax} = \sum_{n=0}^{\infty}\frac{1}{n!}a^nx^{n},
%             % \end{align}
%             De Taylorexpansie van $ e^{ix}+e^{-ix} $ is
%             \begin{align}\label{eq:expImag}
%                 e^{ix}+e^{-ix} = \sum_{n=0}^{\infty}\frac{1}{n!}(i^n+(-i)^n)x^{n}.
%             \end{align}
%             De term $ i^n+(-i)^n $ kunnen we uitschrijven als
%             % \begin{align*}
%             %     i^{2n+1} + (-i)^{2n+1} &= i\cdot (-1)^n - i\cdot (-1)^n=0,\\
%             %     i^{2n} + (-i)^{2n} &= 2\cdot (-1)^{n}.
%             % \end{align*}
%             % \begin{align*}
%             %     i^{4n}+(-i)^{4n} &= 1+1=2\\
%             %     i^{4n+1}+(-i)^{4n+1} &= i - i = 0\\
%             %     i^{4n+2}+(-i)^{4n+2} &= i^2+(-i)^2= -1 - 1 = 2\\
%             %     i^{4n+3}+(-i)^{4n+3} &= i^3+(-i)^3= -i + i = 0.
%             % \end{align*}
%             \begin{align}\label{eq:imagExpansion}
%                 \begin{split}
%                     i^n+(-i)^n = \begin{cases}
%                         2&\mbox{ als $ n=0\text{ (mod 4)} $}\\
%                         0&\mbox{ als $ n=1\text{ (mod 4)} $}\\
%                         -2&\mbox{ als $ n=2\text{ (mod 4)} $}\\
%                         0&\mbox{ als $ n=3\text{ (mod 4)} $}.
%                     \end{cases}
%                 \end{split}
%             \end{align}
%             Als we \eqref{eq:imagExpansion} invullen in \eqref{eq:expImag} vinden we
%             \begin{align*}
%                 e^{ix}+e^{-ix}
%                 %&= \sum_{n=0}^{\infty}\frac{1}{n!}x^{4n}\left(2 + 0x - 2x^2 - 0x^3\right)\\
%                 &= 2\cdot(1 - \frac{1}{2!}x^2 + \frac{1}{4!}x^4 + \dots)\\
%                 &= 2\cdot\cos(x)
%             \end{align*}
%             }
%         \end{column}
%     \end{columns}

%     % ...
%     % Als we \eqref{eq:expTaylor} invullen in $ e^{ix}+e^{-ix} $ krijgen we
%     % \begin{align*}
%     %     e^{ix}+e^{-ix} &= \sum_{n=0}^{\infty}\frac{1}{n!}(i^{n}+(-i)^n)x^n\\
%     %     &= \sum_{n=0}^{\infty}\frac{1}{n!}\begin{cases}
%     %         2&\mbox{ als $ n=0\text{ (mod 4)} $}\\
%     %         0&\mbox{ als $ n=1\text{ (mod 4)} $}\\
%     %         -2&\mbox{ als $ n=2\text{ (mod 4)} $}\\
%     %         0&\mbox{ als $ n=3\text{ (mod 4)} $}
%     %     \end{cases}\\
%     %     &= 2\cos(x)
%     % \end{align*}
% \end{frame}


\begin{frame}[fragile,t]{Referenties}
    \begin{columns}[t]
        \begin{column}{0.5\textwidth}%
            {\only<2->{\color{gray}}%
            \begin{adjustbox}{frame=1pt 10pt,valign=t}%
                \begin{minipage}{\textwidth-22pt}
                        % {\tiny De integraal van $ \cos^2(5x) $ van $ 0 $ tot $ \pi $ is
                        % \begin{align*}
                        %     \int_0^{\pi} \cos^2(5x)\dif x
                        %     &= \int_0^{\pi} \cos^2(y)\left(\frac{\dif y}{\dif x}\right)^{-1}\dif y\quad (y := 5x)\\
                        %     &= 5\int_0^{\pi/5}\cos^2(y)\dif y\\
                        %     &= 5\int_0^{\pi/5}\frac{1}{2}\left(1-\sin^2(y)\right)+\frac{1}{2}\cos^2(y)\dif y\\
                        %     &= \frac{5}{2}\int_0^{\pi/5}\left(1+\cos(2y)\right)\dif y\\
                        %     &= \frac{5}{2}\left(\pi/5 + \right)
                        % \end{align*}}
                        {\tiny De oplossing van de differentiaalvergelijking $ \frac{\dif v}{\dif t} = \cos^2(t) $
                            is
                            \begin{align*}
                                v(t) & = v_0 + \int_{0}^{t}\cos^2(t)\dif t                                                          \\
                                     & = v_0 + \int_{t'=0}^{t'=t}\left(\frac{1}{2}\cos^2(t')+\frac{1}{2}(1-\sin^2(t'))\right)\dif t' \\
                                     & = v_0 + \frac{1}{2}\int_{t'=0}^{t'=t}\left(1+\cos^2(t')-\sin^2(t')\right)\dif t'             \\
                                     & = v_0 + \frac{1}{2}\int_{t'=0}^{t'=t}\left(1+\cos(2t')\right)\dif t'                         \\
                                     & = v_0 + \frac{1}{4}\int_{2t'=0}^{2t'=2t}\left(1+\cos(2t')\right)\dif\, (2t')                 \\
                                     & = v_0 + \frac{1}{4}\left(2t+\sin(2t)\right)                                                  \\
                                     & = v_0 + \frac{t}{2} + \frac{1}{4}\sin(2t)
                            \end{align*}}
                    \vspace{-20pt}
                \end{minipage}
            \end{adjustbox}}
        \end{column}
        \begin{column}{0.5\textwidth}
            \begin{onlyenv}<2->%
                \begin{adjustbox}{frame=1pt 10pt,valign=t}%
                    \begin{minipage}{\textwidth-22pt}
                        {\tiny\setlength{\abovedisplayskip}{6pt}%
                            \setlength{\belowdisplayskip}{6pt}%
                            \setlength{\abovedisplayshortskip}{0pt}%
                            \setlength{\belowdisplayshortskip}{0pt}%
                            \setcounter{equation}{0}%
                            De oplossing van de differentiaalvergelijking $ \frac{\dif v}{\dif t} = \cos^2(t) $ is
                            \begin{align}
                                v(t) & = v_0 + \int_{0}^{t}\cos^2(t)\dif t.
                            \end{align}
                            De cosinus verdubbelingsformule is
                            \begin{align*}
                                \cos(2t) & = \cos^2(t) - \sin^2(t) \\
                                         & = 2\cos^2(t)-1.
                                %,\mbox{oftewel}\nonumber\\
                                % \cos^2(t) &= \frac{1}{2}\left(1+\cos(2t)\right),
                                %\mbox{, oftewel}\\
                            \end{align*}
                            Beide leden integreren geeft
                            \begin{align*}
                                \frac{1}{2}\sin(2t) & = \left(2\int_0^t\cos^2(t')\dif t'\right) - t.
                            \end{align*}
                            % en dus
                            % \begin{align}
                            %     \int_0^t 2\cos^2(t')\dif t' = t + \frac{1}{2}\sin(2t).
                            % \end{align}
                            Hiermee vinden we (1) als
                            \begin{align*}
                                v(t) & = v_0 + \frac{t}{2} + \frac{1}{4}\sin(2t).
                            \end{align*}
                        }%
                    \end{minipage}
                \end{adjustbox}
            \end{onlyenv}%
        \end{column}
    \end{columns}
\end{frame}

\begin{frame}[fragile,t]{Referenties}
    \begin{columns}[t]
        \begin{column}{0.5\textwidth}%
            \begin{minted}[fontsize=\tiny,escapeinside=~~]{tex}
                De oplossing van de differentiaalvergelijking
                $ \frac{\dif v}{\dif t} = \cos^2(t) $ is
                \begin{align}
                    v(t) &= v_0 + \int_{0}^{t}\cos^2(t)\dif t.
                \end{align}

                ...

                Hiermee vinden we (1) als
                \begin{align*}
                    v(t) &= v_0 + \frac{t}{2}
                    + \frac{1}{4}\sin(2t).
                \end{align*}
            \end{minted}
        \end{column}
        \begin{column}{0.5\textwidth}
            \begin{adjustbox}{frame=1pt 10pt,valign=t}%
                \begin{minipage}{\textwidth-22pt}
                    \begin{onlyenv}<1->%
                        {\tiny\setlength{\abovedisplayskip}{6pt}%
                            \setlength{\belowdisplayskip}{6pt}%
                            \setlength{\abovedisplayshortskip}{0pt}%
                            \setlength{\belowdisplayshortskip}{0pt}%
                            \setcounter{equation}{0}%
                            De oplossing van de differentiaalvergelijking $ \frac{\dif v}{\dif t} = \cos^2(t) $ is
                            \begin{align}
                                v(t) & = v_0 + \int_{0}^{t}\cos^2(t)\dif t.
                            \end{align}
                            De cosinus verdubbelingsformule is
                            \begin{align*}
                                \cos(2t) & = \cos^2(t) - \sin^2(t) \\
                                         & = 2\cos^2(t)-1.
                            \end{align*}
                            Beide leden integreren geeft
                            \begin{align*}
                                \frac{1}{2}\sin(2t) & = \left(2\int_0^t\cos^2(t')\dif t'\right) - t.
                            \end{align*}
                            Hiermee vinden we (1) als
                            \begin{align*}
                                v(t) & = v_0 + \frac{t}{2} + \frac{1}{4}\sin(2t).
                            \end{align*}
                        }%
                    \end{onlyenv}%
                \end{minipage}
            \end{adjustbox}
        \end{column}
    \end{columns}
\end{frame}

\begin{frame}[fragile,t]{Referenties}
    \tiny
    \begin{columns}[t]
        \begin{column}{0.5\textwidth}%
            \begin{minted}[fontsize=\tiny,escapeinside=~~]{tex}
                De snelheid $ v $ is gedefinieerd als
                \begin{align}
                    v &:= \dod{x}{t}
                \end{align}
                De oplossing van de differentiaalvergelijking
                $ \frac{\dif v}{\dif t} = \cos^2(t) $ is
                \begin{align}
                    v(t) &= v_0 + \int_{0}^{t}\cos^2(t)\dif t.
                \end{align}

                ...

                Hiermee vinden we (1) als
                \begin{align*}
                    v(t) &= v_0 + \frac{t}{2}
                    + \frac{1}{4}\sin(2t).
                \end{align*}
            \end{minted}
        \end{column}
        \begin{column}{0.5\textwidth}
            \begin{adjustbox}{frame=1pt 10pt,valign=t}%
                \begin{minipage}{\textwidth-22pt}
                    \begin{onlyenv}<1->%
                        {\tiny\setlength{\abovedisplayskip}{3pt}%
                            \setlength{\belowdisplayskip}{3pt}%
                            \setlength{\abovedisplayshortskip}{0pt}%
                            \setlength{\belowdisplayshortskip}{0pt}%
                            \setcounter{equation}{0}%
                            De snelheid $ v $ is gedefinieerd als
                            \begin{align}
                                v &:= \dod{x}{t}
                            \end{align}
                            De oplossing van de differentiaalvergelijking $ \frac{\dif v}{\dif t} = \cos^2(t) $ is
                            \begin{align}
                                v(t) & = v_0 + \int_{0}^{t}\cos^2(t)\dif t.
                            \end{align}
                            De cosinus verdubbelingsformule is
                            \begin{align*}
                                \cos(2t) & = \cos^2(t) - \sin^2(t) \\
                                         & = 2\cos^2(t)-1.
                            \end{align*}
                            Beide leden integreren geeft
                            \begin{align*}
                                \frac{1}{2}\sin(2t) & = \left(2\int_0^t\cos^2(t')\dif t'\right) - t.
                            \end{align*}
                            Hiermee vinden we (1) als
                            \begin{align*}
                                v(t) & = v_0 + \frac{t}{2} + \frac{1}{4}\sin(2t).
                            \end{align*}
                        }%
                    \end{onlyenv}%
                \end{minipage}
            \end{adjustbox}
        \end{column}
    \end{columns}
\end{frame}

\begin{frame}[fragile,t]{Referenties}
    \tiny
    \begin{columns}[t]
        \begin{column}{0.5\textwidth}%
            \begin{minted}[fontsize=\tiny,highlightlines={9,14},escapeinside=~~]{tex}
                De snelheid $ v $ is gedefinieerd als
                \begin{align}
                    v &:= \dod{x}{t}
                \end{align}
                De oplossing van de differentiaalvergelijking
                $ \frac{\dif v}{\dif t} = \cos^2(t) $ is
                \begin{align}
                    v(t) &= v_0 + \int_{0}^{t}\cos^2(t)\dif t.
                    \label{eq:exprVelocity}
                \end{align}

                ...

                Hiermee vinden we (\ref{eq:exprVelocity}) als
                \begin{align*}
                    v(t) &= v_0 + \frac{t}{2}
                    + \frac{1}{4}\sin(2t).
                \end{align*}
            \end{minted}
        \end{column}
        \begin{column}{0.5\textwidth}
            \begin{adjustbox}{frame=1pt 10pt,valign=t}%
                \begin{minipage}{\textwidth-22pt}
                    \begin{onlyenv}<1->%
                        {\tiny\setlength{\abovedisplayskip}{3pt}%
                            \setlength{\belowdisplayskip}{3pt}%
                            \setlength{\abovedisplayshortskip}{0pt}%
                            \setlength{\belowdisplayshortskip}{0pt}%
                            \setcounter{equation}{0}%
                            De snelheid $ v $ is gedefinieerd als
                            \begin{align}
                                v &:= \dod{x}{t}
                            \end{align}
                            De oplossing van de differentiaalvergelijking $ \frac{\dif v}{\dif t} = \cos^2(t) $ is
                            \begin{align}
                                v(t) & = v_0 + \int_{0}^{t}\cos^2(t)\dif t.
                                \label{eq:exprVelocity}
                            \end{align}
                            De cosinus verdubbelingsformule is
                            \begin{align*}
                                \cos(2t) & = \cos^2(t) - \sin^2(t) \\
                                         & = 2\cos^2(t)-1.
                            \end{align*}
                            Beide leden integreren geeft
                            \begin{align*}
                                \frac{1}{2}\sin(2t) & = \left(2\int_0^t\cos^2(t')\dif t'\right) - t.
                            \end{align*}
                            Hiermee vinden we (\ref{eq:exprVelocity}) als
                            \begin{align*}
                                v(t) & = v_0 + \frac{t}{2} + \frac{1}{4}\sin(2t).
                            \end{align*}
                        }%
                    \end{onlyenv}%
                \end{minipage}
            \end{adjustbox}
        \end{column}
    \end{columns}
\end{frame}


% \begin{frame}[fragile]{Label}
%     \begin{minted}[fontsize=\scriptsize,escapeinside=~~]{tex}

%     \end{minted}

%     De Taylorexpansie van $ e^{x} $ is
%     \begin{align}
%         e^{x} = \sum_{n=0}^{\infty}\frac{1}{n!}x^{n}
%         \label{eq:expTaylor}
%     \end{align}
%     \textellipsis

%     Als we \eqref{eq:expTaylor} invullen in $ e^{ix}+e^{-ix} $ krijgen we
%     \begin{align*}
%         e^{ix}+e^{-ix} &= \sum_{n=0}^{\infty}\frac{1}{n!}(i^{n}+(-i)^n)x^n\\
%         &= \sum_{n=0}^{\infty}\frac{1}{n!}\begin{cases}
%             2&\mbox{ als $ n=0\text{ (mod 4)} $}\\
%             0&\mbox{ als $ n=1\text{ (mod 4)} $}\\
%             -2&\mbox{ als $ n=2\text{ (mod 4)} $}\\
%             0&\mbox{ als $ n=3\text{ (mod 4)} $}
%         \end{cases}\\
%         &= 2\cos(x)
%     \end{align*}
% \end{frame}


% % De Taylorexpansie van $ \sin^2(\theta) $ is
% %         \begin{align}
% %             \sin^2(\theta) &= \left(\sum_{i=0}^{\infty}\frac{1}{(2i+1)!}(-1)^{i}x^{2i+1}\right)\\
% %             &= 
% %         \end{align}

% \begin{frame}[fragile]{Label}
%     \begin{minted}[fontsize=\scriptsize,escapeinside=~~]{tex}
%         De Taylorexpansie van $ e^{x} $ is
%         \begin{align}
%             e^{x} = \sum_{n=0}^{\infty}\frac{1}{n!}x^{n}
%             \label{eq:expTaylor}
%         \end{align}
%         ...
%         Als we \eqref{eq:expTaylor} invullen in $ e^{ix}+e^{-ix} $ krijgen we
%         \begin{align*}
%             e^{ix}+e^{-ix} &= \sum_{n=0}^{\infty}\frac{1}{n!}(i^{n}+(-i)^n)x^n\\
%             &= \sum_{n=0}^{\infty}\frac{1}{n!}\begin{cases}
%                2&\mbox{ als $ n=0\text{ (mod 4)} $}\\
%                0&\mbox{ als $ n=1\text{ (mod 4)} $}\\
%                -2&\mbox{ als $ n=2\text{ (mod 4)} $}\\
%                0&\mbox{ als $ n=3\text{ (mod 4)} $}
%             \end{cases}\\
%             &= 2\cos(x)
%         \end{align*}
%     \end{minted}

%     De Taylorexpansie van $ e^{x} $ is
%     \begin{align}
%         e^{x} = \sum_{n=0}^{\infty}\frac{1}{n!}x^{n}
%         \label{eq:expTaylor}
%     \end{align}
%     \textellipsis

%     Als we \eqref{eq:expTaylor} invullen in $ e^{ix}+e^{-ix} $ krijgen we
%     \begin{align*}
%         e^{ix}+e^{-ix} &= \sum_{n=0}^{\infty}\frac{1}{n!}(i^{n}+(-i)^n)x^n\\
%         &= \sum_{n=0}^{\infty}\frac{1}{n!}\begin{cases}
%             2&\mbox{ als $ n=0\text{ (mod 4)} $}\\
%             0&\mbox{ als $ n=1\text{ (mod 4)} $}\\
%             -2&\mbox{ als $ n=2\text{ (mod 4)} $}\\
%             0&\mbox{ als $ n=3\text{ (mod 4)} $}
%         \end{cases}\\
%         &= 2\cos(x)
%     \end{align*}
% \end{frame}

% \providebool{skipNonumber}
% \unless\ifskipNonumber
%     \begin{saveblock}{align}
%         \begin{highlightblock}[gobble=12,linewidth=\textwidth,
%             framexleftmargin=0.25em,xleftmargin=0.25em]
%             De verdubbelingsformule herschrijven we nu als
%             \begin{align}
%                 \cos(2\theta) &= \cos^2(\theta) - \sin^2(\theta)
%                 \nonumber\\
%                 &= 2\cos^2(\theta)-1.
%             \end{align}
%         \end{highlightblock}
%     \end{saveblock}

%     \begin{saveblock}{alignEN}
%         \begin{highlightblock}[gobble=12,linewidth=\textwidth,
%             framexleftmargin=0.25em,xleftmargin=0.25em]
%             The double-angle formula can now be rewritten as
%             \begin{align}
%                 \cos(2\theta) &= \cos^2(\theta) - \sin^2(\theta)
%                 \nonumber\\
%                 &= 2\cos^2(\theta)-1.
%             \end{align}
%         \end{highlightblock}
%     \end{saveblock}

%     \addtorecentlist{\textbackslash nonumber}

%     \begin{frame}{Align}
%         \useblock{align\langsuffix}

%         \includegraphics[width=\linewidth,height=0.4\textheight,keepaspectratio]{%
%             assets/mathAlignSecondNumbered\langsuffix.pdf}
%     \end{frame}
% \fi


% \begin{saveblock}{align}
%     \begin{highlightblock}[gobble=8,linewidth=\textwidth,
%         framexleftmargin=0.25em,xleftmargin=0.25em]
%         De verdubbelingsformule herschrijven we nu als
%         \begin{align*}
%             \cos(2\theta) &= \cos^2(\theta) - \sin^2(\theta)\\
%             &= 2\cos^2(\theta)-1.
%         \end{align*}
%     \end{highlightblock}
% \end{saveblock}

% \begin{saveblock}{alignEN}
%     \begin{highlightblock}[gobble=8,linewidth=\textwidth,
%         framexleftmargin=0.25em,xleftmargin=0.25em]
%         The double-angle formula can now be rewritten as
%         \begin{align*}
%             \cos(2\theta) &= \cos^2(\theta) - \sin^2(\theta)\\
%             &= 2\cos^2(\theta)-1.
%         \end{align*}
%     \end{highlightblock}
% \end{saveblock}

% \addtorecentlist{align*}

% \begin{frame}{Align}
%     \useblock{align\langsuffix}

%     \includegraphics[width=\linewidth,height=0.4\textheight,keepaspectratio]{%
%         assets/mathAlignNoNumbers\langsuffix.pdf}
% \end{frame}

% \updatehighlight{
%     name=accentC,
%     remove={*},
%     %
%     name=accentBlack,
%     color=black,
%     add={*}
% }
