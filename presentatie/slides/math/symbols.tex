\def\frameSelection{100-}
\beginDetail{20}
	\def\frameSelection{1-}
\endDetail

\beginFrameWithSelection{\frameSelection}{\lang,Formulas: Symbols,Formules: Symbolen,}%
	\renewcommand{\arraystretch}{1.5}%
	\begin{tabularx}{0.6\textwidth}{ll}
		\toprule
		\lang,Formula,Formule, {\global\showcount=1\relax}& Code\\
		\midrule
		\showformula{$ x_1,\dots,x_n $}{x_1,\\dots,x_n}\\
		\showformula{$ \alpha,\beta,\gamma $}{\\alpha,\\beta,\\gamma}\\
		\showformula{$ \epsilon,\varepsilon $}{\\epsilon,\\varepsilon}\\
		\showformula{$ \phi,\varphi $}{\\phi,\\varphi}\\
		\bottomrule
	\end{tabularx}%
	\begin{tabularx}{0.4\textwidth}{ll}
		\toprule
		\lang,Formula,Formule, {\global\showcount=5\relax}& Code\\
		\midrule
		\showformula{$ 5\cdot 6 $}{5\\cdot 6}\\
		%\showformula{$ x_1 + \dots + x_n $}{x_1 + \\dots + x_n}\\
		\showformula{$ A,B,\Gamma $}{A,B,\\Gamma}\\
		\showformula{$ \mathcal{P} $}{\\mathcal\{P\}}\\
		\showformula{$ \mathbb{P} $}{\\mathbb\{P\}}\\
		%\showformula{$ f(\{3, 6\}) $}{f(\\\{3, 6\\\})}\\
		\bottomrule
	\end{tabularx}%
	% \par\addvspace{0.5\baselineskip}
	% \unless\ifishandout
	% \fi
\end{frame}

\addtorecentlist{\textbackslash varphi}
\addtorecentlist{\textbackslash mathcal}
\addtorecentlist{\textbackslash mathbb}


% \begin{frame}
	
% 	\begin{align*}
% 		\nabla\times(\nabla\times\mathbf{A}) = \nabla(\nabla\cdot \mathbf{A}) - \nabla^2\mathbf{A}
% 	\end{align*}
% \end{frame}

% \begin{frame}
	
% 	\begin{equation*}
% 		\alpha,\beta,\gamma,\delta,\epsilon,\zeta,\eta,\theta,\iota,\kappa,\lambda,
% 		\mu,\nu,\xi,o,\pi,\rho,\sigma,\tau,\upsilon,\phi,\chi,\psi,\omega
% 	\end{equation*}
% 	\begin{equation*}
% 		A,B,\Gamma,\Delta,E,Z,H,\Theta,I,K,\Lambda,
% 		M,N,\Xi,O,\Pi,P,\Sigma,T,\Upsilon,\Phi,\chi,\Psi,\Omega
% 	\end{equation*}
% 	\begin{equation*}
% 		\varDelta,\varepsilon,\varGamma,\varkappa,\varLambda,%\varnothing,
% 		\varOmega,\varPhi,\varphi,\varPi,\varpi,%\varpropto,
% 		\varPsi,\varrho,
% 		\varSigma,\varsigma,\varTheta,\vartheta,%\vartriangle,
% 		\varUpsilon,\varXi
% 	\end{equation*}

	
% 	\begin{equation}
% 		\mathbb{P}, \mathcal{C}
% 	\end{equation}
	
% 	\begin{equation}
% 		\forall, \exists, \lnot, \land, \lor, \wedge, \hat{\imath}, \hat{n}, \vec{F}_{\text{tot}},
% 		\frac{\partial f}{\partial x}, \frac{\dif f}{\dif y}, %\left A\dfrac{2}{3}\middle B\right C
% 	\end{equation}
% 	% \DeclareMathOperator{\test}{test}
% \end{frame}

