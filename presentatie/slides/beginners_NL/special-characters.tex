\copyrightTim
\copyrightVincent
    \begin{frame}[fragile]
        \frametitle{Speciale tekens}
    
        \begingroup
        \renewcommand{\arraystretch}{1}
        \begin{tabularx}{0.45\textwidth}{ll}
            \toprule
            Code & Resultaat\\
            \midrule
            \hll|\\\{| & \{\only<2->{\hskip-10pt\relax
            \adjustbox{
                padding={-30px 0px 0px 0px},left=2ex,set height=8pt,
                set depth=136pt,cfbox=red 1pt,left=0pt,set depth=0pt,
                set height=0pt
            }{}
            }\\
            \hll|\\\}| & \}\\
            \hll|\\\%| & \%\\
            \hll|\\\_| & \_\\
            \hll|\\textasciicircum| & \textasciicircum\\
            \hll|\\\$| & \$\\
            \hll|\\textbackslash| & \textbackslash\\
            \hll|\\\&| & \&\\
            \hll|\\\#| & \#\\
            \hll|\\textgreater| & \textgreater\\
            \hll|\\textless| & \textless\\
            \bottomrule
        \end{tabularx}%
        \hfil
        \begin{tabularx}{0.5\textwidth}{ll}
            \toprule
            Code & Resultaat,\\
            \midrule
            \hll|\{| & Begin groep,\\
            \hll|\}| & Eindig groep,\\
            \hll|\%| & Comment\\%\footnote{verschijnt niet in output}\\
            \hll|\_| & Betekenis voor wiskunde,\\
            \hll|^| & Betekenis voor wiskunde,\\
            \hll|\$| & Wiskundemodus,\\
            \hll|\\| & Commando,\\
            \hll|\&| & Kolomscheiding,\\
            \hll|\#| & Parameter\\
            \hll|>| & >\\
            \hll|<| & <\\
            \bottomrule
        \end{tabularx}
        \endgroup
    \end{frame}