\copyrightVincent

\def\acrnCopyrightline#1#2#3{%
	\item Copyright (c) #2 \textbf{#1}\relax
}

\acrnPrepare

\resetul

\begin{frame}
	\frametitle{\lang,License,Licentie,}

	\begin{center}
		\setulcolor{red}
		\ul{Contributors}\medskip

		\adjustbox{scale=0.8,set depth={0.4\textheight}}{%
			%\adjustbox{scale=0.8,set depth={0.4\textheight}}{%
			\begin{varwidth}{25em}%
				\ttfamily
				\begin{list}{}{
					\setlength{\leftmargin}{20pt}
					\itemindent-\leftmargin
					\setlength{\itemsep}{0pt}
					\setlength{\parskip}{0pt}
				}
					\copyrightlines
					% 	\item Copyright (c) 2022 Tim Weijers
					% 	\item Copyright (c) 2021-2022 Vincent Kuhlmann
					% 	\item Copyright (c) 2022 Hanneke Schroten
					% 	\item Copyright (c) 2022 Thomas van Maaren
				\end{list}
			\end{varwidth}%
		}
	\end{center}

	\begin{center}
		\tiny

		\lang{
			% By contributing, you license your source code to the \TeX niCie under
			% MIT license.\bigskip
			
			The \TeX niCie licenses this PDF to the public under
	
			\smallskip
	
			\textbf{Creative Commons CC BY-NC-ND 4.0}
	
			\medskip

			To use any alterations of the slides, please
			request a different license from the \TeX niCie first.
		}{
			% Door bij te dragen aan de presentatie, stel je je broncode beschikbaar
			% aan de \TeX niCie onder MIT licentie.\bigskip
	
			De \TeX niCie licenseert deze PDF aan het publiek onder
	
			\smallskip
	
			\textbf{Creative Commons CC BY-NC-ND 4.0}
	
			\medskip

			Als je slide-inhoud in een andere presentatie wil gebruiken, moet je de
			\TeX niCie eerst om een andere licentie vragen.
		}
	\end{center}
\end{frame}
