%% !TeX program = pdflatex

\documentclass[
    dutch,
    everyoneauthor=true,
    % darktheme,
    defaultSlideCollection=vincent,
    %slidenames=true,
    % handout
]{../../cursuspresentatie}

\usepackage{minted}

\ifbool{darktheme}{
    \definecolor{codehighlight}{RGB}{44, 78, 120}
}{
    \definecolor{codehighlight}{RGB}{177, 213, 255}
}

\def\codeEmphasize#1{\ul{#1}}


\def\customhead#1#2{%
    \ifnum\value{section}=#1\relax
        \textcolor{fg!100!bg}{%
            \setul{1pt}{0.5pt}\setulcolor{yellow}\ul{#2}%
        }%
    \else
        #2%
    \fi
}

% \setbeamertemplate{headline}
% {%
%     \begin{beamercolorbox}[colsep=1.5pt]{upper separation line head}%
%     \end{beamercolorbox}
%     \begin{beamercolorbox}{section in head/foot}
%         \vskip2pt
%         \usebeamerfont{section in head/foot}\usebeamercolor[fg]{section in head/foot}%
%         \color{fg!80!bg}
%         % \ifnum\c@section=2%
%         %     \hyperlink{Navigation#3}{{\usebeamertemplate{section in head/foot}}}%
%         % \else%
%         %     \hyperlink{Navigation#3}{{\usebeamertemplate{section in head/foot shaded}}}%
%         % \fi}%
%         \adjustbox{
%             set depth=2pt,
%             padding={5pt 5pt 5pt 5pt},bgcolor=red!40!orange,
%             rndcorners={3pt 0pt 3pt 0pt},margin={10pt 0pt 0pt 0pt}
%         }{%
%             \customhead{2}{Document en referenties}%
%             \hspace{5pt}$ \cdot $\hspace{5pt}
%             \customhead{3}{Figuren}%
%         }%
%         \adjustbox{
%             set depth=2pt,
%             padding={5pt 5pt 5pt 5pt},bgcolor=red!20!orange,
%             rndcorners={0pt 3pt 0pt 3pt},margin={0pt 0pt 10pt 0pt}
%         }{%
%             Oefeningen%
%         }%
%         % \adjustbox{padding={3pt 5pt 3pt 5pt},bgcolor=red!40!orange,rndcorners=3pt,margin={10pt 0pt 10pt 0pt}}{
%         %     `Theorem' en `Lemma'
%         %     \hspace{5pt}$ \cdot $\hspace{5pt}
%         %     Matrices en tabellen
%         % }
%         \adjustbox{
%             %set depth=2pt,
%             set depth=2pt,
%             padding={5pt 5pt 5pt 5pt},bgcolor=red!40!orange,
%             rndcorners={3pt 0pt 3pt 0pt},margin={10pt 0pt 0pt 0pt},
%         }{%
%             \customhead{4}{`Theorem' en `Lemma'}
%             \hspace{5pt}$ \cdot $\hspace{5pt}
%             \customhead{5}{Matrices en tabellen}\vphantom{g} %
%         }%
%         \adjustbox{
%             set depth=2pt,    
%             padding={5pt 5pt 5pt 5pt},bgcolor=red!20!orange,
%             rndcorners={0pt 3pt 0pt 3pt},margin={0pt 0pt 10pt 0pt}
%         }{%
%             Oefeningen%
%         }
%         \hfil
%         Slides op texnicie.nl
%         \hfil
%         %\strut %AAAA
%         \vskip4pt
%     \end{beamercolorbox}%
%     \begin{beamercolorbox}[colsep=1.5pt]{lower separation line head}%
%     \end{beamercolorbox}%
% }

\let\placetarget\relax


\setbeamertemplate{headline}
{%
    \placetarget
    \let\placetarget\relax
    \begin{beamercolorbox}[colsep=1.5pt]{upper separation line head}%
    \end{beamercolorbox}
    \begin{beamercolorbox}{section in head/foot}
        \vskip2pt
        \usebeamerfont{section in head/foot}\usebeamercolor[fg]{section in head/foot}%
        \color{fg!80!bg}
        % \ifnum\c@section=2%
        %     \hyperlink{Navigation#3}{{\usebeamertemplate{section in head/foot}}}%
        % \else%
        %     \hyperlink{Navigation#3}{{\usebeamertemplate{section in head/foot shaded}}}%
        % \fi}%
        \adjustbox{
            set depth=2pt,
            padding={5pt 5pt 5pt 5pt},%bgcolor=navigationTileOutline,
            rndframe={color=navigationTileOutline}{3pt 0pt 3pt 0pt},margin={10pt 0pt 0pt 0pt}
        }{%
            \hyperlink{document}{\customhead{2}{Master your document}}%
            \hspace{5pt}$ \cdot $\hspace{5pt}%
            \hyperlink{inzicht}{\customhead{3}{Inzicht}}%
            % \hspace{5pt}$ \cdot $\hspace{5pt}%
            % \hyperlink{theorem}{\customhead{4}{`Theorem'}}%
        }%
        \adjustbox{
            set depth=2pt,
            padding={5pt 5pt 5pt 5pt},
            bgcolor=exerciseTileBackground,
            rndframe={color=exerciseTileOutline}{0pt 3pt 0pt 3pt},
            margin={-1pt 0pt 10pt 0pt}
        }{%
            \hyperlink{oefeningen1}{\customhead{4}{Oefeningen}}%
        }%
        % \adjustbox{padding={3pt 5pt 3pt 5pt},bgcolor=navigationTileOutline,rndcorners=3pt,margin={10pt 0pt 10pt 0pt}}{
        %     `Theorem' en `Lemma'
        %     \hspace{5pt}$ \cdot $\hspace{5pt}
        %     Matrices en tabellen
        % }
        \adjustbox{
            %set depth=2pt,
            set depth=2pt,
            padding={5pt 5pt 5pt 5pt},%bgcolor=navigationTileOutline,
            rndframe={color=navigationTileOutline}{3pt 0pt 3pt 0pt},margin={10pt 0pt 0pt 0pt}
            % rndcorners={3pt 0pt 3pt 0pt}
        }{%
            \hyperlink{handigheidjes}{\customhead{5}{Handigheidjes}}%
            \hspace{5pt}$ \cdot $\hspace{5pt}%
            \hyperlink{nummeringen}{\customhead{6}{Nummeringen}}%
            \hspace{5pt}$ \cdot $\hspace{5pt}%
            \hyperlink{adjustbox}{\customhead{7}{Adjustbox}}%
            \hspace{5pt}$ \cdot $\hspace{5pt}%
            \hyperlink{opUitstap}{\customhead{8}{Op uitstap}}%
        }%
        % \adjustbox{
        %     set depth=2pt,    
        %     padding={5pt 5pt 5pt 5pt},
        %     % bgcolor=exerciseTileOutline,
        %     % rndcorners={0pt 3pt 0pt 3pt},
        %     bgcolor=exerciseTileBackground,
        %     rndframe={color=exerciseTileOutline}{0pt 3pt 0pt 3pt},
        %     margin={-1pt 0pt 10pt 0pt}
        % }{%
        % \hyperlink{oefeningen2}{\customhead{9}{Oefeningen}}%
        % }
        \hfil
        Slides op texnicie.nl
        \hfil
        %\strut %AAAA
        \vskip4pt
    \end{beamercolorbox}%
    \begin{beamercolorbox}[colsep=1.5pt]{lower separation line head}%
    \end{beamercolorbox}%
}


%\newmintinline[cc]{latex}{}

\ifbool{darktheme}{
    \usemintedstyle{github-dark}
    \def\codeEmphasize#1{\textcolor{white}{\ul{#1}}}
}{
    \usemintedstyle{pastie}
    %\usemintedstyle{default}
}

\setminted[tex]{highlightcolor=codehighlight}%blue!60!white!60!black}


\setminted[tex]{fontsize=\small, autogobble=true, linenos=false, frame=none}
\usepackage{tcolorbox}

%\author{TeXniCie}
\date{\today}
\title{Manual for thesis}

%%% Overview of this file in order:
% Packages which don't need options
% Packages which have one or few options
% Geometry package
% Header/footer settings
% Theorem styles
% Enable/disable parindents
% Reference and bibliography settings
% Front/main/back-matter


%%%%%%%%%%%%%%%%%%%%%%%%%%%%%%%%%%%%%%%%%

\usepackage{
		%layout,		% Allow visualisation of all the margins
		subfiles,		% For separate main and sub documents
		graphicx,		% For image modifications and the figure environment
		amsmath,		% For the AMS math styles
		amssymb,		% The extended AMS math symbol list
		amsthm,			% For use of theorems (works together with thmtools)
		fancyhdr,		% For fancy headers and footers on pages
		%gensymb,		% For easy generic symbols (uniform use in math and text mode)
		%sidecap,		% For use of captions next to a float (figure, table, etc)
		subcaption,		% For easy subfigures in a plot (with nice captions)
		tikz,			% Difficult drawing of awesome vector plots
		%listings,		% For listing pieces of code in a nice and neat way
		multicol,		% For easy local multicolumn use
		color,			% For handy colour definitions (used cause of styling)
		%calc,			% To calculate stuff for the back-end
		%mdwlist,		% For customising lists
		thmtools,		% Lets you define your own theorem style (used for all the fancy theorems, definitions etc.)
		etoolbox,		% Allows adjustment of commands (used to reset the claim counter at the end of a proof).
		xspace,			% Makes latex not eat spaces after commands
        hyperref,		% Makes links, references, the Table of Contents, etc. clickable.
        mathtools,      % Extensible symbols, such as brackets, arrows, harpoons, etc.;
        inputenc,       % Allows to use things like ö instead of \"o in your text.
        wrapfig,        % wrap text around a smaller figure
        framed          % allows for boxes around your important equations or theorems
        }

%%%%%%%%%%%%%%%%%%%%%%%%%%%%%%%%%%%%%%%%%%

\usepackage[english]{babel} % Correct language setting, 'british', 'american'='english' or 'dutch'.
\usepackage[autostyle]{csquotes} % Fixes quotes to correspond to the babel language.
% Note the difference between ``quotes'' and ''quotes'' when using different languages.


%%%%%%%%%%%%%%%%%%%%%%%%%%%%%%%%%%%%%%%%%%

 \usepackage[margin=2.5cm]{geometry}
 % Change the shape of a page (custom margins etc.)
 % paper=a4paper slightly changes the style through the whole document.
    %%%% We set the margins for whole document here, except the titlepage. The titlepage uses special margins; see titlepage.tex.


%%%%%%%%%%%%%%%%%%%%%%%%%%%%%%%%%%%%%%%%%%

%%% This is about changing the headers and footers (i.e. Top and bottom of the page)

\pagestyle{fancy}% use fancyheaders with the bar on the top
\fancyhf{} % Clear the normal style
\fancyhead[L]{\bfseries\leftmark} %this places the section number and name in the top left
\fancyhead[R]{\bfseries\thepage}% this places the pagenumber in the top right
	

%%%%%%%%%%%%%%%%%%%%%%%%%%%%%%%%%%%%%%%%%%%%
%%%%		Theorem style

% The set-up is as follows, first you give the 'style' of your theorem. This determines whether it for instance is plain, or italic. Secondly you can give an option for the symbol on the end, normally it is nothing. But you could add some to increase the readability of your text. Finally you can use numberwithin to add the number of your section/theorem before your equations. This is useful if you want to keep the numbers of your equation in check (In this thesis there where over a 100) and keeps in order where the equations are.
%Finally you can use sibling to let different 'theorems' count together. Hence you will get Theorem 1 Definition 2 Claim 3, instead of Theorem 1 Definition 1 Claim 1. This is a matter of taste.

% Theorem definitions
\declaretheorem[style=definition,numberwithin=subsection]{definition} %If you want your theorems to be counted per section instead of subsection, then just remove the sub from the numberwithin
\declaretheorem[style=definition,qed=$\triangle$,sibling=definition]{example}% sibling says with what type of theorems you wan the numbering to count with.

\declaretheorem[style=plain,sibling=definition]{theorem}
\declaretheorem[style=plain,sibling=definition]{lemma}
\declaretheorem[style=plain,sibling=definition]{proposition}
\declaretheorem[style=plain,sibling=definition]{corollary}
\declaretheorem[style=definition]{claim}
\declaretheorem[style=definition,sibling=example]{remark}

\AtEndEnvironment{proof}{\setcounter{claim}{0}} % Sets the claim number to 0 after ending a proof

% You can make short-hands like these. 

\newcommand{\thm}[2]{\begin{theorem} #1 \begin{proof} #2 \end{proof} \end{theorem}}
\newcommand{\lm}[2]{\begin{lemma} #1 \begin{proof} #2 \end{proof} \end{lemma}}
\newcommand{\df}[1]{\begin{definition} #1 \end{definition}}
\newcommand{\clm}[1]{\begin{claim} #1 \end{claim}}


%%%%%%%%%%%%%%%%%%%%%%%%%%%%%%%%%%%%%%%%%%%%

%Let equation numbers be numbered by 1.1, 1.2, 2.1, etc where the first number is the section number. section can also be replaced by chapter when using book class.
\numberwithin{equation}{section}

%%%%%%%%%%%%%%%%%%%%%%%%%%%%%%%%%%%%%%%%%%%%

% Comment/uncomment the following to disable/enable parindents:
\setlength\parindent{0pt}

%%%%%%%%%%%%%%%%%%%%%%%%%%%%%%%%%%%%%%%%%%%%

%%%% Add the bibliography with some settings:
% package:
\usepackage[% Options
style = numeric-comp, % 
% Choose the style of your citations (and correspondingly your bibliography).
% Few examples:
% numeric = [15, 16, 17, 20], numeric-comp = [15-17, 20], numeric-verb = [15]; [16]; [17]; [20], alphabetic = [Jon99, Wil93, BT86, Zil13], authoryear = Jones 99, Wilfred 93, Bohr, Turing 86, Ziltener 13.
%
% List of all: (you probably want a version of numeric, alphabetic or authoryear)
% numeric, numeric-comp, numeric-verb,
% nature (like numeric, but with '23.' instead of '[23]' in the bibliography),
% apa (does not work well with out with only 'year'; really needs a full date)
% alphabetic, alphabetic-verb,
% authoryear, authoryear-comp, authoryear-ibid, authoryear-icomp,
% authortitle, authortitle-comp, authortitle-ibid, authortitle-icomp, authortitle-terse, authortitle-tcomp, authortitle-ticomp,
% verbose, verbose-ibid, verbose-note, verbose-inote, verbose-trad1, verbose-trad2, verbose-trad3,
% reading, (draft, debug)
sorting = none, % 
% Choose how the bibliography is sorted.
% Options: nty, nyt, nyvt, anyt, anyvt, ynt, ydnt, none, (debug)
% Here n = name, t = title, y = year, v = volume, a = alphabetic label, ...d = ... descending
% So nty = sort by name, then title, then year.
backend = biber%
% This is the default, and should almost alway be kept as such. 'backend = bibtex' is legacy.
]{biblatex}

% If you use APA, you will need:
% \DeclareLanguageMapping{english}{english-apa}

% There exist related packages for specific styles like biblatex-chicago (Chicago manual of style citations) or biblatex-jura (German legal citations). You most likely won't need them or use them.

% For more information and a nice matrix with typesupport, see:
% https://en.wikibooks.org/wiki/LaTeX/Bibliography_Management#biblatex

% The source file for you bibliography:
\addbibresource{bibfile.bib}
% It's possible to add multiple bib files and separate them based on label (so to have two different references lists e.g. to seperate main sources from minor sources or books and theses from misc sources); see in 3.7 of the BibLaTeX documentation if you want to do stuff like that.

%%%%%%%%%%%%%%%%%%%%%%%%%%%%%%%%%%%%%%%%%%%%

%%%%% frontmatter/mainmatter/backmatter:
\newcommand\frontmatter{%
    \cleardoublepage
    \pagenumbering{roman}} %small Roman numbers

\newcommand\mainmatter{%
    \cleardoublepage
    \pagenumbering{arabic}} %normal numbers

\newcommand\backmatter{%
    \cleardoublepage %% double page style
    %\clearpage %% single page style
    \pagenumbering{Roman}} %capital Roman numbers
   




\copyrightVincent

% \def\importslide#1#2{%
% 	\import{../../slides/#1}{#2}
% }



% \def\importslide#1#2{%
%     \IfFileExists{../../slides/vincent/#1/#2}{%
%         \import{../../slides/vincent/#1}{#2}%
%     }{%
%         \import{../../slides/#1}{#2}%
%     }%
% }

% \title[\LaTeX{}-cursus Week 2]{\LaTeX{}-cursus Week 2\\(Lokaal Vincent)}
\title[\LaTeX{}-cursus Week 2]{\LaTeX{}-cursus Week 2}
% \author[\TeX niCie]{\TeX niCie\\{\tiny (Vincent Kuhlmann)}}
\author{\TeX niCie}
\date{5 oktober 2023}

\begin{document}

\lang{
    \section{Introduction}
}{
    \section{Introductie}
}

\begin{frame}
    \titlepage
    \centering

    {\Large\lang,Slides are available at,Slides zijn te vinden op,\\
    \href{https://texnicie.nl}{\ul{\texttt{texnicie.nl}}}}
\end{frame}

\setul{1pt}{2pt}

\begin{frame}
    \frametitle{\lang,Schedule,Agenda,}
    
    \begin{itemize}[label=\textbullet]
        \item \lang,More about text \& maths,Meer over tekst \& formules,
        \item \lang,References,Referenties,
        % \item \lang,`Theorem',`Theorem',
        % \item \lang,References,Referenties,
        \item $ \langle $\lang,Exercises!,Oefeningen!,$ \rangle $
        \item \lang,Figures,Figuren,
        \item \lang,Matrices,Matrices,
        \item \lang,Tables,Tabellen,
        % \item \lang,Formulas,Formules,
        \item $ \langle $\lang,Exercises!,Oefeningen!,$ \rangle $
        % \item \lang,Closing remarks,Afsluitende opmerkingen,
    \end{itemize}
\end{frame}

\section{Meer over tekst \& formules}\label{sec:basics}

\def\placetarget{\hypertarget{basics}{}}

% \begin{align*}
    
% \end{align*}

\importslide{text}{text-parskip.tex}
\importslide{internals}{internals-blank-lines-in-math.tex}
\importslide{text}{text-quotes.tex}

\importslide{document}{document-margins.tex}
\importslide{document}{document-toc.tex}
% \importslide{document}{document-hyperref.tex}

%\section{Referenties}

% \importslide{document}{document-toc.tex}

\importslide{math}{also_in_use.tex}

% \begin{frame}
%     Alternatieve notaties voor inline / display wiskunde modus
% \end{frame}

\importslide{math}{text.tex}
% \begin{frame}
%     Sin vs backslash sin,
%     Ftot vs F{text{tot}}
% \end{frame}

\section{Referenties}\label{sec:referenties}

\def\placetarget{\hypertarget{referenties}{}}

\importslide{math}{align-label.tex}

% \section{`Theorem'}\label{sec:theorem}

% \def\placetarget{\hypertarget{theorem}{}}

\importslide{math}{theorem-proof.tex}

\section{Oefeningen}\label{sec:oefeningen1}

\def\placetarget{\hypertarget{oefeningen1}{}}

\bgroup
\def\showformula#1#2{#1 & \hll|$ #2 $|}%
\begin{frame}[fragile]
    \begin{center}
        {\LARGE Oefeningen!}
        \vspace{10pt}

        \renewcommand{\arraystretch}{1.5}%
        \begin{tabularx}{0.55\textwidth}{ll}
            \toprule
            \lang,Formula,Formule,& Code\\
            \midrule
            % $ \sum_{i=0}^{\infty} $ & \mintinline{tex}{$ \sum_{i=0}^{\infty} $}\\
            \showformula{$ \sum_{i=0}^{\infty} $}{\\sum_\{i=0\}^\{\\infty\}}\\
            \showformula{$ \lim_{x\to\pi}\sin(x) $}{\\lim_\{x\\to\\pi\}\\sin(x)}\\
            \showformula{$ \od{f(x)}{x} $}{\\dod\{f(x)\}\{x\}}\\
            \showformula{$ \int_{a}^{b}f(x)\dif x $}{\\int_\{a\}^\{b\}f(x)\\dif x}\\
            % \showformula{$ \alpha,\beta,\gamma $}{\\alpha,\\beta,\\gamma}\\
            % \showformula{$ \epsilon,\varepsilon $}{\\epsilon,\\varepsilon}\\
            % \showformula{$ \phi,\varphi $}{\\phi,\\varphi}\\
            $ \left(\adjustbox{left=10pt,set height=13pt,valign=b,bgcolor=black!10!white}{}\right) $
            &
            \hll|$ \\left(|
            \adjustbox{left=10pt,set height=10pt,raise={-2pt},bgcolor=black!10!white}{}
            \hll|\\right) $|\\
            \bottomrule
        \end{tabularx}%
        \begin{tabularx}{0.4\textwidth}{ll}
            \toprule
            \lang,Formula,Formule,& Code\\
            \midrule
            \showformula{$ 5\cdot 6 $}{5\\cdot 6}\\
            %\showformula{$ x_1 + \dots + x_n $}{x_1 + \\dots + x_n}\\
            % \showformula{$ A,B,\Gamma $}{A,B,\\Gamma}\\
            \showformula{$ \norm{\mathbf{x}}_2 $}{\\norm\{\\mathbf(x)\}_2}\\
            % \showformula{$ \lfloor x\rfloor $}{\\norm\{\\mathbf(x)\}}\\
            % \showformula{$ \mathcal{P} $}{\\mathcal\{P\}}\\
            \showformula{$ \mathbb{R}^3 $}{\\mathbb\{R\}^3}\\
            %\showformula{$ f(\{3, 6\}) $}{f(\\\{3, 6\\\})}\\
            % \showformula{$ \uparrow,\lfloor $}{\\uparrow,\\lfloor}\\
            % \showformula{$ \oint,\oiint $}{\\oint, \\oiint}\\
            \\
            $ \left. \adjustbox{left=10pt,set height=13pt,valign=b,bgcolor=black!10!white}{}\right| $
            &
            \hll|$ \\left.|
            \adjustbox{left=10pt,set height=10pt,raise={-2pt},bgcolor=black!10!white}{}
            \hll~\\right| $~\\
            \bottomrule
        \end{tabularx}%

        % Vergeet niet de nodige packages toe te voegen.
        
        % Op mijn site
        % staat een basisdocument met alle nodige packages erin:

        % \href{https://vkuhlmann.com/latex/example}{\nolinkurl{vkuhlmann.com/latex/example}}
    \end{center}
\end{frame}
\egroup

%\section{Referenties}


\section{Figuren}\label{sec:figuren}

\def\placetarget{\hypertarget{figuren}{}}

\importslide{images}{includegraphics-center.tex}
\importslide{images}{figure.tex}
%\importslide{images}{figure-label.tex}
\importslide{images}{figure-placement.tex}
\importslide{images}{figure-dimensions-easy.tex}
\importslide{images}{subfigure.tex}

\section{Matrices}\label{sec:matrices}

\def\placetarget{\hypertarget{matrices}{}}

% \setDetail{10}
% \importslide{math}{align-star.tex}
% \setDetail{20}

\importslide{math}{matrices-simple.tex}
\importslide{math}{matrices-others.tex}
\importslide{math}{matrix-array-cases.tex}

\section{Tabellen}\label{sec:tabellen}

\def\placetarget{\hypertarget{tabellen}{}}

\importslide{tables}{table-demo1.tex}

\importslide{tables}{table-simple.tex}
%\importslide{tables}{table-simple-float.tex}
%\importslide{tables}{table-advanced-nofloat.tex}
\importslide{tables}{table-advanced.tex}
\importslide{tables}{table-tabularx.tex}
\import{../../slides/thomas/Eerstejaars-2023/}{lists.tex}
\import{../../slides/thomas/Eerstejaars-2023/}{lists-2.tex}

\section{Oefeningen}\label{sec:oefeningen2}

\def\placetarget{\hypertarget{oefeningen2}{}}

\begin{frame}
    \begin{center}
        {\LARGE Oefeningen!}
        \vspace{30pt}

        % Voor Oefening 10, herinner dat
        % \begin{align*}
        %     \begin{array}{r|rl}
        %         8 & 4 & 2\\
        %         \hline
        %         7
        %     \end{array}
        % \end{align*}

        % Vergeet niet de nodige packages toe te voegen.
        
        % Op mijn site
        % staat een basisdocument met alle nodige packages erin\footnote{Enkel
        % booktabs staat er niet in, want tabellen gebruik je niet standaard in je
        % inleveropgaves.}:

        % \href{https://vkuhlmann.com/latex/example}{\nolinkurl{vkuhlmann.com/latex/example}}
    \end{center}
\end{frame}

\section{Afsluiting}

%\importslide{}{}



% \subfile{1_Inleiding.tex}

% \subfile{2_Tekstopmaak.tex}

% \subfile{3_Documentstructuur.tex}

% \subfile{4_Afbeeldingen.tex}

% \subfile{5_Formules.tex}

% % \subfile{6_Bibliografie.tex}

% \subfile{7_GoedOmTeWeten.tex}

\subsection{Licentie}
\importslide{misc}{misc-license.tex}
    
\end{document}
