\documentclass[allauthors]{../../cursuspresentatie}

\copyrightTim

\def\importslide#1#2{%
	\import{../../slides/#1}{#2}
}

\title{GSNS \LaTeX{} course}
\author{\TeX niCie}
\date{8 September 2022}

% Als je het bestand dumpt naar een format, worden packages hierna niet meegedumpt maar elke keer
% fris ingeladen
\csname endofdump\endcsname

\usepackage{minted} %kan later in de .cls file hierboven
\setminted[tex]{fontsize=\small, autogobble=true, linenos=true, frame=single, escapeinside=||}
\usepackage{tcolorbox}
% \usepackage{}
\definecolor{darkgreen}{rgb}{0.2,0.7,0.11}
\makeatletter
\patchcmd{\@thm}{\thm@headfont{\scshape}}{\thm@headfont{\scshape\bfseries}}{}{}
\patchcmd{\@thm}{\thm@notefont{\fontseries\mddefault\upshape}}{}{}{}
\makeatother
\newtheorem{mylemma}[]{Lemma}
\begin{document}

\section{Introduction}
\importslide{beginners}{welcome.tex}
%\importslide{beginners}{contents.tex}

\begin{frame}
	\frametitle{\lang,Schedule,Agenda,}
	
	\begin{itemize}
		\item \lang,Introduction to LaTeX and Overleaf,Introductie tot LaTeX en Overleaf,
		\item Core concepts
		\item Text documents
		\item Math
		\item Closing remarks
		% \item \lang,Text formatting,Tekstopmaak,
		% \item  $ \langle $\lang,Exercises!,Oefeningen!,$ \rangle $
		% \item \lang,Structure of a document,Documentstructuur,
		% \item $ \langle $\lang,Exercises!,Oefeningen!,$ \rangle $
		% \item \lang,Formulas,Formules,
		% \item $ \mathbf\langle $\lang,Exercises!,Oefeningen!,$ \rangle $
		% \item  \lang,Images,Afbeeldingen,
		% \item $ \mathbf\langle $\lang,Exercises!,Oefeningen!,$ \rangle $
		% \item \lang,Closing remarks,Afsluitende opmerkingen,
	\end{itemize}
\end{frame}

\importslide{beginners}{overleaf-1.tex}
\importslide{beginners}{overleaf-2.tex}

% We switchen naar Overleaf voor Live LaTeX : laat zien waar de compileerknop zit en waar je tekst kan typen.
% ook laten zien dat er een file explorer is.

\importslide{beginners}{simpledoc-1.tex}
\importslide{beginners}{simpledoc-2.tex}
\importslide{beginners}{simpledoc-3.tex}

% simpel document met wat tekst zoals op slide simpledoc-3 namaken. Op compileren klikken en pdf laten zien.
% laten zien wat er gebeurt als je iets fouttypt, dat er dan een error verschijnt.

\importslide{beginners}{commands-1.tex}
\importslide{beginners}{commands-2.tex}
\importslide{beginners}{commands-3.tex}
% switch naar live LaTeX : demonstreer gebruik van de commands van zojuist. leg uit welke in de preamble gaan en 
% welke in de body.

\importslide{beginners}{whitespace.tex}
% tell that quad is as wide a an uppercase M.

\importslide{beginners}{paragraphs-1.tex}
\importslide{beginners}{paragraphs-2.tex}

% # EXCERCISES 10 MIN

% switch

\importslide{beginners}{sections.tex}

% switch

\importslide{beginners}{title.tex}

% switch

\importslide{beginners}{special-characters.tex}

% switch
\importslide{beginners}{format-text-1.tex}

\importslide{beginners}{logical-formatting.tex}

% # EXCERCISES 15 MIN

% switch
\importslide{beginners}{math-1.tex}

\importslide{beginners}{math-2.tex}

% # EXCERCISES 10 MIN

% switch
\importslide{beginners}{math-3.tex}

\importslide{beginners}{math-4.tex}

% # EXCERCISES 10 MIN

\importslide{beginners}{closing-remarks-1.tex}
\importslide{beginners}{closing-remarks-2.tex}

% boek van Stefan Kottwitz LaTeX for beginners 2nd ed aanraden.
% onze mailinglijst voor vervolgcursussen laten rondgaan, ook voor thesis cursus.
% laten weten dat er direct hierna een LaTeX intermediate cursus is.
% ook gratis boek zoeken om aan te raden.
	
\importslide{misc}{misc-license.tex}

\end{document}
