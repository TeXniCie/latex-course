%% !TeX program = pdflatex

\documentclass[
    % english=false,
    dutch,
    everyoneauthor=true,
    %darktheme,
    defaultSlideCollection=vincent,
    % handout
]{../../cursuspresentatie}

\usepackage{minted}


\setbeamertemplate{headline}
{%
    \placetarget
    \let\placetarget\relax
    \begin{beamercolorbox}[colsep=1.5pt]{upper separation line head}%
    \end{beamercolorbox}
    \begin{beamercolorbox}{section in head/foot}
        \vskip2pt
        \usebeamerfont{section in head/foot}\usebeamercolor[fg]{section in head/foot}%
        \color{fg!80!bg}
        % \adjustbox{
        %     set depth=2pt,
        %     padding={5pt 5pt 5pt 5pt},
        %     bgcolor=bg!80!orange,
        %     rndframe={color=red!20!orange}{0pt 3pt 0pt 3pt},
        %     margin={0pt 0pt 10pt 0pt}
        % }{%
        %     \hyperlink{oefeningen1}{\customhead{5}{Oefeningen}}%
        % }%
        \adjustbox{
            set depth=2pt,
            padding={5pt 5pt 5pt 5pt},
            rndframe={color=navigationTileOutline,width=1pt}{3pt 0pt 3pt 0pt},margin={10pt 0pt 0pt 0pt}
        }{%
            \hyperlink{eerste-document}{\customhead{2}{Eerste document}}%
            % \hspace{5pt}$ \cdot $\hspace{5pt}%
            % \hyperlink{figures}{\customhead{5}{Figures}}%
        }%
        \adjustbox{
            set depth=2pt,
            padding={5pt 5pt 5pt 5pt},
            %bgcolor=bg!80!orange,
            %bgcolor=structure.bg!80!orange,
            bgcolor=exerciseTileBackground,
            rndframe={color=exerciseTileOutline,width=1pt}{0pt 3pt 0pt 3pt},
            margin={-1pt 0pt 10pt 0pt}
        }{%
            \hyperlink{oefeningen1}{\customhead{3}{Oefeningen}}%
        }%
        \adjustbox{
            set depth=2pt,
            padding={5pt 5pt 5pt 5pt},
            rndframe={color=navigationTileOutline,width=1pt}{3pt 0pt 3pt 0pt},margin={10pt 0pt 0pt 0pt}
            % rndcorners={3pt 0pt 3pt 0pt}
        }{%
            \hyperlink{tekst}{\customhead{4}{Tekst}}%
            % \hspace{5pt}$ \cdot $\hspace{5pt}%
            % \hyperlink{figures}{\customhead{5}{Figures}}%
            % \hspace{5pt}$ \cdot $\hspace{5pt}%
            % \hyperlink{tabellen}{\customhead{8}{Tabellen}}%
        }%
        \adjustbox{
            set depth=2pt,    
            padding={5pt 5pt 5pt 5pt},
            bgcolor=exerciseTileBackground,
            rndframe={color=exerciseTileOutline,width=1pt}{0pt 3pt 0pt 3pt},
            margin={-1pt 0pt 10pt 0pt}
        }{%
            \hyperlink{oefeningen2}{\customhead{5}{Oefeningen}}%
        }
        \adjustbox{
            set depth=2pt,
            padding={5pt 5pt 5pt 5pt},
            rndframe={color=navigationTileOutline,width=1pt}{3pt 0pt 3pt 0pt},margin={10pt 0pt 0pt 0pt}
            % rndcorners={3pt 0pt 3pt 0pt}
        }{%
            \hyperlink{wiskunde}{\customhead{6}{Wiskunde}}%
            % \hspace{5pt}$ \cdot $\hspace{5pt}%
            % \hyperlink{figures}{\customhead{5}{Figures}}%
            % \hspace{5pt}$ \cdot $\hspace{5pt}%
            % \hyperlink{tabellen}{\customhead{8}{Tabellen}}%
        }%
        \adjustbox{
            set depth=2pt,    
            padding={5pt 5pt 5pt 5pt},
            bgcolor=exerciseTileBackground,
            rndframe={color=exerciseTileOutline,width=1pt}{0pt 3pt 0pt 3pt},
            margin={-1pt 0pt 10pt 0pt}
        }{%
            \hyperlink{oefeningen3}{\customhead{7}{Oefeningen}}%
        }
        \adjustbox{
            set depth=2pt,
            padding={5pt 5pt 5pt 5pt},
            rndframe={color=navigationTileOutline,width=1pt}{3pt 0pt 3pt 0pt},margin={10pt 0pt 0pt 0pt}
        }{%
            % \hyperlink{tables}{\customhead{7}{Tables}}%
            % \hspace{5pt}$ \cdot $\hspace{5pt}%
            \hyperlink{afsluiting}{\customhead{8}{Afsluiting}}%
        }%
        \hfil
        Slides staan op texnicie.nl
        \hfil
        \vskip4pt
    \end{beamercolorbox}%
    \begin{beamercolorbox}[colsep=1.5pt]{lower separation line head}%
    \end{beamercolorbox}%
}

\addbibresource{fakebib.bib}

% \copyrightThomas
% \copyrightVincent

\title[\LaTeX{} scriptiecursus]{\LaTeX{} scriptiecursus}
\author[\TeX niCie]{\TeX niCie}%\\{\scriptsize Presenters: Thomas \& Vincent}}
% \author[\TeX niCie]{Thomas \& Vincent ()}
\date{25 maart 2024}

\begin{document}

\def\placetarget{\hypertarget{introduction}{}}
\section{Introduction}

\importslide{tim/beginners}{welcome.tex}


\begin{frame}
    \frametitle{\lang,Schedule,Agenda,}
    
    \begin{itemize}[label=\textbullet]
        \item Scriptietemplate
        % \item Text document
        \item $ \langle $ Uitproberen $ \rangle $
        \item Bibliografie
        \item Referenties
        \item Afbeeldingen
        % \item Formulas
        % \item Figures
        % \item $ \langle $\lang,Exercises,Oefeningen,$ \rangle $
        % \item Lists and Tables
        % \item Finishing notes
    \end{itemize}
\end{frame}

\section{Scriptietemplate}
\begin{frame}
    Demo op Overleaf
\end{frame}


\section{Exercises}
\def\placetarget{\hypertarget{exercises1}{}}

\begin{frame}
    \begin{center}
        {\LARGE \lang,Try out,Uitproberen,}
        \vspace{30pt}

        % Vergeet niet de nodige packages toe te voegen.
        
        % Op mijn site
        % staat een basisdocument met alle nodige packages erin:

        % \href{https://vkuhlmann.com/latex/example}{\nolinkurl{vkuhlmann.com/latex/example}}

        % {\Large\lang,Slides and exercises are available at,Slides zijn te vinden op,\\
        %  \href{https://texnicie.nl}{\ul{\texttt{texnicie.nl}}}}
        {\Large Scriptietemplate is te vinden op\\
         \href{https://texnicie.nl}{\ul{\texttt{texnicie.nl}}}}
    \end{center}
\end{frame}

% \begin{frame}
%     \titlepage
%     \centering

%     {\Large\lang,Slides are available at,Slides zijn te vinden op,\\
%     \href{https://texnicie.nl}{\ul{\texttt{texnicie.nl}}}}
% \end{frame}

% \setul{1pt}{2pt}

\section{Bibliografie}
\def\placetarget{\hypertarget{bibliografie}{}}

\importslide{vincent/bib}{bib-ref_cmds}
\importslide{vincent/bib}{bib-printbibliography}
\importslide{vincent/bib}{bib-bibentry}
\importslide{vincent/bib}{bib-structure}
\importslide{vincent/bib}{bib-structure-bibtex}

\begin{frame}
    \centering
    De \texttt{.bib} code hoef je bijna nooit zelf te schrijven!

    \bigskip
    \raggedright
    \begin{itemize}[label=\textbullet]
        \item Op meeste wetenschappelijke websites voor artikels en boeken heb je
    een `export citation' functie. Kies hierbij voor type BibTeX.

        \item Of gebruik een citation manager of browser extension die je hiermee
    helpt.
    \end{itemize}
    
    % Finding bib code from article sites
\end{frame}
% \importslide{vincent/bib}{}
\importslide{vincent/bib}{bib-cite}
\importslide{vincent/bib}{bib-style}
\importslide{vincent/bib}{bib-backend}

% \begin{frame}
%     BibTeX+natbib instead of Biber+biblatex
% \end{frame}

\importslide{vincent/bib}{bib-mult_authors}

\importslide{vincent/bib}{bib-sort}
\importslide{vincent/bib}{bib-specialchars}

\importslide{vincent/bib}{bib-misc}

\section{Referenties}\label{sec:textdocument}
\def\placetarget{\hypertarget{referenties}{}}
% \def\placetarget{\hypertarget{referenties}{}}

\importslide{math}{align-label.tex}
% \importslide{vincent/}

% \begin{frame}
%     Referenties
% \end{frame}

% \importslide{vincent/}

\begin{frame}[fragile]{Referenties}
    \begin{codebox}
        \begin{minted}{tex}
            \section{Methods}
            \label{sec:methods}

            ...
            \section{Results}

            This result was obtained with the experiment as
            described in Section~\ref{sec:methods} on page \pageref{sec:methods}.
        \end{minted}
    \end{codebox}
\end{frame}

\section{Afbeeldingen}\label{sec:afbeeldingen}
\def\placetarget{\hypertarget{afbeeldingen}{}}

% \begin{frame}
%     Afbeeldingen
% \end{frame}

% \def\placetarget{\hypertarget{figuren}{}}

\importslide{images}{includegraphics-center.tex}
\importslide{images}{figure.tex}
%\importslide{images}{figure-label.tex}
\importslide{images}{figure-placement.tex}
\importslide{images}{figure-dimensions.tex}
\importslide{images}{subfigure.tex}


% \section{Exercises}
% \def\placetarget{\hypertarget{exercises2}{}}

% \begin{frame}
%     \begin{center}
%         {\LARGE \lang,Exercises!,Oefeningen!,}
%         \vspace{30pt}

%         {\Large\lang,Slides and exercises are available at,Slides zijn te vinden op,\\
%          \href{https://texnicie.nl}{\ul{\texttt{texnicie.nl}}}}
%     \end{center}
% \end{frame}


% \importslide{tim/beginners}{overleaf-1.tex}
% \importslide{tim/beginners}{overleaf-2.tex}

% \importslide{vincent/internals}{command-syntax-beginners.tex}
% \importslide{vincent/document}{document-parts-overleaf.tex}

% \importslide{vincent/internals}{internals-usepackage.tex}

% \importslide{vincent/text}{text-syntax-1.tex}

% \setDetail{10}
% \importslide{vincent/text}{text-specialchars.tex}
% \importslide{vincent/text}{text-quotes.tex}

% \importslide{vincent/text}{text-effects.tex}
% % \importslide{vincent/text}{text-lists.tex}

% \importslide{vincent/document}{document-margins.tex}
% \importslide{vincent/document}{document-sections.tex}
% \importslide{vincent/document}{document-toc.tex}

% \setDetail{20}



% \section{Formulas}
% \def\placetarget{\hypertarget{formulas}{}}
% \importslide{tim/beginners}{math-1.tex}
% \importslide{tim/beginners}{math-2.tex}
% \importslide{tim/beginners}{math-3.tex}
% \importslide{tim/beginners}{math-4.tex}
% \importslide{tim/beginners}{math-5.tex}


% \section{Figures}
% \def\placetarget{\hypertarget{figures}{}}

% % \importslide{vincent/images}{includegraphics-inline.tex}
% % \importslide{vincent/images}{includegraphics-asparagraph.tex}
% \importslide{vincent/images}{includegraphics-center.tex}
% \importslide{vincent/images}{includegraphics-figure.tex}
% \importslide{vincent/images}{figure-placement.tex}
% \importslide{vincent/images}{figure-dimensions.tex}
% \importslide{vincent/images}{subfigure.tex}




% \section{Lists and Tables}
% \def\placetarget{\hypertarget{tables}{}}

% \importslide{thomas/GSNS-2023-02-GSNS}{lists.tex}
% \importslide{thomas/GSNS-2023-02-GSNS}{lists-2.tex}
% \importslide{thomas/GSNS-2023-02-GSNS}{nested-lists.tex}
% \importslide{thomas/GSNS-2023-02-GSNS}{tables.tex}
% \importslide{thomas/GSNS-2023-02-GSNS}{tables-2.tex}
% \importslide{thomas/GSNS-2023-02-GSNS}{tables-3.tex}

\section{Beamer}\label{sec:beamer}
\def\placetarget{\hypertarget{beamer}{}}

\importslide{thomas/Scriptiecursus-2024}{beamer-1.tex}
\importslide{thomas/Scriptiecursus-2024}{beamer-2.tex}
\importslide{thomas/Scriptiecursus-2024}{beamer-3.tex}

\section{Finishing notes}
\def\placetarget{\hypertarget{finishingnotes}{}}

\importslide{vincent/misc}{misc-installation.tex}

% \importslide{vincent/misc}{misc-installation.tex}

\importslide{vincent/misc}{misc-contact.tex}

\section{License}
\importslide{misc}{misc-license.tex}


    
\end{document}

