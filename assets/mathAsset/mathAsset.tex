\documentclass{article}

\usepackage{geometry}
\usepackage[dutch]{babel}
\usepackage{parskip}
\usepackage{amsmath,amssymb}

\geometry{
	paperwidth=9cm,
	paperheight=7.5cm,
	margin=0.0cm,
	paperheight=3cm,
	paperheight=2cm,
    %paperwidth=4cm,
    %paperheight=4cm
}

\begin{document}
    % De trigonometrische identiteit is
    % $ \sin^2(\theta) + 
	% 	\cos^2(\theta) = 1. $

    % De trigonometrische identiteit is 
    % \begin{equation}
    %     \sin^2(\theta) + \cos^2(\theta) = 1.
    % \end{equation}

    De verdubbelingsformule herschrijven we nu als
    \begin{align}
        \cos(2\theta) = \cos^2(\theta)-\sin^2(\theta)\\
        = 2\cos^2(\theta)-1.
    \end{align}

    % Daarmee kunnen we de verdubbelingsformule herschrijven als
    % \begin{align}
    %     \cos(2\theta) &= \cos^2(\theta)-\sin^2(\theta)\\
    %     &= 2\cos^2(\theta)-1.
    % \end{align}

    % De verdubbelingsformule herschrijven we nu als
    % \begin{align*}
    %     \cos(2\theta) &= \cos^2(\theta) - \sin^2(\theta)\\%\nonumber\\
    %     &= 2\cos^2(\theta)-1.\tag{Alt. verd. form.}
    % \end{align*}

    % AA \(\sqrt{2}\)
    % BB \[\sqrt{3}\]
    % CC $$ \sqrt{4} $$

    % Dit doen we met de verdubbelingsformule
    % \begin{align*}
    %     \cos(2\theta) &= \cos^2(\theta) - \sin^2(\theta),
    % % \end{align*}
    % % die we kunnen herschrijven als
    % % \begin{align*}
    % \intertext{die we kunnen herschrijven als}
    %     &= \cos^2(\theta) - (1 - \cos^2(\theta))\\
    %     &= 2\cos^2(\theta)-1.
    % \end{align*}
\end{document}
