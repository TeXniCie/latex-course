%%% This is a file to test and show references. Some of the references will change when you change the settings, some won't.

%% The main ways of citation are:
% \cite{***}, using your normal settings.
% \textcitet{***}, made to fit as part of a sentence.
% \parencite{***}, puts your normal cite between parentheses.
%% Less used:
% \autocite{***}, automatically chooses one of the above (will make some mistakes).
% \footcite{***}, puts the normal cite in a footnote.
% \fullcite{***}, when you want all information citet.

\documentclass[thesis]{subfiles}

\begin{document}

%%% Based on an original piece by Laurens Stoop
\section{A sample section with many citations}

\subsection{Here I refer to some stuff}
Here I first cite an inbook \cite{inbook}. I can also cite in a different matter\footnote{Disclaimer: this only works for certain citing styles, see the preamble.}, e.g. \textcite{inbook}. I will now cite an master thesis \cite{mastersthesis}, and even though the citation command is the same, the style might change (depending on your cite settings). Just look at all these mad citations referring to a booklet \cite{booklet} and its author \parencite{booklet}, a conference \cite{conference} and an article \cite{article}. You can do even more that this! 

\subsection{Here I refer to some other stuff}
For in a collection \cite{incollection} you can refer to a manual \cite{manual} or a book \cite{book} or a thesis written by a PhD student \cite{phdthesis} or a master student \cite{mastersthesis} and if that is not enough for you, then you can even cite some miscellaneous stuff \cite{misc}. A technical report on something irrelevant \cite{techreport} can also be cited, just like the proceeding for citations \cite{proceedings} and if you are still unpublished \cite{unpublished} you can even refer to that stuff.\\
Finally I refer multiple things at once: like this \cite{misc,phdthesis} or this \cite{incollection,conference,techreport} or, if you are introducing many new refs at once, like this \cite{inbook,mastersthesis,booklet,conference} . As a final thing, I refer to the previous section, section \ref{sec:theorems}.

Also remark that the autocite (which will change to what it thinks is best) \autocite{reference} and footcite \footcite{reference} commands exist.
\bigbreak
And if you really want to show everything? Use  `\textbackslash fullcite':\newline  \fullcite{article}

\end{document}

